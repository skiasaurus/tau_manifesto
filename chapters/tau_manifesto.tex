\section{La constante du cercle} % (fold)
\label{sec:the_circle_constant}

\href{https://tauday.com/tau-manifesto}{\emph{Le Manifeste de tau}} est dédié
à l'un des nombres les plus importants en mathématiques, peut-être \emph{le}
plus important~: la \emph{constante du cercle}, qui relie la circonférence d'un
cercle à sa dimension linéaire. Depuis des millénaires, le cercle est considéré
comme la plus parfaite des formes, et la constante du cercle capture la
géométrie du cercle dans un seul nombre. Bien sûr, le choix traditionnel pour la
constante du cercle est $\pi$\ns; mais, comme le note le mathématicien
\href{https://www.math.utah.edu/~palais/}{Bob Palais} dans son superbe article
«\ns $\pi$ Is Wrong!\ns » \footnote{Palais, Robert. «\ns $\pi$ Is Wrong!\ns », \emph{The
Mathematical Intelligencer}, volume~23, numéro~3, 2001, pages~7--8. De nombreux
des arguments du \emph{Manifeste de tau} sont basés sur ou sont inspirés par
«\ns $\pi$ Is Wrong!\ns ». Il est disponible en ligne sur
\href{https://www.math.utah.edu/~palais/pi.html}{https://www.math.utah.edu/~palais/pi.html}.},
$\pi$ \emph{est mauvais}. Il est temps de régler les choses.

  \subsection{Immodeste Proposition} % (fold)
  \label{sec:an_immodest_proposal}

On peut commencer à réparer les dommages causés par $\pi$ en comprenant le nombre
notoire lui-même. La définition traditionnelle pour la constante du cercle
définit $\pi$ (pi) comme égal au rapport de la circonférence d'un cercle
(longueur) à son diamètre (largeur)\ns\footnote{Le symbole $\equiv$ signifie
«\ns défini comme\ns ».}.
\begin{equation}
\label{eq:pi}
\pi \equiv \frac{C}{D} = 3{,}141\,592\,65\ldots
\end{equation}
Le nombre $\pi$ a de nombreuses propriétés remarquables (entre autres, il est
\href{https://fr.wikipedia.org/wiki/Nombre_irrationnel}{\emph{irrationnel}} et
même \href{https://fr.wikipedia.org/wiki/Nombre_transcendant}{\emph{transcendant}}) et
sa présence dans les formules mathématiques est répandue.

\begin{figure}
\image{images/figures/circle.pdf}
\caption{Anatomie d'un cercle.\label{fig:circle}}
\end{figure}

Il va sans dire que $\pi$ n'est pas «\ns mauvais\ns » dans le sens où il
est factuellement incorrect\ns; le nombre $\pi$ est parfaitement bien défini,
et il possède toutes les propriétés qui lui sont normalement attribuées par
les mathématiciens. Quand on dit que «\ns $\pi$ est mauvais\ns », on veut dire que
\emph{$\pi$ est un choix déroutant et contre nature pour la constante du
cercle}. En particulier, un cercle est défini comme l'ensemble des points à
une distance fixe, le \emph{rayon}, d'un point déterminé, le \emph{centre} (figure~\ref{fig:circle}).
Tandis qu'il existe une infinité de formes avec une
largeur constante (figure~\ref{fig:constant_width})\ns\footnote{Image
récupérée de
\href{https://commons.wikimedia.org/wiki/File:Reuleaux_triangle_roll.gif}{Wikimedia}
le 12 mars 2019. Copyright © 2016 par Ruleroll et utilisé sans
modification selon les termes de la licence
\href{https://creativecommons.org/licenses/by-sa/4.0/deed.fr}{Attribution -
Partage dans les Mêmes Conditions 4.0 International de Creative Commons}.}, il
n'y a qu'une seule forme avec un rayon constant. Cela suggère qu'une
définition plus naturelle de la constante du cercle pourrait utiliser $r$ à la
place de~$D$~:
\begin{equation}
\label{eq:circle_constant}
\mbox{constante du cercle} \equiv \frac{C}{r}.
\end{equation}
Parce que le diamètre d'un cercle est le double de son rayon, ce nombre est
numériquement équivalent à $2\pi$. Tout comme $\pi$, il est transcendant et donc
irrationnel, et (comme on le verra dans la section~\ref{sec:the_number_tau})
son utilisation en mathématiques est également répandue.

\begin{figure}
\image{images/figures/Reuleaux_triangle_roll.pdf}
\caption{L'une de l'infinité de formes non circulaires à largeur
constante.\label{fig:constant_width}}
\end{figure}

Dans «\ns $\pi$ Is Wrong!\ns », Bob Palais plaide de manière convaincante en faveur
de la seconde de ces deux définitions de la constante du cercle\ns; et, à mon
avis, il mérite la majorité du crédit pour avoir identifié ce problème et pour
l'avoir présenté à un large public. Il appelle la vraie constante du cercle «\ns un
tour\ns », et il introduit aussi un nouveau symbole pour la représenter
(figure~\ref{fig:palais_tau}). Comme on le verra, la description est presciente,
mais malheureusement, le symbole est plutôt étrange et (comme on en discute dans la
section~\ref{sec:conflict_and_resistance}) il semble peu probable qu'il soit
largement adopté un jour. (\emph{Mise à jour~:} Cela s'est avéré être le cas,
et depuis, Palais lui-même est devenu un fervent partisan des arguments de
ce manifeste.)

\begin{figure}
\imagebox{images/figures/palais-tau.png}
\caption{Le symbole étrange de la constante du cercle de «\ns $\pi$ Is Wrong!\ns ».\label{fig:palais_tau}}
\end{figure}

\emph{Le Manifeste de tau} est dédié à proposer la suivante~:  la bonne
réponse à «\ns $\pi$ est mauvais\ns » est «\ns Non, mais \emph{vraiment}\ns ». Et la vraie
constante du cercle mérite un nom approprié. Comme vous l'aurez probablement deviné,
\emph{Le Manifeste de tau} propose que ce nom soit la lettre grecque $\tau$
(tau)~:

\begin{equation}
\label{eq:tau}
\tau \equiv \frac{C}{r} = 6{,}283\,185\,307\,179\,586\ldots
\end{equation}
Tout au long du reste de ce manifeste, on verra que le \emph{nombre} $\tau$ est
le bon choix, et l'on montrera par l'usage (section~\ref{sec:the_number_tau}
et section~\ref{sec:circular_area}) et par l'argumentation directe
(section~\ref{sec:conflict_and_resistance}) que la \emph{lettre} $\tau$ est aussi
un choix naturel.

\subsection{Un ennemi puissant} % (fold)
 \label{sec:a_powerful_enemy}
 
Avant de procéder à la démonstration que $\tau$ est le choix naturel pour la
constante du cercle, reconnaissons d'abord ce à quoi nous devons faire face\ns; car il existe
une puissante conspiration, vieille de plusieurs siècles, déterminée a propager
la propagande pro-$\pi$. Des
\href{https://www.amazon.fr/fascinant-nombre-ESSAIS-SCIEN-ebook/dp/B07D3SYP5X/}{livres}
\href{https://www.amazon.fr/1-000-000-Décimales-Nombre-Plus-Connu/dp/B085DMM9XV/}{entiers}
\href{https://www.amazon.fr/Autour-du-nombre-pi-HR-ACT-SC-INDUS-ebook/dp/B081HGQSQJ/}{sont
écrits} qui vantent les vertus de $\pi$. (Sérieusement,
\href{https://www.amazon.com/exec/obidos/ISBN=0802713327/}{\emph{des
livres}\ns!}) Et la dévotion irrationnelle à $\pi$ s'est propagée même aux
plus hauts niveaux de la culture geek\ns; par exemple, le «\ns jour de pi\ns » en 2010,
\href{https://www.google.com/}{Google} \emph{a changé son logo} pour honorer
$\pi$ (figure~\ref{fig:google_pi_day.}).

\begin{figure}
\begin{center}
\image{images/figures/google_pi_day.png}
\end{center}
\caption{Le logo Google le 14 mars (écrit 3/14 aux États-Unis\ns; «\ns le Jour de
pi\ns ») 2010.\label{fig:google_pi_day.}}
\end{figure}

Pendant ce temps, certaines personnes mémorisent des dizaines, des centaines,
voire
\href{https://www.guinnessworldrecords.com/world-records/most-pi-places-memorised}{\emph{des
milliers}} de chiffres de ce nombre mystique. Qui-donc, sinon un raté, mémoriserait ne serait-ce
que 40 chiffres de $\pi$ (figure~\ref{fig:futurama_video})\ns\footnote{La vidéo de la
figure~\ref{fig:futurama_video} (disponible sur
\url{https://vimeo.com/12914981}) est un extrait d'une conférence donnée par
\href{https://cs.appstate.edu/~sjg/}{le Dr Sarah Greenwald}, professeur de
mathématiques à \href{https://www.appstate.edu/}{l'Appalachian State
University}. Le Dr Greenwald utilise des références mathématiques des
\emph{Simpson} et de \emph{Futurama} pour susciter l'interêt de ses élèves, et
pour les aider à surmonter leur anxiété mathématique. Elle est aussi la
responsable de la \href{https://cs.appstate.edu/~sjg/futurama/}{«\ns \emph{Futurama}
Math Page\ns »}.}\ns?

\begin{figure}
\begin{center}
%= insert_futurama_video
\includegraphics{images/figures/futurama_math_lecture.png} % html_ignore
\end{center}
\caption{\href{https://tauday.com/tau-manifesto/\#sec-about_the_author}{Michael
Hartl} donne tort à \href{https://fr.wikipedia.org/wiki/Matt_Groening}{Matt
Groening} en récitant 40 décimales de $\pi$.\label{fig:futurama_video}}
\end{figure}

Les partisans de $\tau$ font face à un puissant adversaire, cela va sans dire. Et
pourtant, nous avons un allié puissant~: car la vérité est de notre côté.

% section the_most_important_number (end)

\section{Le nombre tau} % (fold)
\label{sec:the_number_tau}

Nous avons vu dans la section~\ref{sec:an_immodest_proposal} que le nombre $\tau$ peut aussi s'écrire
$2\pi$. Comme Palais l'a remarqué dans «\ns $\pi$ Is Wrong!\ns », il est donc d'un grand
intérêt de découvrir que la combinaison $2\pi$ se produit avec une fréquence
étonnante dans l'ensemble des mathématiques. Par exemple, on considère les
intégrales sur tout l'espace en coordonnées polaires:

\[
  \int_0^{2\pi}\int_0^\infty f(r, \theta)\, r\, dr\, d\theta.
\]
La limite supérieure de l'intégration de $\theta$ est toujours $2\pi$. Le même
facteur apparaît dans la définition de la
\href{https://fr.wikipedia.org/wiki/Loi_normale}{loi gaussienne (normale)},
\[
  \frac{1}{\sqrt{2\pi}\sigma}e^{-\frac{(x-\mu)^2}{2\sigma^2}},
\]
et à nouveau dans la
\href{https://fr.wikipedia.org/wiki/Transformation_de_Fourier}{transformation de
Fourier},
\[
  f(x) = \int_{-\infty}^\infty F(k)\, e^{2\pi ikx}\,dk
\]
\[
    F(k) = \int_{-\infty}^\infty f(x)\, e^{-2\pi ikx}\,dx.
\]
Il revient dans la
\href{https://fr.wikipedia.org/wiki/Formule_intégrale_de_Cauchy}{formule
intégrale de Cauchy},
\[
  f(a) = \frac{1}{2\pi i}\oint_\gamma\frac{f(z)}{z-a}\,dz,
\]
dans les \href{https://fr.wikipedia.org/wiki/Racine_de_l%27unité}{racines
$n$ièmes de l'unité},
\[
  z^n = 1 \Rightarrow z = e^{2\pi i/n},
\]
et dans les valeurs de la
\href{https://fr.wikipedia.org/wiki/Fonction_zêta_de_Riemann}{fonction zêta de
Riemann} pour les entiers pairs positifs\ns\footnote{Ici $B_n$ est le $n$ième
\href{https://fr.wikipedia.org/wiki/Nombre_de_Bernoulli}{nombre de
Bernoulli}.}~:
\[
\begin{split}
  \zeta(2n) & = \sum_{k=1}^\infty \frac{1}{k^{2n}} \\
            & = \frac{|B_{2n}|}{2(2n)!}\,(2\pi)^{2n},\qquad n = 1, 2, 3, \ldots
\end{split}
\]
Ces formules sont loin d'être les seuls exemples~: prenez votre texte
préféré de physique ou de mathématiques et essayez vous-même. Il en existe
\href{http://www.harremoes.dk/Peter/Undervis/Turnpage/Turnpage1.html}{beaucoup
d'autres}, et la conclusion est claire~: il y a quelque chose de spécial dans
$2\pi$.

Pour aller au fond de l'affaire, il faut revenir aux principes fondamentaux en
considérant la nature des cercles, et surtout la nature des \emph{angles}. Bien
qu'il soit probable qu'une grande partie de ce contenu soit familier, cela vaut
la peine de le revoir, car c'est là que commence une véritable compréhension de
$\tau$.

  \subsection{Cercles et angles} % (fold)
  \label{sec:circles_and_angles}

Il y a une relation intime entre les cercles et les angles, comme le montre la
figure~\ref{fig:angle_arclength}. Puisque les cercles concentriques de la
figure~\ref{fig:angle_arclength} ont des rayons différents, les droites de la
figure coupent différentes longueurs d'arc, mais l'angle~$\theta$ (thêta) est le
même dans chaque cas. En d'autres termes, la taille de l'angle ne dépend pas du
rayon du cercle utilisé pour définir l'arc. La tâche principale d'une mesure
d'angle est de créer un système qui capture cette invariance de rayon.

\begin{figure}
\begin{center}
\image{images/figures/angle-arclength.pdf}
\end{center}
\caption{Un angle $\theta$ avec deux cercles
concentriques.\label{fig:angle_arclength}}
\end{figure}

L'unité de mesure d'angle la plus élémentaire est peut-être le \emph{degré}, qui
divise un cercle en 360 parties égales. Un resultat de ce système est
l'ensemble des angles spéciaux (familiers aux étudiants en trigonométrie)
montrés sur la figure~\ref{fig:degree_angles}.

\begin{figure}
\begin{center}
\image{images/figures/degree-angles.pdf}
\end{center}
\caption{Des angles spéciaux, en degrés.\label{fig:degree_angles}}
\end{figure}

Un système plus fondamental de mesure d'angle implique une comparaison directe
de la longueur d'arc $s$ avec le rayon $r$. Bien que les longueurs de la
figure~\ref{fig:angle_arclength} diffèrent, la longueur de l'arc augmente
proportionnellement au rayon, de sorte que le \emph{rapport} de la longueur de
l'arc au rayon est le même dans chaque cas~:
\[
s\propto r \Rightarrow \frac{s_1}{r_1} = \frac{s_2}{r_2}.
\]
Cela suggère la définition suivante de la \emph{mesure d'angle en radians}:
\begin{equation}
\label{eq:radians}
\theta \equiv \frac{s}{r}.
\end{equation}
Cette définition a la propriété requise d'invariance de rayon, et puisque $s$ et
$r$ ont tous deux des unités de longueur, les radians sont
\href{https://fr.wikipedia.org/wiki/Grandeur_sans_dimension}{\emph{sans
dimension}} par construction. L'utilisation de la mesure d'angle en radians conduit
à des formules succinctes et élégantes dans l'ensemble des mathématiques\ns; par
exemple, la formule habituelle pour la dérivée de $\sin\theta$ n'est vraie que
lorsque $\theta$ est exprimé en radians~:
\[
  \frac{d}{d\theta}\sin\theta = \cos\theta. \qquad\mbox{(seulement en radians)}
\]
Naturellement, les angles spéciaux de la figure~\ref{fig:degree_angles} peuvent
être exprimés en radians, et lorsque vous avez appris la trigonométrie au lycée, vous
avez probablement mémorisé les valeurs spéciales indiquées sur la
figure~\ref{fig:pi_angles}. (J'appelle ce système de mesure
«\ns radians-de-$\pi$\ns » pour souligner qu'ils sont écrits en termes de $\pi$.)

\begin{figure}
\begin{center}
\image{images/figures/pi-angles.pdf}
\end{center}
\caption{Des angles spéciaux, en radians-de-$\pi$.\label{fig:pi_angles}}
\end{figure}

\begin{figure}
\begin{center}
\image{images/figures/angle-fractions.pdf}
\end{center}
\caption{Les angles «\ns spéciaux\ns » sont des fractions d'un cercle
complet.\label{fig:angle_fractions}}
\end{figure}

Un instant de réflexion montre que les angles dits «\ns spéciaux\ns » ne sont que
des fractions rationnelles particulièrement simples d'un cercle complet, comme
le montre la figure~\ref{fig:angle_fractions}. Cela peut inspirer une révision de
l'équation~\eqref{eq:radians}, en réécrivant la longueur d'arc~$s$ en termes de
la fraction~$f$ de la circonférence complète~$C$, c'est-à-dire $s = f C$~:
\[ \theta = \frac{s}{r} = \frac{fC}{r} =  f\left(\frac{C}{r}\right) \equiv f\tau. \]
Vous remarquerez à quel point $\tau$ ressort naturellement de cette analyse. Si vous croyez en
$\pi$, je crains que le diagramme des angles spéciaux (figure~\ref{fig:tau_angles})
qui résulte de la manipulation ci-dessus bouleverse votre foi profondément.

\begin{figure}
\begin{center}
\image{images/figures/tau-angles.pdf}
\end{center}
\caption{Des angles spéciaux, en radians.\label{fig:tau_angles}}
\end{figure}

Bien qu'il existe de nombreux autres arguments en faveur de $\tau$, la
figure~\ref{fig:tau_angles} peut être l'un des plus frappants. On voit
aussi dans la figure~\ref{fig:tau_angles} le génie dont Bob Palais
a fait preuve en identifiant la constante du cercle comme
«\ns \href{https://fr.wikipedia.org/wiki/Tour_(angle)}{un tour}\ns »~: $\tau$ est la
mesure d'angle en radians d'un \emph{tour} de cercle. De plus, on remarque
qu'avec $\tau$ il n'y a \emph{rien à mémoriser}~: une douzième de tour est
$\tau/12$, un huitième de tour est $\tau/8$, et ainsi de suite. L'utilisation de
$\tau$ nous donne le meilleur des deux mondes en combinant clarté conceptuelle
avec tous les avantages concrets des radians\ns; la signification abstraite de,
disons, $\tau/12$ est évidente, mais c'est aussi juste un nombre~:
\[
\begin{split}
\mbox{un douzième de tour} = \frac{\tau}{12} & \approx \frac{6{,}283\,185}{12} \\
                                             & = 0{,}523\,598\,8.
\end{split}
\]
Enfin, en comparant la figure~\ref{fig:pi_angles} à la
figure~\ref{fig:tau_angles}, on voit d'où viennent ces fichus facteurs de
2$\pi$~: un tour de cercle vaut 1$\tau$, mais 2$\pi$. Numériquement, ils sont
égaux, mais conceptuellement, ils sont tout à fait distincts.

    \subsubsection{Les ramifications} % (fold)
    \label{sec:the_ramifications}

    % subsubsection the_ramifications (end)

Les facteurs de 2 inutiles qui résultent de l'utilisation de $\pi$ sont assez
ennuyeux en eux-mêmes, mais beaucoup plus grave est leur tendance à
\emph{s'annuler} lorsqu'ils sont divisés par un nombre pair. Les résultats
absurdes, comme un \emph{demi}-$\pi$ pour un \emph{quart} de tour, obscurcissent
la relation sous-jacente entre la mesure d'angle et la constante du cercle. À
ceux qui soutiennent que «\ns peu importe\ns » que l'on utilise $\pi$ ou $\tau$ pour
enseigner la trigonométrie, je leur demande simplement de visualiser la
figure~\ref{fig:pi_angles}, la figure~\ref{fig:angle_fractions} et la
figure~\ref{fig:tau_angles} à travers les yeux d'un enfant. Vous verrez que, du
point de vue d'un débutant, \emph{l'utilisation de $\pi$ au lieu de $\tau$ est
un désastre pédagogique}.

  \subsection{Les fonctions circulaires} % (fold)
  \label{sec:the_circle_functions}

La mesure d'angle en radians nous donne certains des arguments les plus
convaincants pour la vraie constante du cercle. Mais les vertus de $\pi$ et de $\tau$
méritent également d'être comparées dans d'autres contextes. On commence par
considérer les fonctions élémentaires importantes $\sin\theta$ et $\cos\theta$.
Appelées «\ns fonctions circulaires\ns » parce qu'elles donnent les
coordonnées d'un point sur le cercle unité (c'est-à-dire un cercle de rayon 1),
le sinus et le cosinus sont les fonctions fondamentales de la trigonométrie
(figure~\ref{fig:circle_functions}).

\begin{figure}
\begin{center}
\image{images/figures/circle-functions.pdf}
\end{center}
\caption{Les fonctions circulaires sont des coordonnées sur le cercle
unité.\label{fig:circle_functions}}
\end{figure}

Examinons les graphiques des fonctions circulaires pour mieux comprendre
leur comportement\footnote{Ces graphiques ont été produits avec l'aide de
\href{https://www.wolframalpha.com/}{Wolfram|Alpha.}}. On remarquera sur la
figure~\ref{fig:sine_with_tau} et la figure~\ref{fig:cosine_with_tau} que les
deux fonctions sont périodiques de période $T$. Comme le montre la
figure~\ref{fig:sine_with_tau}, la fonction sinus $\sin\theta$ commence à zéro,
atteint son maximum à un quart de période, passe par zéro à une demi-période,
atteint son minimum aux trois quarts de période, et revient à zéro après une
période complète. Pendant ce temps, la fonction cosinus $\cos\theta$ commence à
son maximum, atteint son minimum à une demi-période, et passe par zéro à un quart et
trois quarts de période (figure~\ref{fig:cosine_with_tau}).

\begin{figure}
\begin{center}
\image{images/figures/sine-with-tau.pdf}
\end{center}
\caption{Points importants pour $\sin\theta$ en termes de période
$T$.\label{fig:sine_with_tau}}
\end{figure}

\begin{figure}
\begin{center}
\image{images/figures/cosine-with-tau.pdf}
\end{center}
\caption{Points importants pour $\cos\theta$ en termes de période
$T$.\label{fig:cosine_with_tau}}
\end{figure}

Bien sûr, puisque le sinus et le cosinus passent tous les deux par un cycle
complet pendant un tour du cercle, on a $T = \tau$\ns; c'est-à-dire que les
fonctions circulaires ont des périodes égales à la constante du cercle. Par
conséquent, les valeurs «\ns spéciales\ns »  de $\theta$ sont tout à fait
naturelles~: un quart de période est $\tau/4$, une demi-période est $\tau/2$,
etc. En fait, en faisant la figure~\ref{fig:sine_with_tau}, à un moment donné,
je me suis retrouvé à me demander quelle était la valeur numérique de $\theta$
pour le zéro de la fonction sinus. Puisque le zéro se produit après une
demi-période, et puisque $\tau \approx 6{,}28$, un calcul mental rapide a
conduit au résultat suivant~:
\[
  \theta_\mathrm{zéro} = \frac{\tau}{2} \approx 3{,}14.
\]
Eh oui~: j'étais étonné de découvrir que \emph{j'avais déjà oublié que
$\tau/2$ était parfois appelé «\ns $\pi$\ns »}. Peut-être même que cela vient de vous
arriver. Bienvenue dans ma vie.

  % subsection the_circle_functions (end)

% section radian_angle_measure (end)

   \subsection{L'identité d'Euler} % (fold)
   \label{sec:euler_s_identity}

Je serais négligent de ne pas aborder dans ce manifeste \emph{l'identité
d'Euler}, parfois appelée «\ns la plus belle des équations mathématiques\ns ». Cette
identité implique \emph{l'exponentiation complexe}, qui est profondément liée, à
la fois, aux fonctions circulaires et à la géométrie du cercle lui-même.

Selon l'approche choisie, l'équation suivante peut être prouvée comme
théorème ou prise comme définition\ns; de toute façon, elle est vraiment remarquable~:
\begin{equation}
\label{eq:eulers_formula}
e^{i\theta} = \cos\theta + i\sin\theta. \qquad\mbox{La formule d'Euler}
\end{equation}
Connue sous le nom de \emph{formule d'Euler} (d'après
\href{https://en.wikipedia.org/wiki/Leonhard_Euler}{Leonhard Euler}), cette
équation relie une exponentielle à argument imaginaire aux fonctions circulaires
(sinus et cosinus) et à l'unité imaginaire~$i$. Bien que justifier la formule
d'Euler dépasse le cadre de ce manifeste, sa provenance est au-dessus de tout
soupçon et son importance est incontestable.

L'évaluation de l'équation~\eqref{eq:eulers_formula} à $\theta = \tau$ nous donne
\emph{l'identité d'Euler}\ns\footnote{Ici, je définis implicitement l'identité
d'Euler comme \emph{l'exponentielle complexe de la constante du cercle}, plutôt
que de la définir comme l'exponentielle complexe de n'importe quel nombre spécifique. Si l'on
choisit $\tau$ comme la constante du cercle, on obtient l'identité montrée.
Comme on le verra dans un moment, ce n'est pas la forme traditionnelle de l'identité,
qui implique bien sûr $\pi$\ns; mais la version avec $\tau$ est la déclaration
la plus mathématiquement significative de l'identité, donc je crois qu'elle
mérite ce nom.}~:
\begin{equation}
\label{eq:eulers_identity_tau}
e^{i\tau} = 1. \qquad\mbox{L'identité d'Euler (version de $\tau$)}
\end{equation}
En toutes lettres, l'équation~\eqref{eq:eulers_identity_tau} fait l'observation
fondamentale suivante~:

\begin{center}
\emph{L'exponentielle complexe de la constante du cercle est l'unité.}
\end{center}

Géométriquement, la multiplication par $e^{i\theta}$ correspond à la rotation
d'un nombre complexe d'un angle $\theta$ dans le plan complexe, ce qui suggère
une seconde interprétation de l'identité d'Euler~:

\begin{center}
\emph{Une rotation d'un tour est égale à 1.}
\end{center}

\noindent Puisque le nombre $1$ est
l'\href{https://fr.wikipedia.org/wiki/%C3%89l%C3%A9ment_neutre}{identité
multiplicative}, la signification géométrique de $e^{i\tau} = 1$ est que la
rotation d'un point du plan complexe par un tour le ramène simplement à sa
position d'origine.

Comme dans le cas de la mesure d'angle en radians, on voit à quel point
l'association entre $\tau$ et un tour d'un cercle est naturelle. En effet,
relier $\tau$ avec «\ns un tour\ns » fait que l'identité d'Euler
ressemble presque à une
tautologie\ns\footnote{\href{https://bit.ly/32mB2CF}{Techniquement}, tous les
théorèmes mathématiques sont des tautologies, mais ne soyons pas aussi
pédants.}.

    \subsubsection{Pas la plus belle des équations} % (fold)
    \label{sec:not_the_most_beautiful_equation}

Bien sûr, la forme traditionnelle de l'identité d'Euler est écrite en termes de
$\pi$ au lieu de $\tau$. Pour le dériver, on commence en évaluant la formule
d'Euler à $\theta = \pi$, ce qui nous donne
\begin{equation}
\label{eq:eulers_identity_pi}
e^{i\pi} = -1. \qquad\mbox{Identité d'Euler (version de $\pi$)}
\end{equation}
\noindent Mais ce signe négatif est si moche que
l'équation~\eqref{eq:eulers_identity_pi} est presque toujours réarrangée
immédiatement, nous donnant la «\ns belle\ns » équation suivante~:
\begin{equation}
\label{eq:eulers_pi_rearranged}
e^{i\pi} + 1 = 0. \qquad\mbox{(réarrangé)}
\end{equation}
À ce stade, le présentateur fait généralement une déclaration grandiose sur la façon
dont l'équation~\eqref{eq:eulers_pi_rearranged} relie $0$, $1$, $e$, $i$ et
$\pi$, parfois appelés les «\ns cinq nombres les plus importants en
mathématiques\ns ».

Dans ce contexte, il est remarquable de voir combien de personnes se plaignent
que l'équation~\eqref{eq:eulers_identity_tau} ne relie que \emph{quatre} de ces
cinq nombres. Donc voilà~:
\begin{equation}
\label{eq:euler_tau_zero}
e^{i\tau} = 1 + 0.
\end{equation}
L'équation~\eqref{eq:euler_tau_zero}, \emph{sans} réarrangement, relie en fait
les cinq nombres les plus importants en mathématiques~: $0$, $1$, $e$, $i$ et
$\tau$\ns\footnote{En effet, l'équation (6) peut s'écrire $e^{i\tau} = 1 + 0i$,
ce qui rend la relation entre les cinq nombres encore plus explicite.}.

      \subsubsection{Identités eulériennes} % (fold)
      \label{sec:eulerian_identities}

Puisque l'on peut ajouter zéro n'importe où dans n'importe quelle équation,
l'introduction de $0$ dans l'équation~\eqref{eq:euler_tau_zero} est un
contrepoint quelque peu ironique à $e^{i\pi} + 1 = 0$\ns; mais l'identité
$e^{i\pi} = -1$ a un argument plus sérieux à faire valoir. Voyons ce qui
se passe quand on la réécrit en termes de $\tau$~:
\[
e^{i\tau /2} = -1.
\]
Géométriquement, cela signifie qu'une rotation d'un demi-tour équivaut à une
multiplication par $-1$. Et, en effet, c'est le cas~: sous une rotation de
$\tau/2$ radians, le nombre complexe $z = a + ib$ est caractérisé par $-a - ib$, qui
est en fait juste $-1\cdot z$.

Écrite en termes de $\tau$, on voit que la forme «\ns originale\ns » de l'identité
d'Euler (équation~\eqref{eq:eulers_identity_pi}) a une signification
géométrique transparente qui lui fait défaut lorsqu'elle est écrite en termes de
$\pi$. (Bien sûr, $e^{i\pi} = -1$ peut être interprété comme une rotation par
$\pi$ radians, mais le réarrangement quasi-universel pour former $e^{i\pi} + 1 =
0$ montre comment l'utilisation de $\pi$ nous distrait de la signification
géométrique naturelle de l'identité.) Les identités en quart d'angle ont des
interprétations géométriques similaires~: en évaluant
l'équation~\eqref{eq:eulers_formula} à $\tau/4$, on obtient $e^{i\tau/4} = i$,
ce qui montre qu'un quart de tour dans le plan complexe équivaut à une multiplication
par $i$\ns; de même, $e^{i\cdot(3\tau/4)} = -i$ dit que les trois quarts de tour
équivalent à la multiplication par $-i$. Un résumé de ces résultats, que l'on
appellera les identités eulériennes, figure dans le
tableau~\ref{table:eulerian_identities}.

\begin{table}
\begin{center}
\begin{tabular}{cllr}
Angle de rotation & \multicolumn{3}{c}{Identité eulérienne} \\ \hline
$0$ & $e^{i\cdot0}$ & $ = $ & $1$ \smallskip \\
$\tau/4$ & $e^{i\tau/4}$ & $ = $ & $i$ \smallskip \\
$\tau/2$ & $e^{i\tau/2}$ & $ = $ & $-1$ \smallskip \\
$3\tau/4$ & $e^{i\cdot(3\tau/4)}$ & $ = $ & $-i$ \smallskip \\
$\tau$ & $e^{i\tau}$ & $ = $ & $1$
\end{tabular}
\end{center}
\caption{Identités eulériennes pour les rotations à demi, au quart, et
complète.\label{table:eulerian_identities}}
\end{table}

On peut pousser cette analyse un peu plus loin en notant que, pour n'importe
quel angle~$\theta$, $e^{i\theta}$ peut être interprété comme un point situé sur
le cercle unité dans le plan complexe. Puisque le plan complexe associe l'axe
horizontal avec la partie réelle du nombre et l'axe vertical avec la partie
imaginaire, la formule d'Euler indique que $e^{i\theta}$ correspond aux
coordonnées $(\cos\theta,\sin\theta)$. Remplacer celles-ci dans
l'équation~\eqref{eq:eulers_formula} par les valeurs des angles «\ns spéciaux\ns »
de la figure~\ref{fig:tau_angles} nous donne ensuite les points indiqués dans le
tableau~\ref{table:complex_exponentials}, et le traçage de ces points dans le
plan complexe nous donne la figure~\ref{fig:tau_euler_circle}. Une comparaison de la
figure~\ref{fig:tau_euler_circle} avec la figure~\ref{fig:tau_angles} dissipe
rapidement tout doute quant à laquelle des constantes du cercle révèle le
mieux la relation entre la formule d'Euler et la géométrie du cercle.

\begin{table}
\begin{center}
\begin{tabular}{lcc}
Forme polaire & Forme cartésienne & Coordonnées \\ \hline\hline
$e^{i\theta}$ & $\cos\theta + i\sin\theta$ & $(\cos\theta, \sin\theta)$ \\ \hline
$e^{i\cdot0}$ & $1$ & $(1, 0)$ \smallskip \\
$e^{i\tau/12}$ & $\frac{\sqrt{3}}{2} + \frac{1}{2}i$ & $(\frac{\sqrt{3}}{2}, \frac{1}{2})$ \smallskip \\
$e^{i\tau/8}$ & $\frac{1}{\sqrt{2}} +  \frac{1}{\sqrt{2}}i$ & $(\frac{1}{\sqrt{2}}, \frac{1}{\sqrt{2}})$ \smallskip \\
$e^{i\tau/6}$ & $\frac{1}{2} +\frac{\sqrt{3}}{2} i$ & $(\frac{1}{2}, \frac{\sqrt{3}}{2})$ \smallskip \\
$e^{i\tau/4}$ & $i$ & $(0, 1)$ \smallskip \\
$e^{i\tau/3}$ & $-\frac{1}{2} +\frac{\sqrt{3}}{2} i$ & $(-\frac{1}{2}, \frac{\sqrt{3}}{2})$ \smallskip \\
$e^{i\tau/2}$ & $-1$ & $(-1, 0)$ \smallskip \\
$e^{i\cdot(3\tau/4)}$ & $-i$ & $(0, -1)$ \smallskip \\
$e^{i\tau}$ & $1$ & $(1, 0)$
\end{tabular}
\end{center}
\caption{Exponentielles complexes des angles spéciaux de la
figure~\ref{fig:tau_angles}.\label{table:complex_exponentials}}
\end{table}

\begin{figure}
\begin{center}
\image{images/figures/tau_euler_circle.pdf}
\end{center}
\caption{Exponentielles complexes de certains angles spéciaux, tracées dans le
plan complexe.\label{fig:tau_euler_circle}}
\end{figure}

      % subsubsection eulerian_identities (end)

\section{La superficie circulaire~: le coup de grâce} % (fold)
\label{sec:circular_area}

Si vous êtes arrivé ici en tant que fidèle de $\pi$, vous devez commencer à
remettre en question votre foi. $\tau$ est si naturel, sa signification
si transparente\textellipsis\ n'y a-t-il pas d'exemple où $\pi$ brille d'une
splendeur rayonnante\ns? Une mémoire remue. Oui, il existe une telle
formule~: c'est la formule de la superficie du cercle\ns! Observez~:
\[ A = \tfrac{1}{4} \pi D^2. \]
Non, attendez. La formule de la superficie est toujours écrite en termes du
\emph{rayon}, comme suit~:
\[ A = \pi r^2. \]
On voit ici $\pi$, sans fioritures, dans l'une des équations les plus
importantes en mathématiques\ns; une formule prouvée pour la première fois par
\href{https://fr.wikipedia.org/wiki/Archimède}{Archimède} lui-même. L'ordre est
rétabli\ns! Et pourtant, le nom de cette section semble inquiétant\textellipsis\
Si cette équation est le couronnement de $\pi$, comment peut-elle aussi être le
coup de grâce\ns?


  \subsection{Formes quadratiques} % (fold)
  \label{sec:quadratic_forms}

Examinons cet exemple parfait de $\pi$, $A = \pi r^2$. On remarque qu'il implique
le diamètre\textellipsis\ non, le \emph{rayon}! élevé à la deuxième puissance.
Cela en fait une \emph{forme quadratique} simple. De telles formes surviennent
dans de nombreux contextes\ns; je suis
\href{https://thesis.library.caltech.edu/1940/}{physicien}, donc mes exemples
préférés viennent du cursus de physique élémentaire. Nous allons maintenant en
considérer plusieurs successivement.

    \subsubsection{Tomber dans un champ gravitationnel uniforme} % (fold)
    \label{sec:falling_in_a_uniform_gravitational_field}

\href{https://fr.wikipedia.org/wiki/Galil%C3%A9e_(savant)}{Galilée} a constaté
que le vecteur vitesse d'un objet qui tombe dans un champ gravitationnel
uniforme est proportionnel au temps tomber~:
\[ v \propto t. \]
La constante de proportionnalité est l'accélération de la pesanteur~$g$~:
\[ v = g t. \]
Puisque le vecteur vitesse est la dérivée de la position, on peut calculer la
distance tombée par
intégration\ns\footnote{\href{https://bit.ly/32mB2CF}{Techniquement}, toutes les
intégrales doivent être définies et la variable d'intégration doit être
différente de la limite supérieure (par exemple, $\int_{0}^{t} gt'\, dt'$, que
l'on prononce «\ns l'intégrale de zéro à té de gé té prime dé té prime\ns »). Ces
\href{https://fr.wikipedia.org/wiki/Abus_de_notation}{abus mineurs de notation}
sont courants en physique et dans d'autres contextes mathématiques moins formels
comme celui que l'on considère ici.}~:
\[ y = \int v\,dt = \int_0^t gt\,dt = \textstyle{\frac{1}{2}} gt^2. \]


    \subsubsection{Énergie potentielle dans un ressort linéaire} % (fold)
    \label{sec:potential_energy_in_a_linear_spring}

\href{https://fr.wikipedia.org/wiki/Robert_Hooke}{Robert Hooke} a constaté que
la force externe requise pour étirer un ressort est proportionnelle à la
distance étirée~:
\[ F \propto x. \]
La constante de proportionnalité est la constante de rappel~$k$\ns\footnote{On
peut avoir vu ceci écrit comme $F=-kx$. Dans ce cas, $F$ fait référence à la
force qu'exerce le \emph{ressort}. Selon la troisième loi de Newton, la force
externe discutée ci-dessus est le \emph{négatif} de la force du ressort.}~:
\[ F = k x. \]
L'énergie potentielle du ressort est alors égale au travail fourni par la force
externe~:
\[ U = \int F\,dx = \int_0^x kx\,dx = \textstyle{\frac{1}{2}} kx^2. \]

    \subsubsection{Énergie de mouvement} % (fold)
    \label{sec:energy_of_motion}

\href{https://fr.wikipedia.org/wiki/Isaac_Newton}{Isaac Newton} a constaté que
la force sur un objet est proportionnelle à son accélération~:
\[ F \propto a. \]
La constante de proportionnalité est la masse $m$~:
\[ F = m a. \]
L'énergie de mouvement, ou l'\emph{énergie cinétique}, est égale au travail
total fourni en accélérant la masse au vecteur vitesse~$v$~:
\[
\begin{split}
K = \int F\,dx = \int ma\,dx & = \int m\frac{dv}{dt}\,dx \\ & = \int m\frac{dx}{dt}\,dv \\ & = \int_0^v mv\,dv \\ & = \textstyle{\frac{1}{2}} mv^2.
\end{split}
\]

  \subsection{Un sentiment nerveux} % (fold)
  \label{sec:a_sense_of_foreboding}

Après avoir vu plusieurs exemples de formes quadratiques simples en physique, on
a maintenant peut-être un sentiment nerveux en retournant à la géométrie du
cercle. Ce sentiment est justifié.

\begin{figure}
\begin{center}
\image{images/figures/circular-area.pdf}
\end{center}
\caption{Se décomposer un cercle en anneaux.\label{fig:circular_area}}
\end{figure}


Comme le montre la figure~\ref{fig:circular_area}, on peut calculer la
superficie d'un cercle en le décomposant en anneaux circulaires de longueur $C$
et de largeur $dr$, où la superficie de chaque anneau est $C\,dr$~:
\[ dA = C\,dr. \]
La circonférence d'un cercle est proportionnelle à son rayon~:
\[ C \propto r. \]
La constante de proportionnalité est $\tau$~:
\[ C = \tau\,r. \]
La superficie du cercle est alors l'intégrale sur tous les anneaux~:
\[ A = \int dA = \int_0^r C\,dr = \int_0^r \tau\,r\,dr = \textstyle{\frac{1}{2}} \tau\,r^2. \]

Si l'on était encore partisan de $\pi$ au début de cette section, sa tête a
maintenant explosé. Car on voit que même dans ce cas, où $\pi$ brille
soi-disant, en fait il manque un facteur 2. En effet, la démonstration originale
d'Archimède montre non pas que la superficie d'un cercle est $\pi r^2$, mais
qu'elle est égale à la superficie d'un triangle rectangle de base $C$ et de
hauteur $r$. L'application de la formule pour la superficie triangulaire nous donne
alors
\[
  A = \textstyle{\frac{1}{2}} bh = \textstyle{\frac{1}{2}}Cr = \textstyle{\frac{1}{2}}\tau\,r^2.
\]
Il est tout simplement impossible d'éviter ce facteur de demi
(tableau~\ref{table:quadratic_forms}).

\begin{table}
\begin{center}
\begin{tabular}{lcc}
Quantité & Symbole & Expression \\ \hline
Distance tombée & $y$ & $\textstyle{\frac{1}{2}}gt^2$ \smallskip \\
Énergie de ressort & $U$ & $\textstyle{\frac{1}{2}}kx^2$ \smallskip \\
Énergie cinétique & $K$ & $\textstyle{\frac{1}{2}}mv^2$ \smallskip \\
Superficie circulaire & $A$ & $\textstyle{\frac{1}{2}}\tau\,r^2$
\end{tabular}
\end{center}
\caption{Quelques formes quadratiques courantes.\label{table:quadratic_forms}}
\end{table}

    \subsubsection{Quod erat demonstrandum} % (fold)
    \label{sec:quod_erat_demonstrandum}

On a voulu dans ce manifeste montrer que $\tau$ est la vraie constante du
cercle. Étant donné que la formule de la superficie circulaire était à peu près
le dernier, le meilleur argument que $\pi$ avait pour lui, je vais ici être
effronté et dire~:
\href{https://fr.wikipedia.org/wiki/CQFD_(math%C3%A9matiques)}{QED}.

    % subsubsection quod_erat_demonstrandum (end)

% section circular_area (end)

\section{Conflit et résistance} % (fold)
\label{sec:conflict_and_resistance}

Malgré la démonstration définitive de la supériorité de $\tau$, nombreux sont
néanmoins ceux qui s'y opposent, à la fois en notation et en nombre. Dans cette
section, on répond aux préoccupations de ceux qui acceptent la valeur mais pas
la lettre. On réfute ensuite certains des nombreux arguments opposés à $C/r$
lui-même, y compris le soi-disant «\ns Pi Manifesto\ns » («\ns Manifeste de pi\ns ») qui
défend la primauté de $\pi$. Dans ce contexte, on discutera du sujet assez
avancé du volume d'une hypersphère (la
section~\ref{sec:volume_of_a_hypersphere}), qui augmente et amplifie les
arguments de la section~\ref{sec:circular_area} sur la superficie circulaire.

  \subsection{Un tour} % (fold)
  \label{sec:one_turn}

La véritable épreuve de toute notation et l'utilisation\ns; ayant vu $\tau$
utilisé tout au long de ce manifeste, on est peut-être déjà convaincu qu'il
remplit bien son rôle. Mais pour une constante aussi fondamentale que $\tau$, ce
serait bien d'avoir des raisons plus profondes pour le choix. Pourquoi pas
$\alpha$, par exemple, ou $\omega$\ns? Qu'est-ce qui est si génial avec $\tau$\ns?

Il y a deux raisons principales d'utiliser $\tau$ pour la constante du cercle.
La première est que $\tau$ ressemble visuellement à $\pi$~: après des siècles
d'utilisation, l'association de $\pi$ à la constante du cercle est inévitable,
et l'utilisation de $\tau$ se nourrit de cette association au lieu de la
combattre. (En effet, la ligne horizontale dans chaque lettre suggère que l'on
interprète les «\ns jambes\ns » comme des dénominateurs, de sorte que $\pi$ a deux
jambes dans son dénominateur, tandis que $\tau$ n'en a qu'une. Vu de cette
façon, la relation $\tau = 2\pi$ est parfaitement naturelle\ns\footnote{Merci au
lectur du \emph{Manifeste de tau}, Jim Porter, d'avoir souligné cette
interprétation.}.)

La seconde raison est que $\tau$ correspond à un \emph{tour} d'un cercle, et
l'on a peut-être remarqué que «\ns $\tau$\ns » et «\ns tour\ns » (en anglais,
«\ns turn\ns ») commencent tous les deux par un son «\ns t\ns ». Telle était la
motivation d'origine du choix de $\tau$, et ce n'est pas un hasard~: la racine
du mot anglais «\ns turn\ns » est le mot grec τόρνος (tornos), qui signifie
«\ns tour\ns » (la machine-outil\ns; en anglais, «\ns lathe\ns »). L'utilisation d'une
fonte mathématique pour la première lettre de τόρνος nous donne alors~: $\tau$.

Depuis le lancement original du \emph{Manifeste de tau}, j'ai appris que
\href{http://www.harremoes.dk/Peter/Undervis/Turnpage/Turnpage1.pdf}{Peter
Harremoës} a proposé de manière indépendante d'utiliser $\tau$ à l'auteur de
«\ns $\pi$ Is Wrong!\ns », Bob Palais, en 2010\ns; John Fisher a proposé $\tau$ dans
un
\href{https://groups.google.com/forum/#!msg/sci.math/c-DHmJHSA0A/sLCoOtHB1UAJ}{poste
Usenet} en 2004\ns; et Joseph Lindenberg a anticipé à la fois l'argument et le
symbole plus de vingt ans auparavant\ns\footnote{Lindenberg a inclus à la fois
son manuscrit dactylographié original et un grand nombre d'autres arguments sur
son site \href{https://sites.google.com/site/taubeforeitwascool/}{Tau Before It
Was Cool} («\ns Tau avant qu'il ne soit cool\ns »).}\ns! Le Dr Harremoës en
particulier a souligné l'importance d'un point qui a été soulevé pour la
première fois dans la section~\ref{sec:an_immodest_proposal}~: l'utilisation de
$\tau$ donne un \emph{nom} à la constante du cercle. Puisque $\tau$ est une
lettre grecque ordinaire, les personnes qui la rencontrent pour la première fois
peuvent la prononcer immédiatement. De plus, contrairement à appeler la
constante du cercle un «\ns tour\ns », $\tau$ fonctionne bien à la fois dans des
contextes écrits et parlés. Par exemple, dire qu'un quart d'un cercle a une
mesure d'angle en radians «\ns un quart de tour\ns » sonne bien, mais «\ns tour sur quatre
radians\ns » semble maladroit et «\ns la superficie d'un cercle est de demi-tour $r$
au carré\ns » sonne carrément bizarre. En utilisant $\tau$, on peut dire «\ns tau
sur quatre radians\ns » et «\ns la superficie d'un cercle est de demi tau $r$ au
carré\ns ».

    \subsubsection{Notation ambiguë} % (fold)
    \label{sec:ambiguous_notation}


Bien sûr, avec toute nouvelle notation, il y a un risque de conflit avec
l'utilisation actuelle. Comme le noté dans la
section~\ref{sec:an_immodest_proposal}, «\ns $\pi$ is Wrong!\ns » évite ce problème
en introduisant un nouveau symbole (figure~\ref{fig:palais_tau}). Il y a des
précédents à cela\ns; par exemple, aux premiers jours de la mécanique quantique,
\href{https://fr.wikipedia.org/wiki/Max_Planck}{Max Planck} a introduit la
constante~$h$, qui relie l'énergie d'une particule lumineuse à sa fréquence (via
$E = h\nu$), mais les physiciens se sont vite rendus compte qu'il était souvent
plus pratique d'utiliser $\hbar$, que l'on prononce «\ns h barre\ns », (où $\hbar$
est juste $h$ divisé par\textellipsis{} euh\textellipsis{} $2\pi$) et cet usage
est depuis devenu standard.

Mais il est difficile pour un nouveau symbole d'être accepté~: il faut lui
donner un nom, ce nom doit être popularisé, et le symbole lui-même doit être
ajouté aux systèmes de traitement de texte et de composition. De plus, la
promulgation d'un nouveau symbole pour $2\pi$ nécessiterait la coopération de la
communauté mathématique académique, qui sur le sujet de $\pi$ vs $\tau$ a été
jusqu'à présent apathique au mieux et hostile au pire. L'utilisation d'un
symbole existant permet de contourner l'établissement
mathématique\ns\footnote{Peut-être qu'un jour, les mathématiciens académiques
arriveront à un consensus sur un symbole différent pour le nombre $2\pi$\ns; si
cela se produit, je me réserve le droit de soutenir leur notation proposée. Mais
ils ont eu plus de 300 ans pour résoudre ce problème de $\pi$, donc je ne suis
pas optimiste qu'ils le fassent bientôt.}.

Plutôt que de préconiser un nouveau symbole, \emph{Le Manifeste de tau} opte
pour l'utilisation d'une lettre grecque existante. Par conséquent, puisque
$\tau$ est déjà utilisé dans certains contextes actuels, il faut résoudre les
conflits avec la pratique existante. Heureusement, il existe étonnamment peu
d'utilisations courantes. De plus, alors que $\tau$ est utilisé pour certains
variables \emph{spécifiques} (par exemple, la \emph{contrainte de cisaillement}
en génie mécanique, le \emph{moment d'une force} en mécanique rotationnelle et
\emph{temps propre} en relativité restreinte et générale) il n'y a pas
d'utilisation conflictuelle \emph{universelle}\ns\footnote{La seule exception
possible à cela est le \emph{nombre d'or}, qui est souvent désigné par $\tau$ en
Europe. Mais non seulement existe-t-il une alternative commune à cette notation,
à savoir la lettre grecque $\varphi$, mais cette utilisation montre qu'il existe
un précédent pour utiliser $\tau$ pour désigner une constante mathématique
fondamentale.}. Dans ce cas, on peut soit tolérer l'ambiguïté, soit contourner
les quelques conflits actuels en modifiant sélectivement la notation, comme en
utilisant $N$ pour le moment d'une force\ns\footnote{Cette alternative pour le
moment d'une force est déjà utilisée\ns; voir, par exemple, \emph{Introduction to
Electrodynamics} (\emph{Introduction à l'électrodynamique}) de David Griffiths,
page 162.} ou $\tau_p$ pour temps propre.

Malgré ces arguments, les conflits potentiels d'utilisation se sont révélés être
la plus grande source de résistance à $\tau$. Certains correspondants ont même
catégoriquement nié que $\tau$ (ou, vraisemblablement, tout autre symbole
actuellement utilisé) pourrait éventuellement surmonter ces problèmes. Mais les
scientifiques et les ingénieurs ont une grande tolérance à l'ambiguïté de
notation, et affirmer que $\tau$-la-constante-du-cercle ne peut pas coexister
avec d'autres utilisations ignore des preuves considérables du contraire.

Un exemple d'ambiguïté facilement toléré se produit en mécanique quantique, où
l'on rencontre la formule suivante pour le \emph{rayon de Bohr}, qui (grosso
modo) est la «\ns taille\ns » d'un atome d'hydrogène dans son niveau d'énergie le
plus bas (l'\emph{état fondamental})~:
\[
a_0 = \frac{\hbar^2}{m e^2},
\]
où $m$ est la masse d'un électron et $e$ est sa charge. Pendant ce temps, l'état
fondamental lui-même est décrit par une quantité connue sous le nom de
\href{https://fr.wikipedia.org/wiki/Fonction_d%27onde}{\emph{fonction d'onde}},
qui diminue de façon exponentielle avec un rayon sur une échelle de longueur
définie par le rayon de Bohr~:
\begin{equation}
\label{eq:hydrogen}
\psi(r) = N\,e^{-r/a_0},
\end{equation}
où $N$ est une constante de normalisation.

A-t-on encore remarqué le problème\ns? Probablement pas, ce qui est juste le
point. Le «\ns problème\ns » est que le $e$ dans le rayon de Bohr et le $e$ dans la
fonction d'onde \emph{ne sont pas les mêmes $e$}~: le premier est la charge sur
un électron, tandis que le second est le nombre naturel (la base des logarithmes
naturels). En fait, si l'on développe le facteur de $a_0$ dans l'argument de
l'exposant dans l'équation~\eqref{eq:hydrogen}, on obtient
\[
\psi(r) = N\,e^{-m e^2 r/\hbar^2},
\]
qui a un $e$ augmenté à la puissance de quelque chose qui contient $e$. C'est
encore pire qu'il n'y paraît, car $N$ lui-même contient aussi $e$~:
\[
\psi(r) = \sqrt{\frac{1}{\pi a_0^3}}\,e^{-r/a_0} =
\frac{m^{3/2} e^3}{\pi^{1/2} \hbar^3}\,e^{-m e^2 r/\hbar^2}.
\]

Je ne doute pas que s'il n'existait pas déjà une notation distincte pour le
nombre naturel, quiconque proposerait la lettre $e$ se verrait dire que cela
était impossible en raison des conflits avec d'autres utilisations. Et pourtant,
dans la pratique, personne n'a jamais de problème avec l'utilisation de $e$ dans
les deux contextes ci-dessus. Il existe de nombreux autres exemples, y compris
des situations où même $\pi$ est utilisé pour deux choses
differentes\ns\footnote{Voir, par exemple, \emph{An Introduction to Quantum Field
Theory} (\emph{Une Introduction à la théorie quantique des champs}) par Peskin
et Schroeder, où $\pi$ est utilisé pour désigner à la fois la constante du
cercle et une «\ns quantité conjuguée de mouvement\ns » sur la même page
(page~282).}. Il est difficile de voir en quoi l'utilisation de $\tau$ pour
multiples quantités est différente.

Soit dit en passant, les pédants de $\pi$ (et il y en a eu beaucoup) pourraient
remarquer que la fonction d'onde de l'état fondamental de l'hydrogène a un
facteur $\pi$~:
\[
\psi(r) = \sqrt{\frac{1}{\pi a_0^3}}\,e^{-r/a_0}.
\]
À première vue, cela semble plus naturel que la version avec $\tau$~:
\[
\psi(r) = \sqrt{\frac{2}{\tau a_0^3}}\,e^{-r/a_0}.
\]
Comme d'habitude, les apparences sont trompeuses~: la valeur de $N$ vient du
produit
\[
\frac{1}{\sqrt{2\pi}} \frac{1}{\sqrt{2}} \frac{2}{a_0^{3/2}},
\]
ce qui montre que la constante du cercle entre dans le calcul par
$1/\sqrt{2\pi}$, c'est-à-dire $1/\sqrt{\tau}$. Comme pour la formule de la
superficie circulaire, la neutralisation qui laisse un $\pi$ dépouillé est un
coïncidence.

    % subsubsection ambiguous_notation (end)

  \subsection{Le Manifeste de pi} % (fold)
  \label{sec:the_pi_manifesto_a_rebuttal}

Bien que la plupart des objections à $\tau$ proviennent de la correspondance
électronique dispersée et de divers commentaires sur le Web, il existe aussi une
résistance organisée. En particulier, depuis la publication du \emph{Manifeste
de tau} en juin 2010, un «\ns \href{http://thepimanifesto.com/}{Manifeste de
pi}\ns » est apparu pour plaider en faveur de la constante traditionnelle du
cercle. Cette section et les deux suivantes contiennent une réfutation de ses
arguments. Par nécessité, ce traitement est plus laconique et plus avancé que le
reste du manifeste, mais même une lecture superficielle de ce qui suit donnera
une impression de la faiblesse du cas du Manifeste de pi.

Alors que l'on peut certainement considérer l'apparition du Manifeste de pi
comme un bon signe d'intérêt continu pour ce sujet, il fait plusieurs fausses
affirmations. Par exemple, il dit que le facteur de $2\pi$ dans la loi
gaussienne (normale) est une coïncidence, et qu'il peut plus naturellement
s'écrire comme
\[
\frac{1}{\sqrt\pi(\sqrt 2\sigma)}e^{\frac{-x^2}{(\sqrt 2\sigma)^2}}.
\]
C'est faux~: le facteur de $2\pi$ vient de l'élévation au carré de la loi
gaussienne non normalisée et du passage aux coordonnées polaires, ce que conduit
à un facteur 1 de l'intégrale radiale et $2\pi$ de l'intégrale angulaire. Comme
dans le cas de la superficie circulaire, le facteur de $\pi$ vient de $1/2\times
2\pi$, pas de $\pi$ seul.

Une affirmation connexe est que la
\href{https://fr.wikipedia.org/wiki/Fonction_gamma}{fonction gamma} évaluée à
$1/2$ est plus naturelle en termes de $\pi$~:
\[
\Gamma(\textstyle{\frac{1}{2}}) = \sqrt{\pi},
\]
où
\begin{equation}
\label{eq:gamma}
\Gamma(p) = \int_{0}^{\infty} x^{p-1} e^{-x}\,dx.
\end{equation}
Mais $\Gamma(\frac{1}{2})$ se réduit à la même intégrale gaussienne que dans la
loi normale (lors du réglage de $u = x^{1/2}$), donc $\pi$ dans ce cas est aussi
vraiment $1/2\times 2\pi$. En effet, dans de nombreux cas cités dans Le
Manifeste de pi, la constante du cercle entre par une intégrale sous tous les
angles, c'est-à-dire comme $\theta$ varie de $0$ à $\tau$.

Le Manifeste de pi examine aussi certaines formules pour des polygones réguliers
à $n$ côtés (ou «\ns $n$-gones\ns »). Par exemple, il note que la somme des angles
internes d'un $n$-gone nous est donnée par
\[
\sum_{i=1}^n \theta_i=(n-2)\pi.
\]
Ce problème a été traité dans «\ns $\pi$ Is Wrong!\ns », qui note ce qui suit~:
«\ns La somme des angles intérieurs [d'un triangle] est $\pi$, c'est vrai. Mais la
somme des angles \emph{extérieurs} de \emph{tout} polygone, dont la somme des
angles intérieurs peut facilement être dérivée, et qui se généralise à
l'intégrale de la courbure d'une simple courbe fermée, est de $2\pi$.\ns » De
plus, Le Manifeste de pi offre la formule de la superficie d'un $n$-gone avec un
rayon unité (la distance du centre au sommet),
\[ A=n\sin\frac{\pi}{n}\cos\frac{\pi}{n}, \]
et l'appelle «\ns clairement\textellipsis{} une autre victoire pour $\pi$\ns ». Mais
l'utilisation de l'identité à double angle $\sin\theta\cos\theta =
\frac{1}{2}\sin2\theta$ montre que l'on peut écrire ceci comme
\[ A = n/2\, \sin\frac{2\pi}{n}, \]
ce qui est juste
\begin{equation}
\label{eq:area_polygon}
A = \frac{1}{2} n\, \sin\frac{\tau}{n}.
\end{equation}
En d'autres termes, la superficie d'un $n$-gone a un facteur
naturel de $1/2$. En fait, prendre la limite de
l'équation~\eqref{eq:area_polygon} comme $n\rightarrow \infty$ (et appliquer la
\href{https://fr.wikipedia.org/wiki/R%C3%A8gle_de_L%27H%C3%B4pital}{règle de
L'Hôpital}) nous donne la superficie d'un polygone régulier unité avec une infinité
de côtés, c'est-à-dire un cercle unité~:
\begin{equation}
\label{eq:lhopital}
\begin{split}
A & = \lim_{n\rightarrow\infty} \frac{1}{2} n\, \sin\frac{\tau}{n} \\
  & = \frac{1}{2} \lim_{n\rightarrow\infty} \frac{\sin\frac{\tau}{n}}{1/n} \\
  & = \tfrac{1}{2}\tau.
\end{split}
\end{equation}

Dans ce contexte, il convient de noter que Le Manifeste de pi fait beaucoup de
bruit sur le fait que $\pi$ est la superficie d'un disque unité, de sorte que
(par exemple) la superficie d'un quart de cercle (unité) est $\pi/4$. Ceci,
prétend-on, est tout aussi bon pour $\pi$ que la mesure d'angle en radians pour
$\tau$. Malheureusement pour cet argument, comme noté dans la
section~\ref{sec:circular_area} et comme on le voit à nouveau dans
l'équation~\eqref{eq:lhopital}, le facteur $1/2$ apparaît naturellement dans le
contexte de la superficie circulaire. En effet, la formule de la superficie d'un
secteur circulaire sous-tendu par l'angle $\theta$ est
\[
\tfrac{1}{2}\theta\, r^2,
\]
donc il n'y a aucun moyen d'éviter le facteur $1/2$ en général. (On voit donc
que $A = \frac{1}{2}\tau r^2$ est simplement le cas particulier $\theta =
\tau$.)

En bref, la différence entre la mesure d'angle et la superficie n'est pas
arbitraire. \linebreak Il n'y a pas de facteur naturel de $1/2$ dans le cas de
la mesure d'angle. En revanche, dans le cas de la superficie, la facteur $1/2$
résulte de l'intégrale d'une fonction linéaire en association avec une forme
quadratique simple. En fait, le cas de $\pi$ est encore pire qu'il n'y paraît,
comme le montre la section suivante.

  % subsection the_pi_manifesto_a_rebuttal (end)

\section{Aller au fond de pi et tau} % (fold)
\label{sec:getting_to_the_bottom_of_pi}

Je continue d'être impressionné par la richesse de ce sujet, et ma compréhension
de $\pi$ et $\tau$ continue d'évoluer. Le jour de demi-tau 2012, je pensais
avoir identifié \emph{exactement} ce qui n'allait pas avec $\pi$. Mon argument
reposait sur une analyse de la superficie et du volume d'une sphère à $n$
dimensions, qui (comme le montre ci-dessous) montre clairement que $\pi$ n'a pas
de signification géométrique fondamentale. Mon analyse était cepandant
incomplète~: un fait porté à mon attention dans un message remarquable du
lecture du \emph{Manifeste de tau}, Jeff Cornell. En conséquence, cette section
est une tentative non seulement de discréditer définitivement $\pi$, mais aussi
d'articuler la vérité sur $\tau$, une vérité plus profonde et plus subtile que
je ne l'avais imaginé.

\emph{Note}~: Cette section est plus avancée que le reste du manifeste et peut
être ignorée sans perte de continuité. Si l'on la trouve déroutante, je
recommande de passer directement à la conclusion de la
section~\ref{sec:conclusion}.

  \subsection{Superficie et volume d'une hypersphère} % (fold)
  \label{sec:volume_of_a_hypersphere}

On commence ses investigations par la généralisation d'un cercle à des
dimensions arbitraires\ns\footnote{Cette discussion est basée sur un excellent
commentaire de John Kodegadulo sur \url{http://spikedmath.com}.}. Cet objet,
appelé \emph{hypersphère} ou \emph{$n$-sphère}, peut-être défini comme
suit\ns\footnote{Les géomètres et les topologues utilisent des définitions
incompatibles d'hypersphères\ns; cette discussion utilise les définitions des
géomètres.}. (Par commodité, on suppose que ces sphères sont centrées sur
l'origine.) Une $0$-sphère est l'ensemble vide, et l'on définit son
«\ns intérieur\ns » comme un point\ns\footnote{Cela a du sens, car un point n'a pas
de frontière, c'est-à-dire que la frontière d'un point est l'ensemble vide.}.
Une $1$-sphère est l'ensemble de tous les points qui satisfont
\[
x^2 = r^2,
\]
qui se compose des deux points $\pm r$. Son intérieur, qui satisfait
\[
x^2 \leq r^2,
\]
est le segment de droite de $-r$ à $r$. Une $2$-sphère est un cercle, qui est
l'ensemble de tous les points qui satisfont
\[
x^2 + y^2 = r^2.
\]
Son intérieur, qui satisfait
\[
x^2 + y^2 \leq r^2,
\]
est un disque. De même, une $3$-sphère satisfait
\[
x^2 + y^2 + z^2 = r^2,
\]
et son intérieur est une balle. La généralisation à $n$ arbitraire, bien que
difficile à visualiser pour $n > 3$, est simple~: une $n$-sphère est l'ensemble
de tous les points qui satisfont
\[
\sum_{i=1}^{n} x_i^2 = r^2.
\]

Le Manifeste de pi (discuté dans la
section~\ref{sec:the_pi_manifesto_a_rebuttal}) inclut une formule pour le volume
d'une $n$-sphère unité comme argument en faveur de $\pi$~:
\begin{equation}
\label{eq:unit_n_sphere_pi}
\frac{\sqrt{\pi}^{n} }{\Gamma(1 + \frac{n}{2})},
\end{equation}
où la fonction gamma nous est donnée par l'équation~\eqref{eq:gamma}.
L'équation~\eqref{eq:unit_n_sphere_pi} est un cas particulier de la formule du
rayon général, qui est aussi généralement écrite en termes de $\pi$~:
\begin{equation}
\label{eq:n_sphere_pi}
V_n(r) = \frac{\pi^{n/2} r^n}{\Gamma(1 + \frac{n}{2})}.
\end{equation}
Parce que $V_n(r) = \int S_n(r)\,dr$, on a $S_n(r) = dV_n(r)/dr$, ce qui
signifie que la superficie peut s'écrire comme suit~:
\begin{equation}
\label{eq:n_sphere_pi_r}
S_n(r) = \frac{n \pi^{n/2} r^{n-1}}{\Gamma(1 + \frac{n}{2})}.
\end{equation}

Plutôt que de simplement prendre ces formules à leur valeur nominale,
voyons si l'on peut les démêler pour mieux éclairer la question de $\pi$ vs
$\tau$. On commence son analyse en notant que l'apparente simplicité des
formules ci-dessus est une illusion~: bien que la fonction gamma soit simple sur
le plan de la notation, en fait elle est une intégrale sur un domaine
semi-infini, ce qui n'est pas du tout une idée simple. Heureusement, on peut
simplifier la fonction gamma dans certains cas particuliers. Par exemple,
lorsque $n$ est un entier, il est facile de montrer (en utilisant l'intégration
par parties) que
\[
\Gamma(n) = (n-1)(n-2)\ldots 2\cdot 1 = (n-1)!
\]
Vu de cette façon, on peut interpréter $\Gamma$ comme une généralisation de la
fonction factorielle à des arguments à valeurs réelles\ns\footnote{En effet, la
généralisation aux arguments à valeurs complexes est simple~: il suffit de
remplacer le $x$ réel par le $z$ complexe dans l'équation~\eqref{eq:gamma}.}.

Dans les formules de superficie et de volume à $n$ dimensions, l'argument de
$\Gamma$ n'est pas nécessairement un entier, mais plutôt $\left(1 +
\frac{n}{2}\right)$, qui est un entier lorsque $n$ est pair est un demi-entier
lorsque $n$ est impair. En tenir compte nous donne l'expression suivante, qui est
tirée d'une référence standard,
\href{https://mathworld.wolfram.com/Hypersphere.html}{Wolfram MathWorld}, et
comme d'habitude est écrite en termes de~$\pi$~:
\begin{equation}
\label{eq:surface_area_mathworld}
S_n(r) = \begin{cases}
\displaystyle \frac{2\pi^{n/2}\,r^{n-1}}{(\frac{1}{2}n - 1)!} & n \text{ pair}; \\ \\
 \displaystyle \frac{2^{(n+1)/2}\pi^{(n-1)/2}\,r^{n-1}}{(n-2)!!} & n \text{ impair}.
\end{cases}
\end{equation}

L'intégration par rapport à $r$ nous donne alors
\begin{equation}
\label{eq:volume_mathworld}
V_n(r) = \begin{cases}
\displaystyle \frac{\pi^{n/2}\,r^n}{(\frac{n}{2})!} & n \text{ pair}; \\ \\
\displaystyle \frac{2^{(n+1)/2}\pi^{(n-1)/2}\,r^n}{n!!} & n \text{ impair}.
\end{cases}
\end{equation}

Examinons l'équation~\eqref{eq:volume_mathworld} plus en détail. On
remarque tout d'abord que MathWorld utilise la
\emph{double factorielle}~$n!!$\ns; mais, étrangement, il ne l'utilise que dans le
cas \emph{impair}. (Ceci est un soupçon de choses à venir.) La double
factorielle, bien que rarement rencontrée en mathématiques, est élémentaire~:
elle est comme la fonction normale factorielle, mais implique de soustraire $2$
à la fois au lieu de $1$, de sorte que, par exemple, $5!! = 5 \cdot 3 \cdot 1$
et $6!! = 6 \cdot 4 \cdot 2$. En général, on a
\begin{equation}
\label{eq:double_factorial}
n!! = \begin{cases}
n(n-2)\ldots6\cdot4\cdot2 & n \text{ pair}; \\ \\
n(n-2)\ldots5\cdot3\cdot1 & n \text{ impair}.
\end{cases}
\end{equation}
(Par définition, $0!! =1!! = 1$.) On note que
l'équation~\eqref{eq:double_factorial} se divise naturellement en cas pairs et
impairs, ce qui rend encore plus mystérieuse la décision de MathWorld de
l'utiliser uniquement dans le cas impair.

Pour résoudre ce mystère, on va commencer en regardant de plus près la formule
pour $n$~impair dans l'équation~\eqref{eq:volume_mathworld}~:
\[ \frac{2^{(n+1)/2}\pi^{(n-1)/2}\,r^n}{n!!} \]
En examinant l'expression
\[ 2^{(n+1)/2}\pi^{(n-1)/2}, \]
on remarque que l'on peut la réécrire comme
\[ 2(2\pi)^{(n-1)/2}, \]
et ici on reconnaît son vieil ami~$2\pi$.

Examinons maintenant le cas pair dans
l'équation~\eqref{eq:volume_mathworld}. On a noté ci-dessus combien il est
étrange d'utiliser la factorielle ordinaire dans le cas pair mais la double
factorielle dans le cas impair. En effet, car la double factorielle est déjà
définie par morceaux, si l'on unifie les formules en utilisant $n!!$ dans les
deux cas, on pourrait le retirer comme un facteur commun~:
\[
V_n(r) = \frac{1}{n!!}\times \begin{cases}
\ldots & n \text{ pair}; \\ \\
 \ldots & n \text{ impair}.
 \end{cases}
\]
Alors, y a-t-il un lien entre la factorielle et la double factorielle\ns?
Oui~: lorsque $n$ est pair, les deux sont liés par l'identité suivante~:
\[ \left(\frac{n}{2}\right)! = \frac{n!!}{2^{n/2}}. \]
(Ceci est facile à vérifier en utilisant le
\href{https://fr.wikipedia.org/wiki/Raisonnement_par_récurrence}{raisonnement
par récurrence}.) La substitution de ceci dans la formule de volume pour $n$
pair nous donne alors
\[ \frac{2^{n/2}\pi^{n/2}\,r^n}{n!!}, \]
ce qui ressemble beaucoup à
\[ \frac{(2\pi)^{n/2}\,r^n}{n!!}, \]
et là encore on trouve un facteur $2\pi$.

En rassemblant ces résultats, on voit que l'on peut réécrire
l'équation~\eqref{eq:volume_mathworld} comme
\begin{equation}
\label{eq:volume_2pi}
V_n(r) = \begin{cases}
 \displaystyle \frac{(2\pi)^{n/2}\,r^n}{n!!} & n \text{ pair}; \\ \\
 \displaystyle \frac{2(2\pi)^{(n-1)/2}\,r^n}{n!!} & n \text{ impair}
 \end{cases}
\end{equation}
et l'équation~\eqref{eq:surface_area_mathworld} comme
\begin{equation}
\label{eq:surface_area_2pi}
S_n(r) = \begin{cases}
\displaystyle \frac{(2\pi)^{n/2}\,r^{n-1}}{(n-2)!!} & n \text{ pair}; \\ \\
\displaystyle \frac{2(2\pi)^{(n-1)/2}\,r^{n-1}}{(n-2)!!} & n \text{ impair}.
\end{cases}
\end{equation}

Faire la substitution $\tau=2\pi$ dans l'équation~\eqref{eq:surface_area_2pi}
nous donne alors
\[
S_n(r) = \begin{cases}
\displaystyle \frac{\tau^{n/2}\,r^{n-1}}{(n-2)!!} & n \text{ pair}; \\ \\
\displaystyle \frac{2\tau^{(n-1)/2}\,r^{n-1}}{(n-2)!!} & n \text{ impair}.
\end{cases} \]
Pour unifier davantage les formules, on peut utiliser la \emph{fonction partie
entière} $\lfloor x \rfloor$, qui est tout simplement le plus grand entier
inférieur ou égal à $x$ (équivalent à couper la partie fractionnaire, de sorte
que, par exemple, $\lfloor 3{,}7 \rfloor = \lfloor 3{,}2 \rfloor = 3$). Cela
nous donne
\[ S_n(r) = \begin{cases}
 \displaystyle \frac{\tau^{\left\lfloor \frac{n}{2} \right\rfloor}\,r^{n-1}}{(n-2)!!} & n \text{ pair}; \\ \\
 \displaystyle \frac{2\tau^{\left\lfloor \frac{n}{2} \right\rfloor}\,r^{n-1}}{(n-2)!!} & n \text{ impair},
 \end{cases} \]
ce qui permet d'écrire la formule comme suit~:
\begin{equation}
\label{eq:surface_area_tau}
S_n(r) = \frac{\tau^{\left\lfloor \frac{n}{2} \right\rfloor}\,r^{n-1}}{(n-2)!!}\times \begin{cases}
1 & n \text{ pair}; \\ \\
2 & n \text{ impair}.
\end{cases}
\end{equation}
L'intégration de l'équation~\eqref{eq:surface_area_tau} par rapport à $r$ nous donne
alors
\begin{equation}
\label{eq:volume_tau}
V_n(r) = \frac{\tau^{\left\lfloor \frac{n}{2} \right\rfloor}\,r^n}{n!!}\times \begin{cases}
1 & n \text{ pair}; \\ \\
2 & n \text{ impair}.
\end{cases}
\end{equation}

\subsubsection{Lambda} % (fold)
\label{sec:lambda}

Les formules de l'équation~\eqref{eq:surface_area_tau} et
l'équation~\eqref{eq:volume_tau} représentent une amélioration majeure par
rapport aux formulations originales
(équation~\eqref{eq:surface_area_mathworld} et
équation~\eqref{eq:volume_mathworld}) en termes de $\pi$. Mais en fait une
simplification supplémentaire est possible, en utilisant la mesure d'un angle
droit\ns\footnote{Ce changement de notation et d'analyse générale était suggéré
par Jeff Cornell.}~:
\begin{equation}
\label{eq:lambda}
\lambda = \frac{\tau}{4}.
\end{equation}
Comme on le verra dans la section~\ref{sec:three_families_of_constants}, on peut
plus naturellement réécrire l'équation~\eqref{eq:lambda} en termes de symétries
du cercle~:
\begin{equation}
\label{eq:tau_lambda}
\tau = 2^2 \lambda,
\end{equation}
où le facteur $2^2$ provient des $2^2$ arcs circulaires congruents (un dans
chaque quadrant) dans espace bidimensionnel.

Le plus grand avantage de $\lambda$ est qu'il unifie complètement les cas pair
et impair dans l'équation~\eqref{eq:surface_area_tau} et
l'équation~\eqref{eq:volume_tau}, dont chacun a un facteur $\tau^{\left\lfloor
\frac{n}{2} \right\rfloor}$. Faire la substitution dans
l'équation~\eqref{eq:tau_lambda} nous donne alors
\[
\begin{split}
\tau^{\left\lfloor \frac{n}{2} \right\rfloor} = (2^2\lambda)^{\left\lfloor \frac{n}{2} \right\rfloor} & = 2^{2\left\lfloor \frac{n}{2} \right\rfloor} \lambda^{\left\lfloor \frac{n}{2} \right\rfloor} \\
& = \lambda^{\left\lfloor \frac{n}{2} \right\rfloor}\times
\begin{cases}
 2^n & n \text{ pair}; \\ \\
 2^{n-1} & n \text{ impair}.
 \end{cases}
 \end{split}
\]
Cela signifie que l'on peut réécrire le produit
\[
\tau^{\left\lfloor \frac{n}{2} \right\rfloor}\times \begin{cases}
1 & n \text{ pair}; \\ \\
2 & n \text{ impair}.
\end{cases}
\]
comme
\begin{equation}
\label{eq:prefactor}
\begin{split}
\lambda^{\left\lfloor \frac{n}{2} \right\rfloor} \times
\begin{cases}
 2^n & n \text{ pair}; \\ \\
 2^{n-1} & n \text{ impair}
 \end{cases}
 & \times
\begin{cases}
 1 & n \text{ pair}; \\ \\
 2 & n \text{ impair}
 \end{cases}
\\ & = 2^n\,\lambda^{\left\lfloor \frac{n}{2} \right\rfloor},
\end{split}
\end{equation}
ce qui élimine la dépendance explicite de la
\href{https://fr.wikipedia.org/wiki/Parité_(arithmétique)}{parité}.
L'application de \linebreak l'équation~\eqref{eq:prefactor} à
l'équation~\eqref{eq:surface_area_tau} et à l'équation~\eqref{eq:volume_tau}
nous donne alors
\begin{equation}
\label{eq:surface_area_lambda}
S_n(r) = \frac{2^n\,\lambda^{\left\lfloor \frac{n}{2} \right\rfloor}\,r^{n-1}}{(n-2)!!}
\end{equation}
et
\begin{equation}
\label{eq:volume_lambda}
V_n(r) = \frac{2^n\,\lambda^{\left\lfloor \frac{n}{2} \right\rfloor}\,r^n}{n!!}.
\end{equation}

La simplification de l'équation~\eqref{eq:surface_area_lambda} et de
l'équation~\eqref{eq:volume_lambda} semble se faire au prix d'un facteur $2^n$,
mais même cela a une signification géométrique claire~: une sphère en $n$
dimensions se divise naturellement en $2^n$ morceaux congruents, correspondant
aux $2^n$ familles de solutions à $\sum_{i=1}^{n} x_i^2 = r^2$ (une pour chaque
choix de $\pm x_i$). En deux dimensions, ce sont les arcs circulaires dans
chacun des quatre quadrants\ns; en trois dimensions, ce sont les secteurs de la
sphère dans chaque octant\ns; et ainsi de suite dans des dimensions supérieures.
En d'autres termes, on peut exploiter la symétrie de la sphère en calculant la
superficie ou le volume d'\emph{un} morceau (généralement la \emph{partie
principale} où $x_i > 0$) puis trouver la valeur complète en multipliant
par~$2^n$.

À ma connaissance, l'équation~\eqref{eq:surface_area_lambda} et
l'équation~\eqref{eq:volume_lambda} sont les formulations les plus simples
possibles des formules de superficie et de volume sphérique (et sont en fait les
seules formes que j'ai toujours pu mémoriser). On considère la formule de volume
en particulier~: contrairement à la fausse simplicité de
l'équation~\eqref{eq:n_sphere_pi}, l'équation~\eqref{eq:volume_lambda}
n'implique pas d'intégrales compliquées\ns; juste les fonctions légèrement
exotiques mais néanmoins élémentaires partie entière et double factorielle. Le
volume d'une $n$-sphère unité est juste le volume de chaque morceau symétrique,
$\lambda^{\left\lfloor \frac{n}{2} \right\rfloor}/n!!$, multiplié par le nombre
de morceaux, $2^n$.

% subsubsection lambda (end)

\subsubsection{Récurrences} % (fold)
\label{sec:recurrences}

On a maintenant vu, via l'équation~\eqref{eq:surface_area_lambda} et
l'équation~\eqref{eq:volume_lambda}, que les formules de superficie et de volume
sont les plus simples en termes d'angle droit $\lambda$. Néanmoins, on n'a
toujours pas fini avec $\tau$.

Comme on le voit dans l'équation~\eqref{eq:volume_lambda}, la formule de volume
se divise naturellement en deux familles, qui correspondent respectivement aux
espaces à dimensions paires et impaires. Cela signifie que le volume à quatre
dimensions, $V_4$, est lié simplement à $V_2$ mais pas à $V_3$, tandis que $V_3$
est lié à $V_1$ mais pas à $V_2$. Comment sont-ils exactement liés\ns?

On peut trouver la réponse en dérivant les \emph{relations de récurrence} entre
les dimensions\ns\footnote{L'article
«\ns \href{http://www2.math.uconn.edu/~mariano/research/MathClubsp14\%20.pdf}{The
volume of the unit ball in \emph{n} dimensions}\ns » («\ns Le volume de la balle
unité en $n$ dimensions\ns ») de Phanuel A. Mariano contient une dérivation
alternative de ces récurrences importantes.}. En particulier, divisons le
volume d'une sphère à $n$ dimensions par le volume d'une sphère à $n-2$
dimensions~:
\begin{equation}
\label{eq:volume_recurrence}
\begin{split}
\frac{V_n(r)}{V_{n-2}(r)} & =
\frac{2^n}{2^{n-2}}
\frac{\lambda^{\left\lfloor \frac{n}{2} \right\rfloor}}{\lambda^{\left\lfloor \frac{n-2}{2} \right\rfloor}}
\frac{(n-2)!!}{n!!}
\frac{r^{n}}{r^{n-2}}
\\ & = \frac{2^2\lambda}{n}\,r^2.
\end{split}
\end{equation}
On voit dans l'équation~\eqref{eq:volume_recurrence} que l'on peut obtenir le
volume d'une $n$-sphère simplement en multipliant la formule pour une
$(n-2)$-sphère par $r^2$ (un facteur requis par l'analyse dimensionnelle), en
divisant par $n$ et en multipliant par la «\ns constante de récurrence\ns »
$2^2\lambda$.

De même, pour la superficie on a
\begin{equation}
\label{eq:surface_area_recurrence}
\begin{split}
\frac{S_n(r)}{S_{n-2}(r)} & =
\frac{2^n}{2^{n-2}}
\frac{\lambda^{\left\lfloor \frac{n}{2} \right\rfloor}}{\lambda^{\left\lfloor \frac{n-2}{2} \right\rfloor}}
\frac{(n-2-2)!!}{(n-2)!!}
\frac{r^{n}}{r^{n-2}}
\\ & = \frac{2^2\lambda}{n-2}\,r^2,
\end{split}
\end{equation}
avec la même constante de récurrence $2^2\lambda$.

Ainsi, dans l'équation~\eqref{eq:volume_recurrence} et
l'équation~\eqref{eq:surface_area_recurrence} tous les deux, la constante qui
relie les différentes dimensions n'est pas $\lambda$ lui-même mais plutôt la
combinaison $2^2\lambda$. En comparant avec l'équation~\eqref{eq:tau_lambda}, on
voit que ce n'est autre que $\tau$\ns! En effet, une
\href{https://en.wikipedia.org/wiki/Volume_of_an_n-ball#The_two-dimension_recursion_formula}{dérivation
alternative} de la récurrence du volume par calcul direct (qui utilise $R$ où
l'on écrit $r$) se conclut par l'intégrale
\begin{equation}
\label{eq:integral_recurrence}
\begin{split}
V_n(R) & = \int_0^\tau \int_0^R V_{n-2}\left(\sqrt{R^2 - r^2}\right) \,r\,dr\,d\theta \\
       & = \tau V_{n-2}(R) \left[-\frac{R^2}{n}\left(1 - \left(\frac{r}{R}\right)^2\right)^\frac{n}{2}\right]_{0}^{R} \\
       & = \frac{\tau R^2}{n} V_{n-2}(R),
\end{split}
\end{equation}
ce qui montre ainsi que l'identification de $\tau$ comme «\ns constante de
récurrence\ns » n'est pas un hasard\ns; la constante de récurrence et la constante
du cercle sont vraiment une seule et même chose~:
\[
\begin{split}
\tau & = \mbox{constante du cercle} \\
     & = \mbox{constante de récurrence} = 2^2\lambda.
\end{split}
\]
Par conséquent, c'est $\tau$, et non $\lambda$, qui fournit le fil conducteur
qui relie les deux familles de solutions paires et impaires, comme l'illustre
Joseph Lindenberg sur
\href{https://sites.google.com/site/taubeforeitwascool/}{Tau Before It Was Cool}
(figure~\ref{fig:Nspheres})\ns\footnote{\href{https://sites.google.com/site/taubeforeitwascool/}{Tau
Before It Was Cool} écrit en fait la récurrence en termes de $2\pi$\ns; la
version illustrée à la figure~\ref{fig:Nspheres} a été créée pour moi sur
demande spéciale. Comme toujours, je suis très reconnaissant à Joseph Lindenberg
pour sa générosité et son soutien continus.}.

\begin{figure}
\begin{center}
\image{images/figures/Nspheres.png}
\end{center}
\caption{Récurrences de superficie et de volume.\label{fig:Nspheres}}
\end{figure}

Lorsque l'on discute des sphères générales à $n$ dimensions, pour des raisons de
commodité, on écrira les formules de superficie et de volume en termes de
$\lambda$ comme dans l'équation~\eqref{eq:surface_area_lambda} et
l'équation~\eqref{eq:volume_lambda}, mais pour n'importe quel $n$ déterminé, on
exprimera les résultats en termes de constante de récurrence $\tau$.

% subsubsection recurrences (end)


  % subsection volume_of_a_hypersphere (end)

  \subsection{Trois familles de constantes} % (fold)
  \label{sec:three_families_of_constants}

Équipé des outils développés dans la section~\ref{sec:volume_of_a_hypersphere},
on est maintenant prèt à aller au fond de $\pi$ et $\tau$. Pour terminer, on
utilisera l'équation~\eqref{eq:surface_area_lambda} et
l'équation~\eqref{eq:volume_lambda} pour définir deux familles de constantes,
puis on utilisera la définition de $\pi$ (équation~\eqref{eq:pi}) pour en
définir une troisième, révéilant ainsi exactement ce qui ne va pas avec $\pi$.

Tout d'abord, on définira une famille de «\ns constantes de superficie\ns » $\tau_n$
en divisant \linebreak l'équation~\eqref{eq:surface_area_lambda} par $r^{n-1}$,
la puissance de $r$ nécessaire pour produire une constante sans dimension pour
chaque valeur de~$n$~:
\begin{equation}
\label{eq:surface_area_constants}
\tau_n \equiv \frac{S_n(r)}{r^{n-1}} = \frac{2^n\,\lambda^{\left\lfloor \frac{n}{2} \right\rfloor}}{(n-2)!!}
\end{equation}
Deuxièmement, on définira une famille de «\ns constantes de volume\ns » $\sigma_n$
en divisant la formule de volume, l'équation~\eqref{eq:volume_lambda}, par
$r^n$, produisant à nouveau une constante sans dimensions pour chaque valeur
de~$n$~:
\begin{equation}
\label{eq:volume_constants}
\sigma_n \equiv \frac{V_n(r)}{r^n} = \frac{2^n\,\lambda^{\left\lfloor \frac{n}{2} \right\rfloor}}{n!!}.
\end{equation}
Avec les deux familles de constantes définies dans
l'équation~\eqref{eq:surface_area_constants} et
l'équation~\eqref{eq:volume_constants}, on peut écrire les formules de
superficie et de volume (l'équation~\eqref{eq:surface_area_lambda} et
l'équation~\eqref{eq:volume_lambda}) de manière compacte comme suit~:
\[ S_n(r) = \tau_n\,r^{n-1} \]
et
\[ V_n(r) = \sigma_n\,r^n. \]
En raison de la relation $V_n(r) = \int S_n(r)\,dr$, on a la relation simple
\[
\sigma_n = \frac{\tau_n}{n}.
\]

Faisons quelques observations sur ces deux familles de constantes. La
famille $\tau_n$ a une signification géométrique importante~: en fixant $r=1$
dans l'équation~\eqref{eq:surface_area_constants}, on voit que chaque $\tau_n$
est la superficie d'une $n$-sphère unité, qui est aussi la mesure d'angle d'une
$n$-sphère complète. En particulier, en écrivant $s_n(r)$ comme la «\ns longueur
d'arc\ns » à $n$ dimensions égale à une fraction~$f$ de la superficie
totale~$S_n(r)$, on a
\[
\theta_n \equiv \frac{s_n(r)}{r^{n-1}} = \frac{f S_n(r)}{r^{n-1}} = f\left(\frac{S_n(r)}{r^{n-1}}\right) = f\tau_n.
\]
Ici, $\theta_n$ est simplement la généralisation à $n$ dimensions de la mesure
d'angle radian, et on voit que $\tau_n$ est la généralisation d'«\ns un tour\ns » à
$n$ dimensions, ce qui explique pourquoi la constante de la $2$-sphère (du
cercle) $\tau_2 = 2^2 \lambda = \tau$ conduit naturellement au diagramme
illustré à la figure~\ref{fig:tau_angles}. De plus, on a appris à la
section~\ref{sec:volume_of_a_hypersphere} que $\tau_2$ est aussi la «\ns constante
de récurrence\ns » pour les superficies et les volumes des $n$-sphères.

Pendant ce temps, les $\sigma_n$ sont les volumes des $n$-sphères unité. En
particulier, $\sigma_2$ est la superficie d'un disque unité~:
\[
\sigma_2 = \frac{\tau_2}{2} = \frac{\tau}{2}.
\]
Cela montre que $\sigma_2 = \tau/2 = 3{,}141\,59\ldots$ a en fait une
signification géométrique indépendante. On constante cependant que \emph{cela
n'a rien à voir avec les circonférences ou les diamètres}. En d'autres termes,
\emph{$\pi = C/D$ n'est pas membre de famille $\sigma_n$}.

Alors, à quelle famille de constantes appartient naturellement $\pi$\ns?
Reformulons l'équation~\eqref{eq:pi} en termes plus appropriés pour la
généralisation à des dimensions supérieures~:
\[
\pi = \frac{C}{D} = \frac{S_2}{D^{2-1}}.
\]
On voit ainsi que $\pi$ est naturellement associé à des superficies divisées par
la puissance du diamètre nécessaire pour donner une constante sans dimension.
Cela suggère d'introduire une troisième famille de constantes~$\pi_n$~:
\begin{equation}
\label{eq:diameter_constants}
\pi_n \equiv \frac{S_n(r)}{D^{n-1}}.
\end{equation}
On peut exprimer cela en termes de famille $\tau_n$ en substituant $D = 2r$ dans
l'équation~\eqref{eq:diameter_constants} et en appliquant
l'équation~\eqref{eq:surface_area_constants}~:
\[
\pi_n = \frac{S_n(r)}{D^{n-1}} = \frac{S_n(r)}{(2r)^{n-1}} =
\frac{S_n(r)}{2^{n-1}r^{n-1}} = \frac{\tau_n}{2^{n-1}}.
\]

On est enfin en mesure de comprendre exactement ce qui ne va pas avec $\pi$. La
signification géométrique principale de $3{,}141\,59\ldots$ est qu'il est la
superficie d'un disque unité. Mais ce nombre provient de l'évaluation de
$\sigma_n = \tau_n/n$ lorsque $n=2$~:
\[
\sigma_2 = \frac{\tau_2}{2} = \frac{\tau}{2}.
\]
Il est vrai que cela est égale à $\pi_2$~:
\[
\pi_2 = \pi = \frac{\tau_2}{2^{2-1}} = \frac{\tau}{2}.
\]
Mais cette égalité est une coïncidence~: elle ne se produit que parce qu'il se
trouve que $2^{n-1}$ est égal à $n$ lorsque $n=2$ (c'est-à-dire $2^{2-1} = 2$).
Dans toutes les dimensions supérieures, $n$ et $2^{n-1}$ sont distincts. En
d'autres termes, \emph{la signification géométrique de $\pi$ est le résultat
d'un jeu de mots mathématique}.

  % subsection three_families_of_constants (end)

\section{Conclusion}
\label{sec:conclusion}

Au fils des ans, j'ai entendu de nombreux arguments contre $\pi$ étant mauvais
et contre $\tau$ étant bon, donc avant de conclure la discussion, je répondrai à
certains des questions les plus fréquemment posés.

  \subsection{Questions fréquemment posées} % (fold)
  \label{sec:faq}

\begin{itemize}

  \item \textbf{Êtes-vous sérieux\ns?} \\ Bien sûr. Je veux dire, je m'amuse avec
  ça, et la ton est parfois léger, mais il y a un but sérieux. Régler la
  constante du cercle égale à la circonférence sur le diamètre est une
  convention maladroite et déroutante. Bien que j'aimerais voir les
  mathématiciens changer de façon de faire, je ne suis pas particulièrement
  inquiet pour eux\ns; ils peuvent prendre soin d'eux-mêmes. Ce sont les
  néophytes qui m'inquiètent le plus, car ils subissent de plein fouet les
  dégâts~: comme indiqué dans la section~\ref{sec:circles_and_angles}, $\pi$ est
  un désastre pédagogique. Essayez d'expliquer à un enfant de douze ans (ou à un
  trentenaire) pourquoi la mesure d'angle pour un huitième de cercle (une 
  tranche de pizza) est $\pi/8$. Attendez, je veux dire $\pi/4$. Vous voyez ce
  que je veux dire\ns? C'est de la folie\textellipsis\ folie pure.

  \item \textbf{Comment passer de $\pi$ à $\tau$\ns?} \\ La prochaine fois que
  vous écrivez quelque chose qui utilisa la constante du cercle, dites
  simplement «\ns Par commodité, on définit $\tau=2\pi$\ns », puis procédez comme
  d'habitude. (Bien sûr, cela pourrait simplement conduire à la question~:
  «\ns Pourquoi voudrait-on faire cela\ns?\ns » Et j'admets que ce serait bien
  d'avoir un endroit où les pointer. Si seulement quelqu'un écrivait, disons, un
  \emph{manifeste} sur le sujet\ldots) La façon d'amener les gens à commencer à
  utiliser $\tau$ est de commencer à l'utiliser vous-même.

  \item \textbf{N'est-il pas trop tard pour passer\ns? Ne faudrait-il pas
  réécrire tous les manuels et articles mathématiques\ns?} \\ Non sur les deux
  points. Il est vrai que certaines conventions, bien que regrettables, sont
  effectivement irréversibles. Par exemple, le choix de Benjamin Franklin pour
  les signes de charges électriques conduit à l'exemple le plus familier de
  courant électrique (c'est-à-dire, les électrons libres dans les métaux) étant
  positif lorsque les porteurs de charge sont négatifs, et vice
  versa\ns; maudissant ainsi les étudiants débutants en physique avec des signes
  négatifs déroutants depuis\ns\footnote{On ne pouvait pas déterminer le signe
  des porteurs de charge avec la technologie de l'époque de Franklin, donc ce
  n'est pas de sa faute. C'est juste de la malchance.}. Pour changer cette
  convention, il \emph{faudrait} réécrire tous les manuels (et brûler les
  anciens) car il est impossible de dire d'un coup d'œil quelle convention est
  utilisée. En revanche, alors que la redéfinition de $\pi$ est effectivement
  impossible, nous pouvons passer à la volée de $\pi$ à $\tau$ en utilisant la
  conversion \[ \pi \leftrightarrow \textstyle{\frac{1}{2}}\tau. \] C'est
  purement une question de substitution mécanique, complètement robuste et en
  effet entièrement réversible. Le passage de $\pi$ à $\tau$ peut donc se faire
  de manière incrémentale\ns; contrairement à une redéfinition, cela ne doit pas
  nécessairement se produire d'un seul coup.

  \item \textbf{L'utilisation de $\tau$ ne déroutera-t-elle pas les gens, en
  particulier les étudiants\ns?} \\ Si vous êtes assez intelligent pour
  comprendre la mesure d'angle en radians, vous êtes assez intelligent pour
  comprendre $\tau$\ns;et pourquoi $\tau$ est en fait \emph{moins} déroutant que
  $\pi$. De plus, il n'y a rien de intrinsèquement déroutant dans l'énoncé
  «\ns Soit $\tau=2\pi$\ns »\ns; compris étroitement, c'est juste une simple
  substitution. Enfin, nous pouvons saisir la situation comme une opportunité
  d'enseignement~: l'idée que $\pi$ pourrait être mauvais est
  \emph{intéressante}, et les élèves peuvent s'engager avec le matériel en
  convertissant les équations de leurs manuels de $\pi$ à $\tau$ pour voir par
  eux-mêmes quel choix est le meilleur.

  \item \textbf{Est-ce que cela importe vraiment\ns?} \\ Bien sûr, cela importe.
  \emph{La constante du cercle est importante.} Les gens s'en soucient
  suffisamment pour écrire des livres entiers sur le sujet, pour le célébrer un
  jour particulier chaque année et pour mémoriser des dizaines de milliers de
  ses chiffres. Je me soucie suffisamment d'écrire un manifeste entier, et vous
  vous souciez suffisamment de le lire. C'est précisément parce qu'\emph{il
  importe} qu'il soit difficile d'admettre que la convention actuelle est
  mauvais. (Je veux dire, comment expliquer à
  \href{https://www.guinnessworldrecords.com/world-records/most-pi-places-memorised}{Rajveer
  Meena}, un détenteur du record du monde, qu'il vient de réciter 70\,000
  chiffres de la moitié de la vraie constante du cercle\ns?) Puisque la constante
  du cercle est importante, il est important qu'elle soit bonne, et nous avons
  vu dans ce manifeste que le bon nombre est $\tau$. Bien que $\pi$ soit d'une
  grande importance \emph{historique}, la signification \emph{mathématique} de
  $\pi$ est qu'il est la moitié de $\tau$.

  \item \textbf{Pourquoi quelqu'un a-t-il déjà utlilsé $\pi$ en premier lieu\ns?}
  \\ En tant que notation, $\pi$ a été popularisé il y a environ 300 ans par
  \href{https://fr.wikipedia.org/wiki/Leonhard_Euler}{Leonhard Euler} (d'après
  les travaux de
  \href{https://fr.wikipedia.org/wiki/William_Jones_(mathématicien)}{William
  Jones}), mais les origines des $\pi$-le-nombre se perdent dans la nuit des
  temps. Je soupçonne que la convention d'utiliser $C/D$ au lieu de $C/r$ est
  née simplement parce qu'il est plus facile de \emph{mesurer} le diamètre d'un
  objet circulaire que de mesurer son rayon. Mais cela ne le rend pas de bonnes
  mathématiques, et je suis surpris qu'Archimède, qui \href{http://itech.fgcu.edu/faculty/clindsey/mhf4404/archimedes/archimedes.html}{a
  fameusement approximé la constante du cercle}, n'ait pas réalisé que $C/r$ est
  le nombre le plus fondamental. Je suis encore plus surpris qu'Euler n'ait pas
  corrigé le problème quand il en avait l'occasion\ns; contrairement à Archimède,
  Euler avait l'avantage de la notation algébrique moderne, qui (comme nous
  l'avons vu à partir de la section~\ref{sec:circles_and_angles}) rend les
  relations sous-jacentes entre les cercles et la constante du cercle très
  claires. Incroyablement, Euler a en fait utilisé le symbole $\pi$ pour
  signifier $C/D$ \emph{ou} $C/r$ \href{https://bit.ly/2OERR3G}{à des moments
  différents}\ns! Quel dommage qu'il n'ait pas standardisé le choix le plus
  pratique.

  \item \textbf{Pourquoi êtes-vous intéressé par ce sujet\ns?} \\ D'abord, en
  tant que chercheur de vérité, je me soucie de l'exactitude des explications.
  Deuxièmement, en tant que professeur, je me soucie de la clarté de
  l'exposition. Troisièmement, en tant que hacker, j'aime un bon hack.
  Quatrièmement, ent tant qu'étudiant en histoire et en nature humaine, je
  trouve fascinant que l'absurdité de $\pi$ soit restée bien en vue pendant des
  siècles avant que quelqu'un ne semble le remarquer. De plus, beaucoup des gens
  qui ont loupé la vrai constante du cercle sont parmi les personnes les plus
  rationnelles et intelligentes à avoir jamais vécu. Quoi d'autre pourrait nous
  regarder en face, attendant juste que nous le découvrions\ns?

  \item \textbf{Êtes-vous, genre, un fou\ns?} \\ Ce n'est vraiment pas votre
  affaire, mais non. Hormis le port occasionnel de
  \href{https://fr.wikipedia.org/wiki/Chaussure_à_orteils}{chaussures
  inhabituelles}, je suis pour toutes les apparences extérieures normal à tous
  points de vue. On ne devinerait jamais que, loin d'être un citoyen ordinaire,
  je suis en fait un propagandiste mathématique notoire.

  \item \textbf{Mais qu'en est-il des jeux de mots\ns?} \\ Nous arrivons
  maintenant à la dernière objection. Je sais, je sais,
  «\ns $\pi$ in the sky\ns »\ns\footnote{N.\ du T.~: Ce jeu de mots vient du fait
  qu'en anglais, la lettre grecque $\pi$ se prononce comme homophone de
  \emph{pie}
  \href{https://fr.wikipedia.org/wiki/Alphabet_phonétique_international}{/paɪ/},
  ce qui signifie «\ns tarte\ns »\ns;
  \href{https://en.wiktionary.org/wiki/pie_in_the_sky}{\emph{pie in the sky}}
  est un idiome anglais qui signifie «\ns une notion fantaisiste\ns ».} est
  tellement très ingénieux. Et pourtant, $\tau$ lui-même regorge de
  possibilités. Le $\tau$isme\ns\footnote{N.\ du T.~: En anglais, la
  prononciation de la lettre grecque $\tau$ (\href{https://en.wiktionary.org/wiki/pie_in_the_sky}{/taʊ/}) correspond à une prononciation alternative courante de \emph{Tao}.} nous dit~:
  ce n'est pas $\tau$ qui est un morceau de $\pi$, mais $\pi$ qui est un morceau
  de $\tau$\ns; la moitié de $\tau$, pour être exact. L'identité $e^{i\tau} = 1$
  dit~: «\ns \emph{Soyez un avec le $\tau$.}\ns » Et même si l'observation selon
  laquelle «\ns \emph{une rotation d'un tour vaut~1}\ns » peut sembler une
  $\tau$-tologie, c'est la vraie nature du $\tau$. En contemplant cette nature
  pour chercher le \href{https://fr.wikipedia.org/wiki/Tao_(culture)}{chemin} du
  $\tau$, il faut nous rappeler que le $\tau$isme est basé sur la raison, pas
  sur la foi~: les $\tau$istes ne sont jamais $\pi$eux.

\end{itemize}

  % subsection faq (end)

  \subsection{Embrassez le tau} % (fold)
  \label{sec:embrace_the_tau}

On a vu dans \emph{Le Manifeste de tau} que le choix naturel pour la constante
du cercle est le rapport de la circonférence d'un cercle non pas à son diamètre,
mais à son rayon. Ce nombre a besoin d'un nom, et j'espère que les autres se
joindront à moi pour l'appeler~$\tau$~:
\[
\begin{split}
\mbox{constante du cercle} = \tau & \equiv \frac{C}{r} \\
                                  & = 6{,}283\,185\,307\,179\,586\ldots
\end{split}
\]
L'usage est naturel, la motivation est claire et les implications sont
profondes. De plus, il est livré avec un diagramme vraiment cool
(figure~\ref{fig:tauism}). On voit sur la figure~\ref{fig:tauism} un mouvement à
travers le \emph{yang} («\ns clair, blanc, montant\ns ») jusqu'à $\tau/2$ et un
retour à travers le \emph{yin} («\ns sombre, noir, descendant\ns ») à
$\tau$\ns\footnote{Les interprétations du yin et du yang citées ici sont tirées
de \emph{Zen Yoga: A Path to Enlightenment through Breathing, Movement and
Meditation} (\emph{Yoga zen~: Un chemin vers l'illumination par la respiration,
le mouvement et la méditation}) par Aaron Hoopes.}. Utiliser $\pi$ au lieu de
$\tau$, c'est comme avoir le yang sans le yin.

\begin{figure}
\begin{center}
\image{images/figures/tauism_rotated.pdf}
\end{center}
\caption{Les disciples du $\tau$isme cherchent le chemin du
$\tau$.\label{fig:tauism}}
\end{figure}

% Si jamais on s'entend dire des choses comme «\ns  Parfois $\pi$ est le meilleur choix, et parfois c'est $2\pi$\ns », on devrait s'arrêter et se souvenir des paroles de \href{http://vihart.com/}{Vi Hart} dans sa magnifique \href{https://www.youtube.com/watch?v=jG7vhMMXagQ}{vidéo sur $\tau$}~: «\ns Non\ns! Vous cherchez des excuses pour $\pi$.\ns » Il est temps d'arrêter de chercher des excuses.

  % subsection conclusion (end)

  \subsection{Jour de tau} % (fold)
  \label{sec:tau_day}

\emph{Le Manifeste de tau} a été lancé pour la première fois le
\href{https://tauday.com/}{Jour de tau}~: 28 juin (écrit 6/28 aux États-Unis)
2010. Le Jour de tau est un moment pour célébrer et se réjouir de tout ce qui
concerne les mathématiques\ns\footnote{Puisque 6 et 28 sont les deux premiers
\href{https://fr.wikipedia.org/wiki/Nombre_parfait}{\emph{nombres parfaits}},
6/28 est en fait une journée \emph{parfaite}.}. Si vous voudriez recevoir des
mises à jour sur $\tau$, y compris des notifications sur les futurs événements
possibles du Jour de tau, veuillez vous inscrire à la liste de diffusion du
\emph{Manifeste de tau} ci-dessous. Et si vous pensez que les pâtisseries
circulaires du Jour de pi sont goûteuses, attendez simplement~: le Jour de tau a
deux fois plus de \og pi(e)\ns »\ns\footnote{N. de T.~: Un autre jeu de mots\ns;
voir note 26 de bas de page.}\ns!

%= <!-- insert_user_engagement -->

  \subsubsection{Remerciements} % (fold)
  \label{sec:acknowledgments}

Je voudrais tout d'abord remercier \href{https://www.math.utah.edu/~palais}{Bob
Palais} d'avoir écrit «\ns $\pi$ Is Wrong!\ns » Je ne me souviens pas à quel point
mes soupçons sur $\pi$ étaient profonds avant de rencontrer cet article, mais
«\ns $\pi$ Is Wrong!\ns » certainement ouvert mes yeux, et chaque section du
\emph{Manifeste de tau} lui doit une dette de gratitude. Je voudrais aussi
remercier Bob pour ses commentaires utiles sur ce manifeste, et surtout pour sa
gentillesse à ce sujet.

Je pense au \emph{Manifeste de tau} depuis un certain temps maintenant, et bon
nombre des idées présentées ici ont été développées au cours de conversations
avec mon ami Sumit Daftuar. Sumit a agi comme une caisse de résonance et un
avocat occasionnel du Diable, et sa perspicacité et en tant que mathématicien a
influencé ma pensée de plusieurs manières.

J'ai également reçu des encouragements et des commentaires utiles de plusieurs
lecteurs. Je voudrais remercier
\href{https://www.youtube.com/watch?v=jG7vhMMXagQ}{Vi Hart} et
\href{https://www.youtube.com/watch?v=3174T-3-59Q}{Michael Blake} pour leurs
incroyables vidéos inspirées de $\tau$, ainsi que Don «\ns Blue\ns » McConnell et
Skona Brittain pour avoir aidé $\tau$ à faire partie de la culture du binoclard
(via \href{http://tauclock.com/}{l'application pour iPhone de temps-en-$\tau$}
et \href{http://www.sbcrafts.net/clocks/}{l'horloge tau}, respectivement).
L'interprétation agréable du symbole yin-yang utilisé dans \emph{Le Manifeste de
tau} est due à une suggestion de \href{http://www.harremoes.dk/Peter/}{Peter
Harremoës}, qui (comme indiqué ci-dessus) a la rare distinction d'avoir proposé
indépendamment d'utiliser $\tau$ pour la constante du cercle. Un autre tauiste
pré--\emph{Manifeste de tau},
\href{https://sites.google.com/site/taubeforeitwascool/}{Joseph Lindenberg}, a
également été un fervent partisan, et son enthousiasme est très apprécié. J'ai
reçu plusieurs bonnes suggestions de
\href{https://christopherolah.wordpress.com/about-me}{Christopher Olah}, et la
section~\ref{sec:eulerian_identities} sur les identités eulériennes était
inspirée par une excellente suggestion de Timothy «\ns Patashu\ns » Stiles.
\href{http://www.blahedo.org/blog/archives/001083.html}{Don Blaheta} a anticipé
et inspiré une partie du matériel sur les hypersphères, et
\href{https://bit.ly/2WyHqmK}{John Kodegadulo} l'a assemblé d'une manière
particulièrement claire et divertissante. Puis Jeff Cornell, avec son
observation sur l'importance de $\tau/4$ dans ce contexte, a ébranlé ma foi et
m'a époustouflé.

Enfin, je voudrais remercier \href{https://techiferous.com/}{Wyatt Greene} pour
ses commentaires extraordinairement utiles sur une ébauche du manifeste avant le
lancement\ns; entre autres choses, si jamais vous avez besoin de quelqu'un pour
vous dire que «\ns pratiquement toute de la section 5 [maintenant supprimée] est
de la merde totale\ns », Wyatt est l'homme de la situation.


    % subsubsection acknowledgments (end)

    \subsubsection{À propos de l'auteur} % (fold)
    \label{sec:about_the_author}

    % subsection about_the_author (end)

\href{https://www.michaelhartl.com/}{Michael Hartl} est un éducateur, un auteur
et un entrepreneur. Il est cofondateur et auteur principal de
\href{https://www.learnenough.com/}{Learn Enough} et du
\href{https://www.railstutorial.org/}{\emph{Tutoriel Ruby on Rails}}.
Auparavant, il a enseigné la physique théorique et computationnelle au
\href{https://www.caltech.edu/}{California Institute of Technology (Caltech)},
où il a reçu le \href{https://www.michaelhartl.com/ascit/awards2000.html}{Prix
d'excellence pour l'enseignement dans la vie} et été éditeur de Caltech pour
\href{https://www.feynmanlectures.caltech.edu/}{\emph{Le Cours de physique de
Feynman}}. Il est diplômé du \href{https://college.harvard.edu/}{Harvard
College}, il a un \href{https://thesis.library.caltech.edu/1940/}{doctorat en
physique} de Caltech et il est ancien élève du programme d'entrepreneur
\href{https://ycombinator.com/}{Y Combinator}.

Michael a honte d'admettre qu'il connaît $\pi$ à 50 décimales,
\href{\#fig-futurama_video}{soit environ 48 de plus que Matt Groening}. Pour
expier cela, il a mémorisé
\href{https://www.wolframalpha.com/input/?i=N[2+Pi,+53]}{52 décimales} de
$\tau$.

    \subsubsection{Copyright} % (fold)
    \label{sec:copyright_and_license}

    \emph{Le Manifeste de tau}. Copyright \copyright\ 2010--2018 par Michael
	Hartl. Des versions de livres électroniques du \emph{Manifeste de tau} sont
	disponibles à l'achat sur le site de vente du \emph{Manifeste de tau}.
	N'hesitez pas à imprimer et à distribuer des copies du \emph{Manifeste de
	tau} pour une utilisation en classe ou similaire.

    % subsubsection copyright_and_license (end)
