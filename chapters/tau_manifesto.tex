\section{La constante du cercle} % (fold)
\label{sec:the_circle_constant}

\href{https://tauday.com/tau-manifesto}{\emph{Le Manifeste de tau}} est dédié à l'un des nombres les plus importants en mathématiques, peut-être \emph{le} plus important~: la \emph{constante du cercle}, qui relie la circonférence d'un cercle à sa dimension linéaire. Depuis des millénaires, le cercle est considéré la plus parfaite des formes, et la constante du cercle capture la géométrie du cercle dans un seul nombre. Bien sûr, le choix traditionnel pour la constante du cercle est $\pi$---mais, comme le note le mathématicien \href{https://www.math.utah.edu/~palais/}{Bob Palais} dans son article délicieux «\,$\pi$ Is Wrong!\,» («\,$\pi$ est mauvais\,»)\,\footnote{Palais, Robert. «\,$\pi$ Is Wrong!\,», \emph{The Mathematical Intelligencer}, volume~23, numéro~3, 2001, pages~7--8. De nombreux des arguments du \emph{Manifeste de tau} sont basés sur ou sont inspirés par «\,$\pi$ Is Wrong!\,». Il est disponible en ligne sur \href{https://www.math.utah.edu/~palais/pi.html}{https://www.math.utah.edu/~palais/pi.html}.}, $\pi$ \emph{est mauvais}. Il est temps de régler les choses.

  \subsection{Immodeste Proposition} % (fold)
  \label{sec:an_immodest_proposal}

On commence à réparer les dommages causés par $\pi$ en comprenant le nombre notoire lui-même. La définition traditionnelle pour la constante du cercle définit $\pi$ (pi) comme égal au rapport de la circonférence d'un cercle (longueur) à son diamètre (largeur)\,\footnote{Le symbole $\equiv$ signifie «\,défini comme\,».}.
\begin{equation}
\label{eq:pi}
\pi \equiv \frac{C}{D} = 3{,}141\,592\,65\ldots
\end{equation}
Le nombre $\pi$ a de nombreuses propriétés remarquables---entre autres, il est \href{https://fr.wikipedia.org/wiki/Nombre_irrationnel}{\emph{irrationnel}} et même \href{https://fr.wikipedia.org/wiki/Nombre_transcendant}{\emph{transcendant}}---et sa présence dans les formules mathématiques est répandue.

\begin{figure}
\image{images/figures/circle.pdf}
\caption{Anatomie d'un cercle.\label{fig:circle}}
\end{figure}

Ça devrait être évident que $\pi$ n'est pas «\,mauvais\,» dans le sens où il est factuellement incorrect\,; le nombre $\pi$ est parfaitement bien défini, et il possède toutes les propriétés qui lui sont normalement attribuées par les mathématiciens. Quand on dit que «\,$\pi$ est mauvais\,», on veut dire que \emph{$\pi$ est un choix déroutant et contre nature pour la constante du cercle}. En particulier, un cercle est défini comme l'ensemble des points à une distance fixe, le \emph{rayon}, d'un point déterminé, le \emph{centre} (la figure~\ref{fig:circle}). Tandis qu'il existe une infinité de formes avec une largeur constante (la figure~\ref{fig:constant_width})\,\footnote{Image récupérée de \href{https://commons.wikimedia.org/wiki/File:Reuleaux_triangle_roll.gif}{Wikimedia} le 12 mars 2019. Copyright © 2016 par Ruleroll et utilisé sans modification selon les termes de la licence \href{https://creativecommons.org/licenses/by-sa/4.0/deed.fr}{Attribution - Partage dans les Mêmes Conditions 4.0 International de Creative Commons}.}, il n'y a qu'une seule forme avec un rayon constant. Cela suggère qu'une définition plus naturelle de la constante du cercle pourrait utiliser $r$ à la place de~$D$~:
\begin{equation}
\label{eq:circle_constant}
\mbox{constante du cercle} \equiv \frac{C}{r}.
\end{equation}
Parce que le diamètre d'un cercle est le double de son rayon, ce nombre est numériquement équivalent à $2\pi$. Comme $\pi$, il est transcendant et donc irrationnel, et (comme on le verra dans la section~\ref{sec:the_number_tau}) son utilisation en mathématiques est également répandue.

\begin{figure}
\image{images/figures/Reuleaux_triangle_roll.pdf}
\caption{L'une des infinies formes non circulaires à largeur constante.\label{fig:constant_width}}
\end{figure}

Dans «\,$\pi$ Is Wrong!\,», Bob Palais plaide de manière convaincante en faveur de la seconde de ces deux définitions de la constante du cercle\,; et, à mon avis, il mérite le crédit principal pour l'identification de ce problème et pour le présenter à un large public. Il appelle la vraie constante du cercle «\,un tour\,», et il introduit aussi un nouveau symbole pour le réprésenter (la figure~\ref{fig:palais_tau}). Comme on verra, la description est presciente, mais malheureusement, le symbole est plutôt étrange et (comme discuté dans la section~\ref{sec:conflict_and_resistance}) il semble peu probable qu'il soit largement adopté. (\emph{Mise à jour~:} Cela s'est en effet avéré être le cas, et Palais lui-même est depuis lors devenu un fervent partisan des arguments de ce manifeste.)

\begin{figure}
\imagebox{images/figures/palais-tau.png}
\caption{Le symbole étrange de la constante du cercle de «\,$\pi$ Is Wrong!\,».\label{fig:palais_tau}}
\end{figure}

\emph{Le Manifeste de tau} est dédié à la proposition selon laquelle la bonne réponse à «\,$\pi$ est mauvais\,» est «\,Non, \emph{vraiment}\,». Et la vraie constante du cercle mérite un nom approprié. Comme on l'a peut-être deviné, \emph{Le Manifeste de tau} propose que ce nom soit la lettre grecque $\tau$ (tau)~:

\begin{equation}
\label{eq:tau}
\tau \equiv \frac{C}{r} = 6{,}283\,185\,307\,179\,586\ldots
\end{equation}
Tout au long du reste de ce manifeste, on verra que le \emph{nombre} $\tau$ est le bon choix, et l'on montrera par l'usage (la section~\ref{sec:the_number_tau} et la section~\ref{sec:circular_area}) et par l'argumentation directe (la section~\ref{sec:conflict_and_resistance}) que la \emph{lettre} $\tau$ est aussi un choix naturel.

\subsection{Un ennemi puissant} % (fold)
 \label{sec:a_powerful_enemy}
 
Avant de procéder à la démonstration que $\tau$ est le choix naturel pour la constante du cercle, on fait d'abord connaître ce que l'on affronte---car il y a une puissante conspiration, vieille de plusieurs siècles, déterminée a propager la propagande pro-$\pi$. Des \href{https://www.amazon.fr/fascinant-nombre-ESSAIS-SCIEN-ebook/dp/B07D3SYP5X/}{livres} \href{https://www.amazon.fr/1-000-000-Décimales-Nombre-Plus-Connu/dp/B085DMM9XV/}{entiers} \href{https://www.amazon.fr/Autour-du-nombre-pi-HR-ACT-SC-INDUS-ebook/dp/B081HGQSQJ/}{sont écrits} qui vantent les vertus de $\pi$. (Je veux dire, \href{https://www.amazon.com/exec/obidos/ISBN=0802713327/}{\emph{des livres}\,!}) Et la dévotion irrationnelle à $\pi$ s'est propagée même aux binoclards du plus haut niveau\,; par exemple, au «\,jour de pi\,» en 2010, \href{https://www.google.com/}{Google} \emph{a changé son logo} pour honorer $\pi$ (la figure~\ref{fig:google_pi_day.}).

\begin{figure}
\begin{center}
\image{images/figures/google_pi_day.png}
\end{center}
\caption{Le logo Google le 14 mars (écrit 3/14 aux États-Unis\,; «\,le Jour de pi\,») 2010.\label{fig:google_pi_day.}}
\end{figure}

Pendant ce temps, certaines personnes mémorisent des dizaines, des centaines, voire \href{https://www.guinnessworldrecords.com/world-records/most-pi-places-memorised}{\emph{des milliers}} de chiffres de ce nombre mystique. Qui voudrait jamais mémoriser même 40 chiffres de $\pi$, sauf un raté (la figure~\ref{fig:futurama_video})\,\footnote{La vidéo de la figure~\ref{fig:futurama_video} (disponible sur \url{https://vimeo.com/12914981}) est un extrait d'une conférence donnée par \href{https://cs.appstate.edu/~sjg/}{le Dr Sarah Greenwald}, professeur de mathématiques à \href{https://www.appstate.edu/}{l'Appalachian State University}. Le Dr Greenwald utilise des références mathématiques des \emph{Simpson} et de \emph{Futurama} pour susciter l'interêt de ses élèves, et pour les aider à surmonter leur anxiété mathématique. Elle est aussi la responsable du \href{https://cs.appstate.edu/~sjg/futurama/}{«\,\emph{Futurama} Math Page\,»}.}\,?

\begin{figure}
\begin{center}
%= insert_futurama_video
\includegraphics{images/figures/futurama_math_lecture.png} % html_ignore
\end{center}
\caption{\href{https://tauday.com/tau-manifesto/\#sec-about_the_author}{Michael Hartl} prouve que \href{https://fr.wikipedia.org/wiki/Matt_Groening}{Matt Groening} a tort en récitant $\pi$ à 40 décimales.\label{fig:futurama_video}}
\end{figure}

Vraiment, les partisans de $\tau$ font face à un puissant adversaire. Et pourtant, on a un allié puissant---car la vérité est de son côté.

% section the_most_important_number (end)

\section{Le nombre tau} % (fold)
\label{sec:the_number_tau}

On a vu dans la section 1.1 que le nombre $\tau$ peut aussi être écrit comme $2\pi$. Comme remarqué dans «\,$\pi$ Is Wrong!\,», il est donc d'un grand intérêt de découvrir que la combinaison $2\pi$ se produit avec une fréquence étonnante tout au long de mathématiques. Par exemple, on considère les intégrales sur tout l'espace en coordonnées polaires:

\[
  \int_0^{2\pi}\int_0^\infty f(r, \theta)\, r\, dr\, d\theta.
\]
La limite supérieure de l'intégration de $\theta$ est toujours $2\pi$. Le même facteur apparaît dans la définition de la \href{https://fr.wikipedia.org/wiki/Loi_normale}{loi gaussienne (normale)},
\[
  \frac{1}{\sqrt{2\pi}\sigma}e^{-\frac{(x-\mu)^2}{2\sigma^2}},
\]
et encore dans la \href{https://fr.wikipedia.org/wiki/Transformation_de_Fourier}{transformation de Fourier},
\[
  f(x) = \int_{-\infty}^\infty F(k)\, e^{2\pi ikx}\,dk
\]
\[
    F(k) = \int_{-\infty}^\infty f(x)\, e^{-2\pi ikx}\,dx.
\]
Il revient dans la \href{https://fr.wikipedia.org/wiki/Formule_intégrale_de_Cauchy}{formule intégrale de Cauchy},
\[
  f(a) = \frac{1}{2\pi i}\oint_\gamma\frac{f(z)}{z-a}\,dz,
\]
dans les \href{https://fr.wikipedia.org/wiki/Racine_de_l%27unité}{racines $n$ième de l'unité},
\[
  z^n = 1 \Rightarrow z = e^{2\pi i/n},
\]
et dans les valeurs de la \href{https://fr.wikipedia.org/wiki/Fonction_zêta_de_Riemann}{fonction zêta de Riemann} pour les entiers pairs positifs\,\footnote{Ici $B_n$ est le $n$ième \href{https://fr.wikipedia.org/wiki/Nombre_de_Bernoulli}{nombre de Bernoulli}.}~:
\[
\begin{split}
  \zeta(2n) & = \sum_{k=1}^\infty \frac{1}{k^{2n}} \\
            & = \frac{|B_{2n}|}{2(2n)!}\,(2\pi)^{2n},\qquad n = 1, 2, 3, \ldots
\end{split}
\]
Ces formules sont loin d'être les seuls exemples---on peut ouvrir son texte préféré de physique ou de mathématiques et l'essayer soi-même. Il en existe \href{http://www.harremoes.dk/Peter/Undervis/Turnpage/Turnpage1.html}{beaucoup d'autres}, et la conclusion est claire~: il y a quelque chose de spécial dans $2\pi$.

Pour aller au fond de ce mystère, il faut revenir aux premiers principes en considérant la nature des cercles, et surtout la nature des \emph{angles}. Bien qu'il soit probable qu'une grande partie de ce matériel soit familier, cela vaut la peine de le revoir, car c'est là que commence la véritable compréhension de $\tau$.

  \subsection{Cercles et angles} % (fold)
  \label{sec:circles_and_angles}

Il y a une relation intime entre les cercles et les angles, comme le montre la figure~\ref{fig:angle_arclength}. Puisque les cercles concentriques de la figure~\ref{fig:angle_arclength} ont des rayons différents, les lignes de la figure coupent différentes longueurs d'arc, mais l'angle~$\theta$ (thêta) est le même dans chaque cas. En d'autres termes, la taille de l'angle ne dépend pas du rayon du cercle utilisé pour définir l'arc. La tâche principale de la mesure d'angle est de créer un système qui capture cette invariance de rayon.

\begin{figure}
\begin{center}
\image{images/figures/angle-arclength.pdf}
\end{center}
\caption{Un angle $\theta$ avec deux cercles concentriques.\label{fig:angle_arclength}}
\end{figure}

Le système d'angle le plus élémentaire est peut-être les \emph{degrés}, qui divisent un cercle en 360 parties égales. Un resultat de ce système est l'ensemble des angles spéciaux (familiers aux étudiants en trigonométrie) montrés sur la figure~\ref{fig:degree_angles}.

\begin{figure}
\begin{center}
\image{images/figures/degree-angles.pdf}
\end{center}
\caption{Des angles spéciaux, en degrés.\label{fig:degree_angles}}
\end{figure}

Un système plus fondamental de mesure d'angle implique une comparaison directe de la longueur d'arc $s$ avec le rayon $r$. Bien que les longueurs de la figure~\ref{fig:angle_arclength} diffèrent, la longueur de l'arc augmente proportionnellement au rayon, de sorte que le \emph{rapport} de la longueur de l'arc au rayon est le même dans chaque cas~:
\[
s\propto r \Rightarrow \frac{s_1}{r_1} = \frac{s_2}{r_2}.
\]
Cela suggère la définition suivante de la \emph{mesure d'angle radian}:
\begin{equation}
\label{eq:radians}
\theta \equiv \frac{s}{r}.
\end{equation}
Cette définition a la propriété requise d'invariance de rayon, et puisque $s$ et $r$ ont tous deux des unités de longueur, les radians sont \href{https://fr.wikipedia.org/wiki/Grandeur_sans_dimension}{\emph{sans dimension}} par construction. L'utilisation de la mesure d'angle radian conduit à des formules succinctes et élégantes tout au long des mathématiques\,; par exemple, la formule habituelle pour la dérivée de $\sin\theta$ n'est vraie que lorsque $\theta$ est exprimé en radians~:
\[
  \frac{d}{d\theta}\sin\theta = \cos\theta. \qquad\mbox{(seulement en radians)}
\]
Naturellement, les angles spéciaux de la figure~\ref{fig:degree_angles} peuvent être exprimés en radians, et lorsque l'on a appris la trigonométrie au lycée, on a probablement mémorisé les valeurs spéciales indiquées sur la figure~\ref{fig:pi_angles}. (J'appelle ce système de mesure «\,radians-de-$\pi$\,» pour souligner qu'ils sont écrits en termes de $\pi$.)

\begin{figure}
\begin{center}
\image{images/figures/pi-angles.pdf}
\end{center}
\caption{Des angles spéciaux, en radians-de-$\pi$.\label{fig:pi_angles}}
\end{figure}

\begin{figure}
\begin{center}
\image{images/figures/angle-fractions.pdf}
\end{center}
\caption{Les angles «\,spéciaux\,» sont des fractions d'un cercle complet.\label{fig:angle_fractions}}
\end{figure}

Un instant de réflexion montre que les angles dits «\,spéciaux\,» ne sont que des fractions rationnelles particulièrement simples d'un cercle complet, comme le montre la figure~\ref{fig:angle_fractions}. Cela suggère de revoir l'équation~\eqref{eq:radians}, en réécrivant la longueur d'arc~$s$ en termes de la fraction~$f$ de la circonférence complète~$C$, c'est-à-dire $s = f C$~:
\[ \theta = \frac{s}{r} = \frac{fC}{r} =  f\left(\frac{C}{r}\right) \equiv f\tau. \]
On remarque comment naturellement $\tau$ sort de cette analyse. Si l'on croit en $\pi$, je crains que le diagramme des angles spéciaux qui résulte (la figure~\ref{fig:tau_angles}) ébranle sa foi en son cœur.

\begin{figure}
\begin{center}
\image{images/figures/tau-angles.pdf}
\end{center}
\caption{Des angles spéciaux, en radians.\label{fig:tau_angles}}
\end{figure}

Bien qu'il existe de nombreux autres arguments en faveur de $\tau$, la figure~\ref{fig:tau_angles} peut être l'un qui est le plus frappante. On voit aussi sur la figure~\ref{fig:tau_angles} le génie de l'identification par Bob Palais de la constante du cercle comme «\,\href{https://fr.wikipedia.org/wiki/Tour_(angle)}{un tour}\,»~: $\tau$ est la mesure d'angle en radians pour un \emph{tour} d'un cercle. De plus, on remarque qu'avec $\tau$ il n'y a \emph{rien à mémoriser}~: une douzième de tour est $\tau/12$, un huitième de tour est $\tau/8$, et ainsi de suite. L'utilisation de $\tau$ on donne le meilleur des deux mondes en combinant la clarté conceptuelle avec tous les avantages concrets des radians\,; la signification abstraite de, disons, $\tau/12$ est évidente, mais c'est aussi juste un nombre~:
\[
\begin{split}
\mbox{un douzième de tour} = \frac{\tau}{12} & \approx \frac{6{,}283\,185}{12} \\
                                             & = 0{,}523\,598\,8.
\end{split}
\]
Enfin, en comparant la figure~\ref{fig:pi_angles} à la figure~\ref{fig:tau_angles}, on voit d'où viennent ces facteurs embêtants de 2$\pi$~: un tour de cercle vaut 1$\tau$, mais 2$\pi$. Numériquement, ils sont égaux, mais conceptuellement, ils sont assez distincts.

    \subsubsection{Les ramifications} % (fold)
    \label{sec:the_ramifications}

    % subsubsection the_ramifications (end)

Les facteurs inutiles de 2 qui résultent de l'utilisation de $\pi$ sont assez ennuyeux en eux-mêmes, mais beaucoup plus grave est leur tendance à \emph{annuler} lorsqu'ils sont divisés par un nombre pair. Les résultats absurdes, comme un \emph{demi}-$\pi$ pour un \emph{quart} de tour, obscurcissent la relation sous-jacente entre la mesure d'angle et la constante du cercle. À celui qui soutient que «\,peu n'importe\,» que l'on utilise $\pi$ ou $\tau$ pour enseigner la trigonométrie, je lui demande simplement de visualiser la figure ~\ref{fig:pi_angles}, la figure~\ref{fig:angle_fractions} et la figure~\ref{fig:tau_angles} à travers les yeux d'un enfant. On verra que, du point de vue d'un débutant, \emph{l'utilisation de $\pi$ au lieu de $\tau$ est un désastre pédagogique}.

  \subsection{Les fonctions circulaires} % (fold)
  \label{sec:the_circle_functions}

Bien que la mesure d'angle radian fournisse certains des arguments les plus convaincants pour la vraie constante du cercle, il vaut également la peine de comparer les vertus de $\pi$ et $\tau$ dans d'autres contextes. On commence en considérant les fonctions élémentaires importantes $\sin\theta$ et $\cos\theta$. Connus comme les «\,fonctions circulaires\,» parce qu'ils donnent les coordonnées d'un point sur le cercle unité (c'est-à-dire un cercle de rayon 1), le sinus et le cosinus sont les fonctions fondamentales de la trigonométrie (la figure~\ref{fig:circle_functions}).

\begin{figure}
\begin{center}
\image{images/figures/circle-functions.pdf}
\end{center}
\caption{Les fonctions circulaires sont des coordonnées sur le cercle unité.\label{fig:circle_functions}}
\end{figure}

On fait examiner les graphiques des fonctions circulaires pour mieux comprendre leur comportement\footnote{Ces graphiques ont été produits avec l'aide de \href{https://www.wolframalpha.com/}{Wolfram|Alpha.}}. On remarquera sur la figure~\ref{fig:sine_with_tau} et la figure~\ref{fig:cosine_with_tau} que les deux fonctions sont périodiques avec la période $T$. Comme le montre la figure~\ref{fig:sine_with_tau}, la fonction sinus $\sin\theta$ commence à zéro, atteint un maximum à un quart de période, passe par zéro à une demi-période, atteint un minimum aux trois quarts de période et revient à zéro après une période complète. Pendant ce temps, la fonction cosinus $\cos\theta$ commence à un maximum, a un minimum à une demi-période et passe par zéro à un quart et trois quarts de période (la figure~\ref{fig:cosine_with_tau}).

\begin{figure}
\begin{center}
\image{images/figures/sine-with-tau.pdf}
\end{center}
\caption{Points importants pour $\sin\theta$ en termes de période $T$.\label{fig:sine_with_tau}}
\end{figure}

\begin{figure}
\begin{center}
\image{images/figures/cosine-with-tau.pdf}
\end{center}
\caption{Points importants pour $\cos\theta$ en termes de période $T$.\label{fig:cosine_with_tau}}
\end{figure}

Bien sûr, puisque le sinus et le cosinus passent tous les deux par un cycle complet pendant un tour du cercle, on a $T = \tau$\,; c'est-à-dire que les fonctions circulaires ont des périodes égales à la constante du cercle. Par conséquent, les valeurs «\,spéciales\,»  de $\theta$ sont tout à fait naturelles~: un quart de période est $\tau/4$, une demi-période est $\tau/2$, etc. En fait, en faisant la figure~\ref{fig:sine_with_tau}, à un moment donné, je me suis retrouvé à me poser des questions sur la valeur numérique de $\theta$ pour le zéro de la fonction sinus. Puisque le zéro se produit après une demi-période, et puisque $\tau \approx 6{,}28$, un calcul mental rapide a conduit au résultat suivant~:
\[
  \theta_\mathrm{zéro} = \frac{\tau}{2} \approx 3{,}14.
\]
C'est vrai~: j'étais étonné de découvrir que \emph{j'avais déjà oublié que $\tau/2$ est parfois appelé «\,$\pi$\,»}. Peut-être que cela vient de vous arriver. Bienvenue dans ma vie.

  % subsection the_circle_functions (end)

% section radian_angle_measure (end)

   \subsection{Identité d'Euler} % (fold)
   \label{sec:euler_s_identity}

Je serais négligent dans ce manifeste de ne pas aborder \emph{l'identité d'Euler}, parfois appelée «\,la plus belle équation en mathématiques\,». Cette identité implique \emph{l'exponentiation complexe}, qui est profondément liée à la fois aux fonctions circulaires et à la géométrie du cercle lui-même.

Selon lequel itinéraire est choisi, l'équation suivante peut être prouvée comme théorème ou prise comme définition\,; de toute façon, c'est assez remarquable~:
\begin{equation}
\label{eq:eulers_formula}
e^{i\theta} = \cos\theta + i\sin\theta. \qquad\mbox{Formule d'Euler}
\end{equation}
Connue sous le nom de \emph{formule d'Euler} (d'après \href{https://en.wikipedia.org/wiki/Leonhard_Euler}{Leonhard Euler}), cette équation relie un exponentiel à argument imaginaire aux fonctions circulaires (sinus et cosinus) et à l'unité imaginaire~$i$. Bien que justifier la formule d'Euler dépasse le cadre de ce manifeste, sa provenance est au-dessus de tout soupçon et son importance est incontestable.

L'évaluation de l'équation~\eqref{eq:eulers_formula} à $\theta = \tau$ donne \emph{l'identité d'Euler}\,\footnote{Ici, je définis implicitement l'identité d'Euler comme \emph{l'exponentiel complexe de la constante du cercle}, plutôt que de la définir comme l'exponentiel complexe d'un nombre particulier. Si l'on choisit $\tau$ comme la constante du cercle, on obtient l'identité montrée. Comme on le verra momentanémant, ce n'est pas la forme traditionnelle de l'identité, qui implique bien sûr $\pi$\,; mais la version avec $\tau$ est la déclaration mathématiquement la plus significative de l'identité, donc je crois qu'elle mérite le nom.}~:
\begin{equation}
\label{eq:eulers_identity_tau}
e^{i\tau} = 1. \qquad\mbox{Identité d'Euler (version de $\tau$)}
\end{equation}
En mots, l'équation~\eqref{eq:eulers_identity_tau} fait l'observation fondamentale suivante~:

\begin{center}
\emph{L'exponentiel complexe de la constante du cercle est l'unité.}
\end{center}

Géométriquement, la multiplication par $e^{i\theta}$ correspond à la rotation d'un nombre complexe d'un angle $\theta$ dans le plan complexe, ce qui suggère une seconde interprétation de l'identité d'Euler~:

\begin{center}
\emph{Une rotation d'un tour égale 1.}
\end{center}

\noindent Puisque le nombre $1$ est l'\href{https://fr.wikipedia.org/wiki/%C3%89l%C3%A9ment_neutre}{identité multiplicative}, la signification géométrique de $e^{i\tau} = 1$ est que la rotation d'un point du plan complexe par un tour le ramène simplement à sa position d'origine.

Comme dans le cas de la mesure d'angle radian, on voit à quel point l'association est naturelle entre $\tau$ et un tour d'un cercle. En effet, l'identification de $\tau$ avec «\,un tour\,» fait que l'identité d'Euler ressemble presque à une tautologie\,\footnote{\href{https://bit.ly/32mB2CF}{Techniquement}, tous les théorèmes mathématiques sont des tautologies, mais ne soyons pas aussi pédantes.}.

    \subsubsection{Pas la plus belle équation} % (fold)
    \label{sec:not_the_most_beautiful_equation}

Bien sûr, la forme traditionnelle de l'identité d'Euler est écrite en termes de $\pi$ au lieu de $\tau$. Pour le dériver, on commence en évaluant la formule d'Euler à $\theta = \pi$, ce qui donne
\begin{equation}
\label{eq:eulers_identity_pi}
e^{i\pi} = -1. \qquad\mbox{Identité d'Euler (version de $\pi$)}
\end{equation}
\noindent Mais ce signe-ci moins est si moche que l'équation~\eqref{eq:eulers_identity_pi} est presque toujours réarrangée immédiatement, donnant la «\,belle\,» équation suivante~:
\begin{equation}
\label{eq:eulers_pi_rearranged}
e^{i\pi} + 1 = 0. \qquad\mbox{(réarrangé)}
\end{equation}
À ce stade, l'exposant fait généralement une déclaration grandiose sur la façon dont l'équation~\eqref{eq:eulers_pi_rearranged} relie $0$, $1$, $e$, $i$ et $\pi$, parfois appelés les «\,cinq nombres les plus importants en mathématiques\,».

Dans ce contexte, il est remarquable de voir combien de personnes se plaignent que l'équation~\eqref{eq:eulers_identity_tau} ne relie que \emph{quatre} de ces cinq. Bien~:
\begin{equation}
\label{eq:euler_tau_zero}
e^{i\tau} = 1 + 0.
\end{equation}
L'équation~\eqref{eq:euler_tau_zero}, \emph{sans} réarrangement, relie en fait les cinq nombres les plus importants en mathématiques~: $0$, $1$, $e$, $i$ et $\tau$\,\footnote{En effet, l'équation (6) peut s'écrire $e^{i\tau} = 1 + 0i$, ce qui rend la relation entre les cinq nombres encore plus explicite.}.

      \subsubsection{Identités eulériennes} % (fold)
      \label{sec:eulerian_identities}

Puisque l'on peut ajouter zéro n'importe où dans n'importe quelle équation, l'introduction de $0$ dans l'équation~\eqref{eq:euler_tau_zero} est un contrepoint quelque peu ironique à $e^{i\pi} + 1 = 0$\,; mais l'identité $e^{i\pi} = -1$ a un argument plus sérieux à faire valoir. On fait voir ce qui se passe quand on le réécrit en termes de $\tau$~:
\[
e^{i\tau /2} = -1.
\]
Géométriquement, cela signifie qu'une rotation d'un demi-tour équivaut à une multiplication par $-1$. Et, en effet, c'est le cas~: sous une rotation de $\tau/2$ radians, le nombre complexe $z = a + ib$ est mappé sur $-a - ib$, qui est en fait juste $-1\cdot z$.

Écrite en termes de $\tau$, on voit que la forme «\,originale\,» de l'identité d'Euler (l'équation~\eqref{eq:eulers_identity_pi}) a une signification géométrique transparente qui lui manque lorsqu'elle est écrite en termes de $\pi$. (Bien sûr, $e^{i\pi} = -1$ peut être interprété comme une rotation par $\pi$ radians, mais le réarrangement quasi universel pour former $e^{i\pi} + 1 = 0$ montre comment l'utilisation de $\pi$ distrait de la signification géométrique naturelle de l'identité.) Les identités en quart d'angle ont des interprétations géométriques similaires~: en évaluant l'équation~\eqref{eq:eulers_formula} à $\tau/4$, on obtient $e^{i\tau/4} = i$, qui dit qu'un quart de tour dans le plan complexe équivaut à une multiplication par $i$\,; de même, $e^{i\cdot(3\tau/4)} = -i$ dit que les trois quarts de tour équivalent à la multiplication par $-i$. Un résumé de ces résultats, que l'on appellera les identités eulériennes, figure dans le tableau~\ref{table:eulerian_identities}.

\begin{table}
\begin{center}
\begin{tabular}{cllr}
Angle de rotation & \multicolumn{3}{c}{Identité eulérienne} \\ \hline
$0$ & $e^{i\cdot0}$ & $ = $ & $1$ \smallskip \\
$\tau/4$ & $e^{i\tau/4}$ & $ = $ & $i$ \smallskip \\
$\tau/2$ & $e^{i\tau/2}$ & $ = $ & $-1$ \smallskip \\
$3\tau/4$ & $e^{i\cdot(3\tau/4)}$ & $ = $ & $-i$ \smallskip \\
$\tau$ & $e^{i\tau}$ & $ = $ & $1$
\end{tabular}
\end{center}
\caption{Identités eulériennes pour les rotations à demi, quart et complet.\label{table:eulerian_identities}}
\end{table}

On peut pousser cette analyse un peu plus loin en notant que, pour n'importe quel angle~$\theta$, $e^{i\theta}$ peut être interprété comme un point situé sur le cercle unité dans le plan complexe. Puisque le plan complexe identifie l'axe horizontal avec la partie réelle du nombre et l'axe vertical avec la partie imaginaire, la formule d'Euler indique que $e^{i\theta}$ correspond aux coordonnées $(\cos\theta,\sin\theta)$. Le remplacement de ceux-ci dans l'équation~\eqref{eq:eulers_formula} par les valeurs des angles «\,spéciaux\,» de la figure~\ref{fig:tau_angles} donne ensuite les points indiqués dans le tableau 2, et le traçage de ces points dans le plan complexe donne la figure~\ref{fig:tau_euler_circle}. Une comparaison de la figure~\ref{fig:tau_euler_circle} avec la figure~\ref{fig:tau_angles} dissipe rapidement tout doute quant au choix de la constante du cercle qui révèle le mieux la relation entre la formule d'Euler et la géométrie du cercle.

\begin{table}
\begin{center}
\begin{tabular}{lcc}
Forme polaire & Forme cartésienne & Coordonnées \\ \hline\hline
$e^{i\theta}$ & $\cos\theta + i\sin\theta$ & $(\cos\theta, \sin\theta)$ \\ \hline
$e^{i\cdot0}$ & $1$ & $(1, 0)$ \smallskip \\
$e^{i\tau/12}$ & $\frac{\sqrt{3}}{2} + \frac{1}{2}i$ & $(\frac{\sqrt{3}}{2}, \frac{1}{2})$ \smallskip \\
$e^{i\tau/8}$ & $\frac{1}{\sqrt{2}} +  \frac{1}{\sqrt{2}}i$ & $(\frac{1}{\sqrt{2}}, \frac{1}{\sqrt{2}})$ \smallskip \\
$e^{i\tau/6}$ & $\frac{1}{2} +\frac{\sqrt{3}}{2} i$ & $(\frac{1}{2}, \frac{\sqrt{3}}{2})$ \smallskip \\
$e^{i\tau/4}$ & $i$ & $(0, 1)$ \smallskip \\
$e^{i\tau/3}$ & $-\frac{1}{2} +\frac{\sqrt{3}}{2} i$ & $(-\frac{1}{2}, \frac{\sqrt{3}}{2})$ \smallskip \\
$e^{i\tau/2}$ & $-1$ & $(-1, 0)$ \smallskip \\
$e^{i\cdot(3\tau/4)}$ & $-i$ & $(0, -1)$ \smallskip \\
$e^{i\tau}$ & $1$ & $(1, 0)$
\end{tabular}
\end{center}
\caption{Exponentiels complexes des angles spéciaux de la figure~\ref{fig:tau_angles}.\label{table:complex_exponentials}}
\end{table}

\begin{figure}
\begin{center}
\image{images/figures/tau_euler_circle.pdf}
\end{center}
\caption{Exponentiels complexes de certains angles spéciaux, tracées dans le plan complexe.\label{fig:tau_euler_circle}}
\end{figure}

      % subsubsection eulerian_identities (end)

\section{La superficie circulaire~: le coup de grâce} % (fold)
\label{sec:circular_area}

Si quelqu'un est arrivé ici en tant que croyant $\pi$, il doit maintenant remettre en question sa foi. $\tau$ est tellement naturel, sa signification tellement transparente---n'y a-t-il pas d'exemple où $\pi$ brille dans toute sa splendeur rayonnante\,? Une mémoire remue---oui, il existe une telle formule---c'est la formule de la superficie du cercle\,! Voici~:
\[ A = \tfrac{1}{4} \pi D^2. \]
Non, attends. La formule de la superficie est toujours écrite en termes de \emph{rayon}, comme suit~:
\[ A = \pi r^2. \]
On voit ici $\pi$, sans fioritures, dans l'une des équations les plus importantes en mathématiques---une formule prouvée premièrement par \href{https://fr.wikipedia.org/wiki/Archimède}{Archimède} lui-même. L'ordre est rétabli\,! Et pourtant, le nom de cette section semble inquiétant\textellipsis\ Si cette équation est le couronnement de $\pi$, comment peut-elle aussi être le coup de grâce\,?


  \subsection{Formes quadratiques} % (fold)
  \label{sec:quadratic_forms}

On fait examiner ce parangon de $\pi$, $A = \pi r^2$. On remarque qu'il implique le diamètre---non, attends, le \emph{rayon}---élevé à la deuxième puissance. Cela en fait une \emph{forme quadratique} simple. De telles formes surviennent dans de nombreux contextes\,; je suis \href{https://thesis.library.caltech.edu/1940/}{physicien}, donc mes exemples préférés viennent du cursus de physique élémentaire. On fait maintenant en considérer plusieurs successivement.

    \subsubsection{Tomber dans un champ gravitationnel uniforme} % (fold)
    \label{sec:falling_in_a_uniform_gravitational_field}

\href{https://fr.wikipedia.org/wiki/Galil%C3%A9e_(savant)}{Galilée} a constaté que le vecteur vitesse d'un objet qui tombe dans un champ gravitationnel uniforme est proportionnel au temps tomber~:
\[ v \propto t. \]
La constante de proportionnalité est l'accélération de la pesanteur~$g$~:
\[ v = g t. \]
Puisque le vecteur vitesse est la dérivée de la position, on peut calculer la distance tombée par intégration\,\footnote{\href{https://bit.ly/32mB2CF}{Techniquement}, toutes les intégrales doivent être définies et la variable d'intégration doit être différente de la limite supérieure (par exemple, $\int_{0}^{t} gt'\, dt'$, que l'on prononce «\,l'intégrale de zéro à té de gé té prime dé té prime\,»). Ces \href{https://fr.wikipedia.org/wiki/Abus_de_notation}{abus mineurs de notation} sont courants en physique et dans d'autres contextes mathématiques moins formels comme celui que l'on considère ici.}~:
\[ y = \int v\,dt = \int_0^t gt\,dt = \textstyle{\frac{1}{2}} gt^2. \]


    \subsubsection{Énergie potentielle dans un ressort linéaire} % (fold)
    \label{sec:potential_energy_in_a_linear_spring}

\href{https://fr.wikipedia.org/wiki/Robert_Hooke}{Robert Hooke} a constaté que la force externe requise pour étirer un ressort est proportionnelle à la distance étirée~:
\[ F \propto x. \]
La constante de proportionnalité est la constante de rappel~$k$\,\footnote{On peut avoir vu ceci écrit comme $F=-kx$. Dans ce cas, $F$ fait référence à la force qu'exerce le \emph{ressort}. Selon la troisième loi de Newton, la force externe discutée ci-dessus est le \emph{négatif} de la force du ressort.}~:
\[ F = k x. \]
L'énergie potentielle du ressort est alors égale au travail fourni par la force externe~:
\[ U = \int F\,dx = \int_0^x kx\,dx = \textstyle{\frac{1}{2}} kx^2. \]

    \subsubsection{Énergie de mouvement} % (fold)
    \label{sec:energy_of_motion}

\href{https://fr.wikipedia.org/wiki/Isaac_Newton}{Isaac Newton} a constaté que la force sur un objet est proportionnelle à son accélération~:
\[ F \propto a. \]
La constante de proportionnalité est la masse $m$~:
\[ F = m a. \]
L'énergie de mouvement, ou l'\emph{énergie cinétique}, est égale au travail total fourni en accélérant la masse au vecteur vitesse~$v$~:
\[
\begin{split}
K = \int F\,dx = \int ma\,dx & = \int m\frac{dv}{dt}\,dx \\ & = \int m\frac{dx}{dt}\,dv \\ & = \int_0^v mv\,dv \\ & = \textstyle{\frac{1}{2}} mv^2.
\end{split}
\]

  \subsection{Un sentiment nerveux} % (fold)
  \label{sec:a_sense_of_foreboding}

Après avoir vu plusieurs exemples de formes quadratiques simples en physique, on a maintenant peut-être un sentiment nerveux en retournant à la géométrie du cercle. Ce sentiment est justifié.

\begin{figure}
\begin{center}
\image{images/figures/circular-area.pdf}
\end{center}
\caption{Se décomposer un cercle en anneaux.\label{fig:circular_area}}
\end{figure}


Comme le montre la figure~\ref{fig:circular_area}, on peut calculer la superficie d'un cercle en le décomposant en anneaux circulaires de longueur $C$ et de largeur $dr$, où la superficie de chaque anneau est $C\,dr$~:
\[ dA = C\,dr. \]
La circonférence d'un cercle est proportionnelle à son rayon~:
\[ C \propto r. \]
La constante de proportionnalité est $\tau$~:
\[ C = \tau\,r. \]
La superficie du cercle est alors l'intégrale sur tous les anneaux~:
\[ A = \int dA = \int_0^r C\,dr = \int_0^r \tau\,r\,dr = \textstyle{\frac{1}{2}} \tau\,r^2. \]

Si l'on était encore partisan de $\pi$ au début de cette section, sa tête a maintenant explosé. Car on voit que même dans ce cas, où $\pi$ brille soi-disant, en fait il manque un facteur 2. En effet, la démonstration originale d'Archimède montre non pas que la superficie d'un cercle est $\pi r^2$, mais qu'elle est égale à la superficie d'un triangle rectangle de base $C$ et de hauteur $r$. L'application de la formule pour la superficie triangulaire donne alors
\[
  A = \textstyle{\frac{1}{2}} bh = \textstyle{\frac{1}{2}}Cr = \textstyle{\frac{1}{2}}\tau\,r^2.
\]
Il est tout simplement impossible d'éviter ce facteur de demi (le tableau~\ref{table:quadratic_forms}).

\begin{table}
\begin{center}
\begin{tabular}{lcc}
Quantité & Symbole & Expression \\ \hline
Distance tombée & $y$ & $\textstyle{\frac{1}{2}}gt^2$ \smallskip \\
Énergie de ressort & $U$ & $\textstyle{\frac{1}{2}}kx^2$ \smallskip \\
Énergie cinétique & $K$ & $\textstyle{\frac{1}{2}}mv^2$ \smallskip \\
Superficie circulaire & $A$ & $\textstyle{\frac{1}{2}}\tau\,r^2$
\end{tabular}
\end{center}
\caption{Quelques formes quadratiques courantes.\label{table:quadratic_forms}}
\end{table}

    \subsubsection{Quod erat demonstrandum} % (fold)
    \label{sec:quod_erat_demonstrandum}

On a voulu dans ce manifeste montrer que $\tau$ est la vraie constante du cercle. Étant donné que la formule de la superficie circulaire était à peu près le dernier, le meilleur argument que $\pi$ avait pour lui, je vais ici être effronté et dire~: \href{https://fr.wikipedia.org/wiki/CQFD_(math%C3%A9matiques)}{QED}.

    % subsubsection quod_erat_demonstrandum (end)

% section circular_area (end)

\section{Conflit et résistance} % (fold)
\label{sec:conflict_and_resistance}

Malgré la démonstration définitive de la supériorité de $\tau$, nombreux sont néanmoins ceux qui s'y opposent, à la fois en notation et en nombre. Dans cette section, on répond aux préoccupations de ceux qui acceptent la valeur mais pas la lettre. On réfute ensuite certains des nombreux arguments opposés à $C/r$ lui-même, y compris le soi-disant «\,Pi Manifesto\,» («\,Manifeste de pi\,») qui défend la primauté de $\pi$. Dans ce contexte, on discutera du sujet assez avancé du volume d'une hypersphère (la section~\ref{sec:volume_of_a_hypersphere}), qui augmente et amplifie les arguments de la section~\ref{sec:circular_area} sur la superficie circulaire.

  \subsection{Un tour} % (fold)
  \label{sec:one_turn}

Le véritable épreuve de toute notation et l'utilisation\,; ayant vu $\tau$ utilisé tout au long de ce manifeste, on est peut-être déjà convaincu qu'il remplit bien son rôle. Mais pour une constante aussi fondamentale que $\tau$, ce serait bien d'avoir des raisons plus profondes pour le choix. Pourquoi pas $\alpha$, par exemple, ou $\omega$\,? Qu'est-ce qui est si génial avec $\tau$\,?

Il y a deux raisons principales d'utiliser $\tau$ pour la constante du cercle. La première est que $\tau$ ressemble visuellement à $\pi$~: après des siècles d'utilisation, l'association de $\pi$ à la constante du cercle est inévitable, et l'utilisation de $\tau$ se nourrit de cette association au lieu de la combattre. (En effet, la ligne horizontale dans chaque lettre suggère que l'on interprète les «\,jambes\,» comme des dénominateurs, de sorte que $\pi$ a deux jambes dans son dénominateur, tandis que $\tau$ n'en a qu'une. Vu de cette façon, la relation $\tau = 2\pi$ est parfaitement naturelle\,\footnote{Merci au lectur du \emph{Manifeste de tau}, Jim Porter, d'avoir souligné cette interprétation.}.)

La seconde raison est que $\tau$ correspond à un \emph{tour} d'un cercle, et l'on a peut-être remarqué que «\,$\tau$\,» et «\,tour\,» (en anglais, «\,turn\,») commencent tous les deux par un son «\,t\,». Telle était la motivation d'origine du choix de $\tau$, et ce n'est pas un hasard~: la racine du mot anglais «\,turn\,» est le mot grec τόρνος (tornos), qui signifie «\,tour\,» (la machine-outil\,; en anglais, «\,lathe\,»). L'utilisation d'une fonte mathématique pour la première lettre de τόρνος donne alors~: $\tau$.

Depuis le lancement original du \emph{Manifeste de tau}, j'ai appris que \href{http://www.harremoes.dk/Peter/Undervis/Turnpage/Turnpage1.pdf}{Peter Harremoës} a proposé de manière indépendante d'utiliser $\tau$ à l'auteur de «\,$\pi$ Is Wrong!\,», Bob Palais, en 2010\,; John Fisher a proposé $\tau$ dans un \href{https://groups.google.com/forum/#!msg/sci.math/c-DHmJHSA0A/sLCoOtHB1UAJ}{poste Usenet} en 2004\,; et Joseph Lindenberg a anticipé à la fois l'argument et le symbole plus de vingt ans auparavant\,\footnote{Lindenberg a inclus à la fois son manuscrit dactylographié original et un grand nombre d'autres arguments sur son site \href{https://sites.google.com/site/taubeforeitwascool/}{Tau Before It Was Cool} («\,Tau avant qu'il ne soit cool\,»).}\,! Le Dr Harremoës en particulier a souligné l'importance d'un point qui a été soulevé pour la première fois dans la section~\ref{sec:an_immodest_proposal}~: l'utilisation de $\tau$ donne un \emph{nom} à la constante du cercle. Puisque $\tau$ est une lettre grecque ordinaire, les personnes qui la rencontrent pour la première fois peuvent la prononcer immédiatement. De plus, contrairement à appeler la constante du cercle un «\,tour\,»,
$\tau$ fonctionne bien à la fois dans des contextes écrits et parlés. Par exemple, dire qu'un quart d'un cercle a une mesure d'angle radian «\,un quart de tour\,» sonne bien, mais «\,tour sur quatre radians\,» semble maladroit et «\,la superficie d'un cercle est de demi-tour $r$ au carré\,» sonne carrément bizarre. En utilisant $\tau$, on peut dire «\,tau sur quatre radians\,» et «\,la superficie d'un cercle est de demi tau $r$ au carré\,».

    \subsubsection{Notation ambiguë} % (fold)
    \label{sec:ambiguous_notation}


Bien sûr, avec toute nouvelle notation, il y a un risque de conflit avec l'utilisation actuelle. Comme le noté dans la section~\ref{sec:an_immodest_proposal}, «\,$\pi$ is Wrong!\,» évite ce problème en introduisant un nouveau symbole (la figure~\ref{fig:palais_tau}). Il y a des précédents à cela\,; par exemple, aux premiers jours de la mécanique quantique, \href{https://fr.wikipedia.org/wiki/Max_Planck}{Max Planck} a introduit la constante~$h$, qui relie l'énergie d'une particule lumineuse à sa fréquence (via $E = h\nu$), mais les physiciens se sont vite rendus compte qu'il était souvent plus pratique d'utiliser $\hbar$ (que l'on prononce «\,h barre\,»)---où $\hbar$ est juste $h$ divisé par\textellipsis{} euh\textellipsis{} $2\pi$---et cet usage est depuis devenu standard.

Mais il est difficile pour un nouveau symbole d'être accepté~: il faut lui donner un nom, ce nom doit être popularisé, et le symbole lui-même doit être ajouté aux systèmes de traitement de texte et de composition. De plus, la promulgation d'un nouveau symbole pour $2\pi$ nécessiterait la coopération de la communauté mathématique académique, qui sur le sujet de $\pi$ vs $\tau$ a été jusqu'à présent apathique au mieux et hostile au pire. L'utilisation d'un symbole existant permet de contourner l'établissement mathématique\,\footnote{Peut-être qu'un jour, les mathématiciens académiques arriveront à un consensus sur un symbole différent pour le nombre $2\pi$\,; si cela se produit, je me réserve le droit de soutenir leur notation proposée. Mais ils ont eu plus de 300 ans pour résoudre ce problème de $\pi$, donc je ne suis pas optimiste qu'ils le feront bientôt.}.

Plutôt que de préconiser un nouveau symbole, \emph{Le Manifeste de tau} opte pour l'utilisation d'une lettre grecque existante. Par conséquent, puisque $\tau$ est déjà utilisé dans certains contextes actuels, il faut résoudre les conflits avec la pratique existante. Heureusement, il existe étonnamment peu d'utilisations courantes. De plus, alors que $\tau$ est utilisé pour certains variables \emph{spécifiques}---par exemple, la \emph{contrainte de cisaillement} en génie mécanique, le \emph{moment d'une force} en mécanique rotationnelle et \emph{temps propre} en relativité restreinte et générale---il n'y a pas d'utilisation conflictuelle \emph{universelle}\,\footnote{La seule exception possible à cela est le \emph{nombre d'or}, qui est souvent désigné par $\tau$ en Europe. Mais non seulement existe-t-il une alternative commune à cette notation, à savoir la lettre grecque $\varphi$, mais cette utilisation montre qu'il existe un précédent pour utiliser $\tau$ pour désigner une constante mathématique fondamentale.}. Dans ce cas, on peut soit tolérer l'ambiguïté, soit contourner les quelques conflits actuels en modifiant sélectivement la notation, comme en utilisant $N$ pour le moment d'une force\,\footnote{Cette alternative pour le moment d'une force est déjà utilisée\,; voir, par exemple, \emph{Introduction to Electrodynamics} (\emph{Introduction à l'électrodynamique}) de David Griffiths, page 162.} ou $\tau_p$ pour temps propre.

Malgré ces arguments, les conflits potentiels d'utilisation se sont révélés être la plus grande source de résistance à $\tau$. Certains correspondants ont même catégoriquement nié que $\tau$ (ou, vraisemblablement, tout autre symbole actuellement utilisé) pourrait éventuellement surmonter ces problèmes. Mais les scientifiques et les ingénieurs ont une grande tolérance à l'ambiguïté de notation, et affirmer que $\tau$-la-constante-du-cercle ne peut pas coexister avec d'autres utilisations ignore des preuves considérables du contraire.

Un exemple d'ambiguïté facilement toléré se produit en mécanique quantique, où l'on rencontre la formule suivante pour le \emph{rayon de Bohr}, qui (grosso modo) est la «\,taille\,» d'un atome d'hydrogène dans son niveau d'énergie le plus bas (l'\emph{état fondamental})~:
\[
a_0 = \frac{\hbar^2}{m e^2},
\]
où $m$ est la masse d'un électron et $e$ est sa charge. Pendant ce temps, l'état fondamental lui-même est décrit par une quantité connue sous le nom de \href{https://fr.wikipedia.org/wiki/Fonction_d%27onde}{\emph{fonction d'onde}}, qui diminue de façon exponentielle avec un rayon sur une échelle de longueur définie par le rayon de Bohr~:
\begin{equation}
\label{eq:hydrogen}
\psi(r) = N\,e^{-r/a_0},
\end{equation}
où $N$ est une constante de normalisation.

A-t-on encore remarqué le problème\,? Probablement pas, ce qui est juste le point. Le «\,problème\,» est que le $e$ dans le rayon de Bohr et le $e$ dans la fonction d'onde \emph{ne sont pas les mêmes $e$}---le premier est la charge sur un électron, tandis que le second est le nombre naturel (la base des logarithmes naturels). En fait, si l'on développe le facteur de $a_0$ dans l'argument de l'exposant dans l'équation~\eqref{eq:hydrogen}, on obtient
\[
\psi(r) = N\,e^{-m e^2 r/\hbar^2},
\]
qui a un $e$ augmenté à la puissance de quelque chose qui contient $e$. C'est encore pire qu'il n'y paraît, car $N$ lui-même contient aussi $e$~:
\[
\psi(r) = \sqrt{\frac{1}{\pi a_0^3}}\,e^{-r/a_0} =
\frac{m^{3/2} e^3}{\pi^{1/2} \hbar^3}\,e^{-m e^2 r/\hbar^2}.
\]

Je ne doute pas que s'il n'existait pas déjà une notation distincte pour le nombre naturel, quiconque proposerait la lettre $e$ se verrait dire que cela était impossible en raison des conflits avec d'autres utilisations. Et pourtant, dans la pratique, personne n'a jamais de problème avec l'utilisation de $e$ dans les deux contextes ci-dessus.
Il existe de nombreux autres exemples, y compris des situations où même $\pi$ est utilisé pour deux choses differentes\,\footnote{Voir, par exemple, \emph{An Introduction to Quantum Field Theory} (\emph{Une Introduction à la théorie quantique des champs}) par Peskin et Schroeder, où $\pi$ est utilisé pour désigner à la fois la constante du cercle et une «\,quantité conjuguée de mouvement\,» sur la même page (page~282).}. Il est difficile de voir en quoi l'utilisation de $\tau$ pour multiples quantités est différente.

Soit dit en passant, les pédants de $\pi$ (et il y en a eu beaucoup) pourraient remarquer que la fonction d'onde de l'état fondamental de l'hydrogène a un facteur $\pi$~:
\[
\psi(r) = \sqrt{\frac{1}{\pi a_0^3}}\,e^{-r/a_0}.
\]
À première vue, cela semble plus naturel que la version avec $\tau$~:
\[
\psi(r) = \sqrt{\frac{2}{\tau a_0^3}}\,e^{-r/a_0}.
\]
Comme d'habitude, les apparences sont trompeuses~: la valeur de $N$ vient du produit
\[
\frac{1}{\sqrt{2\pi}} \frac{1}{\sqrt{2}} \frac{2}{a_0^{3/2}},
\]
ce qui montre que la constante du cercle entre dans le calcul par $1/\sqrt{2\pi}$, c'est-à-dire $1/\sqrt{\tau}$. Comme pour la formule de la superficie circulaire, la neutralisation qui laisse un $\pi$ dépouillé est un coïncidence.

    % subsubsection ambiguous_notation (end)

  \subsection{Le Manifeste de pi} % (fold)
  \label{sec:the_pi_manifesto_a_rebuttal}

Bien que la plupart des objections à $\tau$ proviennent de la correspondance électronique dispersée et de divers commentaires sur le Web, il existe aussi une résistance organisée. En particulier, depuis la publication du \emph{Manifeste de tau} en juin 2010, un «\,\href{http://thepimanifesto.com/}{Manifeste de pi}\,» est apparu pour plaider en faveur de la constante traditionnelle du cercle. Cette section et les deux suivantes contiennent une réfutation de ses arguments. Par nécessité, ce traitement est plus laconique et plus avancé que le reste du manifeste, mais même une lecture superficielle de ce qui suit donnera une impression de la faiblesse du cas du Manifeste de pi.

Alors que l'on peut certainement considérer l'apparition du Manifeste de pi comme un bon signe d'intérêt continu pour ce sujet, il fait plusieurs fausses affirmations. Par exemple, il dit que le facteur de $2\pi$ dans la loi gaussienne (normale) est une coïncidence, et qu'il peut plus naturellement s'écrire comme
\[
\frac{1}{\sqrt\pi(\sqrt 2\sigma)}e^{\frac{-x^2}{(\sqrt 2\sigma)^2}}.
\]
C'est faux~: le facteur de $2\pi$ vient de l'élévation au carré de la loi gaussienne non normalisée et du passage aux coordonnées polaires, ce que conduit à un facteur 1 de 
l'intégrale radiale et $2\pi$ de l'intégrale angulaire. Comme dans le cas de la superficie circulaire, le facteur
de $\pi$ vient de $1/2\times 2\pi$, pas de $\pi$ seul.

Une affirmation connexe est que la \href{https://fr.wikipedia.org/wiki/Fonction_gamma}{fonction gamma} évaluée à $1/2$ est plus naturelle en termes de $\pi$~:
\[
\Gamma(\textstyle{\frac{1}{2}}) = \sqrt{\pi},
\]
où
\begin{equation}
\label{eq:gamma}
\Gamma(p) = \int_{0}^{\infty} x^{p-1} e^{-x}\,dx.
\end{equation}
Mais $\Gamma(\frac{1}{2})$ se réduit à la même intégrale gaussienne que dans la loi normale (lors du réglage de $u =
x^{1/2}$), donc $\pi$ dans ce cas est aussi vraiment $1/2\times 2\pi$. En effet, dans de nombreux cas cités dans Le Manifeste de pi,
la constante du cercle entre par une intégrale sous tous les angles, c'est-à-dire
comme $\theta$ varie de $0$ à $\tau$.

Le Manifeste de pi examine aussi certaines formules pour des polygones
réguliers à $n$ côtés (ou «\,$n$-gones\,»). Par exemple, il note que la somme des angles internes d'un $n$-gone est donnée par
\[
\sum_{i=1}^n \theta_i=(n-2)\pi.
\]
Ce problème a été traité dans «\,$\pi$ Is Wrong!\,», qui note ce qui suit~: «\,La somme des angles intérieurs [d'un
triangle] est $\pi$, c'est vrai. Mais la somme des angles \emph{extérieurs} de \emph{tout}
polygone, dont la somme des angles intérieurs peut facilement être
dérivée, et qui se généralise à l'intégrale de la courbure d'une
simple courbe fermée, est de $2\pi$.\,» De plus, Le Manifeste de pi offre la formule de la superficie d'un $n$-gone avec un rayon unité (la distance du centre au sommet),
\[ A=n\sin\frac{\pi}{n}\cos\frac{\pi}{n}, \]
et l'appelle «\,clairement\textellipsis{} une autre victoire pour $\pi$\,». Mais l'utilisation de l'identité à double angle $\sin\theta\cos\theta = \frac{1}{2}\sin2\theta$ montre que l'on peut écrire ceci comme
\[ A = n/2\, \sin\frac{2\pi}{n}, \]
ce qui est juste
\begin{equation}
\label{eq:area_polygon}
A = \frac{1}{2} n\, \sin\frac{\tau}{n}.
\end{equation}
En d'autres termes, la superficie d'un $n$-gone a un facteur
naturel de $1/2$. En fait, prendre la limite de l'équation~\eqref{eq:area_polygon} comme $n\rightarrow \infty$ (et appliquer la \href{https://fr.wikipedia.org/wiki/R%C3%A8gle_de_L%27H%C3%B4pital}{règle de L'Hôpital}) donne la superficie d'un polygone régulier unité avec une infinité de côtés, c'est-à-dire un cercle unité~:
\begin{equation}
\label{eq:lhopital}
\begin{split}
A & = \lim_{n\rightarrow\infty} \frac{1}{2} n\, \sin\frac{\tau}{n} \\
  & = \frac{1}{2} \lim_{n\rightarrow\infty} \frac{\sin\frac{\tau}{n}}{1/n} \\
  & = \tfrac{1}{2}\tau.
\end{split}
\end{equation}

Dans ce contexte, il convient de noter que Le Manifeste de pi fait beaucoup de bruit sur le fait que $\pi$ est la superficie d'un disque unité, de sorte que (par exemple) la superficie d'un quart de cercle (unité) est $\pi/4$. Ceci, prétend-on, est tout aussi bon pour $\pi$ que la mesure d'angle radian pour $\tau$. Malheureusement pour cet argument, comme noté dans la section~\ref{sec:circular_area} et comme on le voit à nouveau dans l'équation~\eqref{eq:lhopital}, le facteur $1/2$ apparaît naturellement dans le contexte de la superficie circulaire. En effet, la formule de la superficie d'un secteur circulaire sous-tendu par l'angle $\theta$ est
\[
\tfrac{1}{2}\theta\, r^2,
\]
donc il n'y a aucun moyen d'éviter le facteur $1/2$ en général. (On voit donc que $A =
\frac{1}{2}\tau r^2$ est simplement le cas particulier $\theta = \tau$.)

En bref, la différence entre la mesure d'angle et la superficie n'est pas
arbitraire. \linebreak Il n'y a pas de facteur naturel de $1/2$ dans le cas
de la mesure d'angle. En revanche, dans le cas de la superficie, la facteur $1/2$ résulte de l'intégrale d'une fonction linéaire en association avec une forme quadratique simple. En fait, le cas de $\pi$ est encore pire qu'il n'y paraît, comme le montre la section suivante.

  % subsection the_pi_manifesto_a_rebuttal (end)

\section{Aller au fond de pi et tau} % (fold)
\label{sec:getting_to_the_bottom_of_pi}

Je continue d'être impressionné par la richesse de ce sujet, et ma compréhension de $\pi$ et $\tau$ continue d'évoluer. Le jour de demi-tau 2012, je pensais avoir identifié \emph{exactement} ce qui n'allait pas avec $\pi$. Mon argument reposait sur une analyse de la superficie et du volume d'une sphère à $n$ dimensions, qui (comme le montre ci-dessous) montre clairement que $\pi$ n'a pas de signification géométrique fondamentale. Mon analyse était cepandant incomplète---un fait porté à mon attention dans un message remarquable du lecture du \emph{Manifeste de tau}, Jeff Cornell. En conséquence, cette section est une tentative non seulement de discréditer définitivement $\pi$, mais aussi d'articuler la vérité sur $\tau$, une vérité plus profonde et plus subtile que je ne l'avais imaginé.

\emph{Note}~: Cette section est plus avancée que le reste du manifeste et peut être ignorée sans perte de continuité. Si l'on la trouve déroutant, je recommande de passer directement à la conclusion de la section~\ref{sec:conclusion}.

  \subsection{Surface area and volume of a hypersphere} % (fold)
  \label{sec:volume_of_a_hypersphere}

We start our investigations with the generalization of a circle to arbitrary dimensions.\footnote{This discussion is based on an excellent comment by John Kodegadulo at spikedmath.com.} This object, called a \emph{hypersphere} or an \emph{$n$-sphere}, can be defined as follows.\footnote{Geometers and topologists use \href{https://mathworld.wolfram.com/Hypersphere.html}{incompatible definitions of hyperspheres}; this discussion uses the geometers' definitions.} (For convenience, we assume that these spheres are centered on the origin.) A $0$-sphere is the empty set, and we define its ``interior'' to be a point.\footnote{This makes sense, because a point has no boundary, i.e., the boundary of a point is the empty set.} A $1$-sphere is the set of all points satisfying
\[
x^2 = r^2,
\]
which consists of the two points $\pm r$. Its interior, which satisfies
\[
x^2 \leq r^2,
\]
is the line segment from $-r$ to $r$. A $2$-sphere is a circle, which is the set of all points satisfying
\[
x^2 + y^2 = r^2.
\]
Its interior, which satisfies,
\[
x^2 + y^2 \leq r^2,
\]
is a disk. Similarly, a $3$-sphere satisfies
\[
x^2 + y^2 + z^2 = r^2,
\]
and its interior is a ball. The generalization to arbitrary~$n$, although difficult to visualize for $n > 3$, is straightforward: an $n$-sphere is the set of all points satisfying
\[
\sum_{i=1}^{n} x_i^2 = r^2.
\]

The Pi Manifesto (discussed in Section~\ref{sec:the_pi_manifesto_a_rebuttal}) includes a formula for the volume of a unit $n$-sphere as an argument in favor of $\pi$:
\begin{equation}
\label{eq:unit_n_sphere_pi}
\frac{\sqrt{\pi}^{n} }{\Gamma(1 + \frac{n}{2})},
\end{equation}
where the Gamma function is given by Eq.~\eqref{eq:gamma}. Eq.~\eqref{eq:unit_n_sphere_pi} is a special case of the formula for general radius, which is also typically written in terms of $\pi$:
\begin{equation}
\label{eq:n_sphere_pi}
V_n(r) = \frac{\pi^{n/2} r^n}{\Gamma(1 + \frac{n}{2})}.
\end{equation}
Because $V_n(r) = \int S_n(r)\,dr$, we have $S_n(r) = dV_n(r)/dr$, which means that the surface area can be written as follows:
\begin{equation}
\label{eq:n_sphere_pi_r}
S_n(r) = \frac{n \pi^{n/2} r^{n-1}}{\Gamma(1 + \frac{n}{2})}.
\end{equation}

Rather than simply take these formulas at face value, let's see if we can untangle them to shed more light on the question of $\pi$ vs. $\tau$. We begin our analysis by noting that the apparent simplicity of the above formulas is an illusion: although the Gamma function is notationally simple, in fact it is an integral over a semi-infinite domain, which is not a simple idea at all. Fortunately, the Gamma function can be simplified in certain special cases. For example, when $n$ is an integer, it is easy to show (using integration by parts) that
\[
\Gamma(n) = (n-1)(n-2)\ldots 2\cdot 1 = (n-1)!
\]
Seen this way, $\Gamma$ can be interpreted as a generalization of the factorial function to real-valued arguments.\footnote{Indeed, the generalization to complex-valued arguments is straightforward: just replace real $x$ with complex $z$ in Eq.~\eqref{eq:gamma}.}

In the $n$-dimensional surface area and volume formulas, the argument of $\Gamma$ is not necessarily an integer, but rather is $\left(1 + \frac{n}{2}\right)$, which is an integer when $n$ is even and is a \emph{half}-integer when $n$ is odd. Taking this into account gives the following expression, which is taken from a standard reference, \href{https://mathworld.wolfram.com/Hypersphere.html}{Wolfram MathWorld}, and as usual is written in terms of~$\pi$:
\begin{equation}
\label{eq:surface_area_mathworld}
S_n(r) = \begin{cases}
\displaystyle \frac{2\pi^{n/2}\,r^{n-1}}{(\frac{1}{2}n - 1)!} & n \text{ even}; \\ \\
 \displaystyle \frac{2^{(n+1)/2}\pi^{(n-1)/2}\,r^{n-1}}{(n-2)!!} & n \text{ odd}.
\end{cases}
\end{equation}

Integrating with respect to $r$ then gives
\begin{equation}
\label{eq:volume_mathworld}
V_n(r) = \begin{cases}
\displaystyle \frac{\pi^{n/2}\,r^n}{(\frac{n}{2})!} & n \text{ even}; \\ \\
\displaystyle \frac{2^{(n+1)/2}\pi^{(n-1)/2}\,r^n}{n!!} & n \text{ odd}.
\end{cases}
\end{equation}

Let's examine Eq.~\eqref{eq:volume_mathworld} in more detail. Notice first that MathWorld uses the \emph{double factorial function}~$n!!$---but, strangely, it uses it only in the \emph{odd} case. (This is a hint of things to come.) The double factorial function, although rarely encountered in mathematics, is elementary: it's like the normal factorial function, but involves subtracting $2$ at a time instead of $1$, so that, e.g., $5!! = 5 \cdot 3 \cdot 1$ and $6!! = 6 \cdot 4 \cdot 2$. In general, we have
\begin{equation}
\label{eq:double_factorial}
n!! = \begin{cases}
n(n-2)\ldots6\cdot4\cdot2 & n \text{ even}; \\ \\
n(n-2)\ldots5\cdot3\cdot1 & n \text{ odd}.
\end{cases}
\end{equation}
(By definition, $0!! = 1!! = 1$.) Note that Eq.~\eqref{eq:double_factorial} naturally divides into even and odd cases, making MathWorld's decision to use it only in the odd case still more mysterious.

To solve this mystery, we'll start by taking a closer look at the formula for odd~$n$ in Eq.~\eqref{eq:volume_mathworld}:
\[ \frac{2^{(n+1)/2}\pi^{(n-1)/2}\,r^n}{n!!} \]
Upon examining the expression
\[ 2^{(n+1)/2}\pi^{(n-1)/2}, \]
we notice that it can be rewritten as
\[ 2(2\pi)^{(n-1)/2}, \]
and here we recognize our old friend~$2\pi$.

Now let's look at the even case in Eq.~\eqref{eq:volume_mathworld}. We noted above how strange it is to use the ordinary factorial in the even case but the double factorial in the odd case. Indeed, because the double factorial is already defined piecewise, if we unified the formulas by using $n!!$ in both cases we could pull it out as a common factor:
\[
V_n(r) = \frac{1}{n!!}\times \begin{cases}
\ldots & n \text{ even}; \\ \\
 \ldots & n \text{ odd}.
 \end{cases}
\]
So, is there any connection between the factorial and the double factorial? Yes---when $n$ is even, the two are related by the following identity:
\[ \left(\frac{n}{2}\right)! = \frac{n!!}{2^{n/2}}. \]
(This is easy to verify using \href{https://en.wikipedia.org/wiki/Mathematical_induction}{mathematical induction}.) Substituting this into the volume formula for even $n$ then yields
\[ \frac{2^{n/2}\pi^{n/2}\,r^n}{n!!}, \]
which bears a striking resemblance to
\[ \frac{(2\pi)^{n/2}\,r^n}{n!!}, \]
and again we find a factor of $2\pi$.

Putting these results together, we see that Eq.~\eqref{eq:volume_mathworld} can be rewritten as
\begin{equation}
\label{eq:volume_2pi}
V_n(r) = \begin{cases}
 \displaystyle \frac{(2\pi)^{n/2}\,r^n}{n!!} & n \text{ even}; \\ \\
 \displaystyle \frac{2(2\pi)^{(n-1)/2}\,r^n}{n!!} & n \text{ odd}
 \end{cases}
\end{equation}
and Eq.~\eqref{eq:surface_area_mathworld} can be rewritten as
\begin{equation}
\label{eq:surface_area_2pi}
S_n(r) = \begin{cases}
\displaystyle \frac{(2\pi)^{n/2}\,r^{n-1}}{(n-2)!!} & n \text{ even}; \\ \\
\displaystyle \frac{2(2\pi)^{(n-1)/2}\,r^{n-1}}{(n-2)!!} & n \text{ odd}.
\end{cases}
\end{equation}

Making the substitution $\tau=2\pi$ in Eq.~\eqref{eq:surface_area_2pi} then yields
\[
S_n(r) = \begin{cases}
\displaystyle \frac{\tau^{n/2}\,r^{n-1}}{(n-2)!!} & n \text{ even}; \\ \\
\displaystyle \frac{2\tau^{(n-1)/2}\,r^{n-1}}{(n-2)!!} & n \text{ odd}.
\end{cases} \]
To unify the formulas further, we can use the \emph{floor function} $\lfloor x \rfloor$, which is simply the largest integer less than or equal to $x$ (equivalent to chopping off the fractional part, so that, e.g., $\lfloor 3.7 \rfloor = \lfloor 3.2 \rfloor = 3$). This gives
\[ S_n(r) = \begin{cases}
 \displaystyle \frac{\tau^{\left\lfloor \frac{n}{2} \right\rfloor}\,r^{n-1}}{(n-2)!!} & n \text{ even}; \\ \\
 \displaystyle \frac{2\tau^{\left\lfloor \frac{n}{2} \right\rfloor}\,r^{n-1}}{(n-2)!!} & n \text{ odd},
 \end{cases} \]
which allows us to write the formula as follows:
\begin{equation}
\label{eq:surface_area_tau}
S_n(r) = \frac{\tau^{\left\lfloor \frac{n}{2} \right\rfloor}\,r^{n-1}}{(n-2)!!}\times \begin{cases}
1 & n \text{ even}; \\ \\
2 & n \text{ odd}.
\end{cases}
\end{equation}
Integrating Eq.~\eqref{eq:surface_area_tau} with respect to $r$ then yields
\begin{equation}
\label{eq:volume_tau}
V_n(r) = \frac{\tau^{\left\lfloor \frac{n}{2} \right\rfloor}\,r^n}{n!!}\times \begin{cases}
1 & n \text{ even}; \\ \\
2 & n \text{ odd}.
\end{cases}
\end{equation}

\subsubsection{Lambda} % (fold)
\label{sec:lambda}

The formulas in Eq.~\eqref{eq:surface_area_tau} and Eq.~\eqref{eq:volume_tau} represent a major improvement over the original formulations (Eq.~\eqref{eq:surface_area_mathworld} and Eq.~\eqref{eq:volume_mathworld}) in terms of $\pi$. But in fact an additional simplification is possible, using the measure of a \emph{right angle}:\footnote{This change of notation and general analysis was suggested by Jeff Cornell.}
\begin{equation}
\label{eq:lambda}
\lambda = \frac{\tau}{4}.
\end{equation}
As we'll see in Section~\ref{sec:three_families_of_constants}, Eq.~\eqref{eq:lambda} can be more naturally rewritten in terms of the symmetries of the circle:
\begin{equation}
\label{eq:tau_lambda}
\tau = 2^2 \lambda,
\end{equation}
where the factor of $2^2$ comes from the $2^2$ congruent circular arcs (one in each quadrant) in two-dimensional space.

The biggest advantage of $\lambda$ is that it completely unifies the even and odd cases in Eq.~\eqref{eq:surface_area_tau} and Eq.~\eqref{eq:volume_tau}, each of which has a factor of $\tau^{\left\lfloor \frac{n}{2} \right\rfloor}$. Making the substitution in Eq.~\eqref{eq:tau_lambda} then gives
\[
\begin{split}
\tau^{\left\lfloor \frac{n}{2} \right\rfloor} = (2^2\lambda)^{\left\lfloor \frac{n}{2} \right\rfloor} & = 2^{2\left\lfloor \frac{n}{2} \right\rfloor} \lambda^{\left\lfloor \frac{n}{2} \right\rfloor} \\
& = \lambda^{\left\lfloor \frac{n}{2} \right\rfloor}\times
\begin{cases}
 2^n & n \text{ even}; \\ \\
 2^{n-1} & n \text{ odd}.
 \end{cases}
 \end{split}
\]
This means that we can rewrite the product
\[
\tau^{\left\lfloor \frac{n}{2} \right\rfloor}\times \begin{cases}
1 & n \text{ even}; \\ \\
2 & n \text{ odd}.
\end{cases}
\]
as
\begin{equation}
\label{eq:prefactor}
\begin{split}
\lambda^{\left\lfloor \frac{n}{2} \right\rfloor} \times
\begin{cases}
 2^n & n \text{ even}; \\ \\
 2^{n-1} & n \text{ odd}.
 \end{cases}
 & \times
\begin{cases}
 1 & n \text{ even}; \\ \\
 2 & n \text{ odd}.
 \end{cases}
\\ & = 2^n\,\lambda^{\left\lfloor \frac{n}{2} \right\rfloor},
\end{split}
\end{equation}
which eliminates the explicit dependence on \href{https://en.wikipedia.org/wiki/Parity_(mathematics)}{parity}. Applying Eq.~\eqref{eq:prefactor} to \linebreak Eq.~\eqref{eq:surface_area_tau} and Eq.~\eqref{eq:volume_tau} then gives
\begin{equation}
\label{eq:surface_area_lambda}
S_n(r) = \frac{2^n\,\lambda^{\left\lfloor \frac{n}{2} \right\rfloor}\,r^{n-1}}{(n-2)!!}
\end{equation}
and
\begin{equation}
\label{eq:volume_lambda}
V_n(r) = \frac{2^n\,\lambda^{\left\lfloor \frac{n}{2} \right\rfloor}\,r^n}{n!!}.
\end{equation}

The simplification in Eq.~\eqref{eq:surface_area_lambda} and Eq.~\eqref{eq:volume_lambda} appears to come at the cost of a factor of $2^n$, but even this has a clear geometric meaning: a sphere in $n$ dimensions divides naturally into $2^n$ congruent pieces, corresponding to the $2^n$ families of solutions to $\sum_{i=1}^{n} x_i^2 = r^2$ (one for each choice of $\pm x_i$). In two dimensions, these are the circular arcs in each of the four quadrants; in three dimensions, they are the sectors of the sphere in each octant; and so on in higher dimensions. In other words, we can exploit the symmetry of the sphere by calculating the surface area or volume of \emph{one} piece---typically the \emph{principal part} where $x_i > 0$ for every $i$---and then find the full value by multiplying by~$2^n$.

To my knowledge, Eq.~\eqref{eq:surface_area_lambda} and Eq.~\eqref{eq:volume_lambda} are the simplest possible formulations of the spherical surface area and volume formulas (and indeed are the only forms I have ever consistently been able to memorize). Consider the volume formula in particular: unlike the \emph{faux} simplicity of Eq.~\eqref{eq:n_sphere_pi}, Eq.~\eqref{eq:volume_lambda} involves no fancy integrals---just the slightly exotic but nevertheless elementary floor and double-factorial functions. The volume of a unit $n$-sphere is just the volume of each symmetric piece, $\lambda^{\left\lfloor \frac{n}{2} \right\rfloor}/n!!$, multiplied by the number of pieces, $2^n$.

% subsubsection lambda (end)

\subsubsection{Recurrences} % (fold)
\label{sec:recurrences}

We've now seen, via Eq.~\eqref{eq:surface_area_lambda} and Eq.~\eqref{eq:volume_lambda}, that the surface area and volume formulas are simplest in terms of the right angle~$\lambda$. Nevertheless, we're still not done with $\tau$.

As seen in Eq.~\eqref{eq:volume_lambda}, the volume formula divides naturally into two families, corresponding to even- and odd-dimensional spaces, respectively. This means that the four-dimensional volume, $V_4$, is related simply to $V_2$ but not to $V_3$, while $V_3$ is related to $V_1$ but not to $V_2$. How exactly are they related?

We can find the answer by deriving the \emph{recurrence relations} between dimensions.\footnote{The article ``\href{http://www2.math.uconn.edu/~mariano/research/MathClubsp14\%20.pdf}{The volume of the unit ball in \emph{n} dimensions}'' by Phanuel A. Mariano contains an alternate derivation of these important recurrences.} In particular, let's divide the volume of an $n$-dimensional sphere by the volume for an $(n-2)$-dimensional sphere:
\begin{equation}
\label{eq:volume_recurrence}
\begin{split}
\frac{V_n(r)}{V_{n-2}(r)} & =
\frac{2^n}{2^{n-2}}
\frac{\lambda^{\left\lfloor \frac{n}{2} \right\rfloor}}{\lambda^{\left\lfloor \frac{n-2}{2} \right\rfloor}}
\frac{(n-2)!!}{n!!}
\frac{r^{n}}{r^{n-2}}
\\ & = \frac{2^2\lambda}{n}\,r^2.
\end{split}
\end{equation}
We see from Eq.~\eqref{eq:volume_recurrence} that we can obtain the volume of an $n$-sphere simply by multiplying the formula for an $(n-2)$-sphere by $r^2$ (a factor required by dimensional analysis), dividing by $n$, and multiplying by the ``recurrence constant'' $2^2\lambda$.

Similarly, for the surface area we have
\begin{equation}
\label{eq:surface_area_recurrence}
\begin{split}
\frac{S_n(r)}{S_{n-2}(r)} & =
\frac{2^n}{2^{n-2}}
\frac{\lambda^{\left\lfloor \frac{n}{2} \right\rfloor}}{\lambda^{\left\lfloor \frac{n-2}{2} \right\rfloor}}
\frac{(n-2-2)!!}{(n-2)!!}
\frac{r^{n}}{r^{n-2}}
\\ & = \frac{2^2\lambda}{n-2}\,r^2,
\end{split}
\end{equation}
with the same recurrence constant $2^2\lambda$.

Thus, in both Eq.~\eqref{eq:volume_recurrence} and Eq.~\eqref{eq:surface_area_recurrence}, the constant relating the different dimensions is not $\lambda$ itself but rather the combination $2^2\lambda$. Comparing with Eq.~\eqref{eq:tau_lambda}, we see this is none other than $\tau$! Indeed, an \href{https://en.wikipedia.org/wiki/Volume_of_an_n-ball#The_two-dimension_recursion_formula}{alternate derivation} of the volume recurrence by direct calculation (which uses $R$ where we write $r$) concludes with the integral
\begin{equation}
\label{eq:integral_recurrence}
\begin{split}
V_n(R) & = \int_0^\tau \int_0^R V_{n-2}\left(\sqrt{R^2 - r^2}\right) \,r\,dr\,d\theta \\
       & = \tau V_{n-2}(R) \left[-\frac{R^2}{n}\left(1 - \left(\frac{r}{R}\right)^2\right)^\frac{n}{2}\right]_{0}^{R} \\
       & = \frac{\tau R^2}{n} V_{n-2}(R),
\end{split}
\end{equation}
thus showing that the identification of $\tau$ as the ``recurrence constant'' isn't a coincidence---the recurrence constant and the circle constant really are one and the same:
\[
\begin{split}
\tau & = \mbox{circle constant} \\
     & = \mbox{recurrence constant} = 2^2\lambda.
\end{split}
\]
As a result, it is $\tau$, not $\lambda$, that provides the common thread tying together the two families of even and odd solutions, as illustrated by Joseph Lindenberg in \href{http://sites.google.com/site/taubeforeitwascool/}{Tau Before It Was Cool} (Figure~\ref{fig:Nspheres}).\footnote{\href{http://sites.google.com/site/taubeforeitwascool/}{Tau Before It Was Cool} actually writes the recurrence in terms of $2\pi$; the version shown in Figure~\ref{fig:Nspheres} was created for me by special request. As always, I am most grateful to Joseph Lindenberg for his continuing generosity and support.}

\begin{figure}
\begin{center}
\image{images/figures/Nspheres.png}
\end{center}
\caption{Surface area and volume recurrences.\label{fig:Nspheres}}
\end{figure}

When discussing general $n$-dimensional spheres, for convenience \linebreak we'll write the surface area and volume formulas in terms of $\lambda$ as in Eq.~\eqref{eq:surface_area_lambda} and Eq.~\eqref{eq:volume_lambda}, but for any given $n$ we'll express the results in terms of the recurrence constant $\tau$.

% subsubsection recurrences (end)


  % subsection volume_of_a_hypersphere (end)

  \subsection{Three families of constants} % (fold)
  \label{sec:three_families_of_constants}

Equipped with the tools developed in Section~\ref{sec:volume_of_a_hypersphere}, we're now ready to get to the bottom of $\pi$ and $\tau$. To complete the excavation, we'll use Eq.~\eqref{eq:surface_area_lambda} and Eq.~\eqref{eq:volume_lambda} to define two families of constants, and then use the definition of $\pi$ (Eq.~\eqref{eq:pi}) to define a third, thereby revealing exactly what is wrong with $\pi$.

First, we'll define a family of ``surface area constants'' $\tau_n$ by dividing \linebreak Eq.~\eqref{eq:surface_area_lambda} by $r^{n-1}$, the power of $r$ needed to yield a dimensionless constant for each value of~$n$:
\begin{equation}
\label{eq:surface_area_constants}
\tau_n \equiv \frac{S_n(r)}{r^{n-1}} = \frac{2^n\,\lambda^{\left\lfloor \frac{n}{2} \right\rfloor}}{(n-2)!!}
\end{equation}
Second, we'll define a family of ``volume constants'' $\sigma_n$ by dividing the volume formula Eq.~\eqref{eq:volume_lambda} by $r^n$, again yielding a dimensionless constant for each value of~$n$:
\begin{equation}
\label{eq:volume_constants}
\sigma_n \equiv \frac{V_n(r)}{r^n} = \frac{2^n\,\lambda^{\left\lfloor \frac{n}{2} \right\rfloor}}{n!!}.
\end{equation}
With the two families of constants defined in Eq.~\eqref{eq:surface_area_constants} and Eq.~\eqref{eq:volume_constants}, we can write the surface area and volume formulas (Eq.~\eqref{eq:surface_area_lambda} and Eq.~\eqref{eq:volume_lambda}) compactly as follows:
\[ S_n(r) = \tau_n\,r^{n-1} \]
and
\[ V_n(r) = \sigma_n\,r^n. \]
Because of the relation $V_n(r) = \int S_n(r)\,dr$, we have the simple relationship
\[
\sigma_n = \frac{\tau_n}{n}.
\]

Let us make some observations about these two families of constants. The family $\tau_n$ has an important geometric meaning: by setting $r=1$ in Eq.~\eqref{eq:surface_area_constants}, we see that each $\tau_n$ is the surface area of a unit $n$-sphere, which is also the angle measure of a full $n$-sphere. In particular, by writing $s_n(r)$ as the $n$-dimensional ``arclength'' equal to a fraction~$f$ of the full surface area~$S_n(r)$, we have
\[
\theta_n \equiv \frac{s_n(r)}{r^{n-1}} = \frac{f S_n(r)}{r^{n-1}} = f\left(\frac{S_n(r)}{r^{n-1}}\right) = f\tau_n.
\]
Here $\theta_n$ is simply the $n$-dimensional generalization of radian angle measure, and we see that $\tau_n$ is the generalization of ``one turn'' to $n$ dimensions, which explains why the 2-sphere (circle) constant $\tau_2 = 2^2\lambda = \tau$ leads naturally to the diagram shown in Figure~\ref{fig:tau_angles}. Furthermore, we learned in Section~\ref{sec:volume_of_a_hypersphere} that $\tau_2$ is also the ``recurrence constant'' for $n$-sphere surface areas and volumes.

Meanwhile, the $\sigma_n$ are the volumes of unit $n$-spheres. In particular, $\sigma_2$ is the area of a unit disk:
\[
\sigma_2 = \frac{\tau_2}{2} = \frac{\tau}{2}.
\]
This shows that $\sigma_2 = \tau/2 = 3.14159\ldots$ does have an independent geometric significance. Note, however, that \emph{it has nothing to do with circumferences or diameters}. In other words, \emph{$\pi = C/D$ is not a member of the family~$\sigma_n$}.

So, to which family of constants does $\pi$ naturally belong?
Let's rewrite Eq.~\eqref{eq:pi} in terms more appropriate for generalization to higher dimensions:
\[
\pi = \frac{C}{D} = \frac{S_2}{D^{2-1}}.
\]
We thus see that $\pi$ is naturally associated with surface areas divided by the power of the diameter necessary to yield a dimensionless constant. This suggests introducing a third family of constants~$\pi_n$:
\begin{equation}
\label{eq:diameter_constants}
\pi_n \equiv \frac{S_n(r)}{D^{n-1}}.
\end{equation}
We can express this in terms of the family $\tau_n$ by substituting $D = 2r$ in Eq.~\eqref{eq:diameter_constants} and applying Eq.~\eqref{eq:surface_area_constants}:
\[
\pi_n = \frac{S_n(r)}{D^{n-1}} = \frac{S_n(r)}{(2r)^{n-1}} =
\frac{S_n(r)}{2^{n-1}r^{n-1}} = \frac{\tau_n}{2^{n-1}}.
\]

We are now finally in a position to understand exactly what is wrong with $\pi$. The principal geometric significance of $3.14159\ldots$ is that it is the area of a unit disk. But this number comes from evaluating $\sigma_n = \tau_n/n$ when $n=2$:
\[
\sigma_2 = \frac{\tau_2}{2} = \frac{\tau}{2}.
\]
It's true that this equals $\pi_2$:
\[
\pi_2 = \pi = \frac{\tau_2}{2^{2-1}} = \frac{\tau}{2}.
\]
But this equality is a coincidence: it occurs only because $2^{n-1}$ happens to equal $n$ when $n=2$ (that is, $2^{2-1} = 2$). In all higher dimensions, $n$ and $2^{n-1}$ are distinct. In other words, \emph{the geometric significance of $\pi$ is the result of a mathematical pun}.

  % subsection three_families_of_constants (end)

\section{Conclusion}
\label{sec:conclusion}

Over the years, I have heard many arguments against the wrongness of $\pi$ and against the rightness of $\tau$, so before concluding our discussion allow me to answer some of the most frequently asked questions.

  \subsection{Frequently Asked Questions} % (fold)
  \label{sec:faq}

\begin{itemize}

  \item \textbf{Are you serious?} \\ Of course. I mean, I'm having fun with this, and the tone is occasionally lighthearted, but there is a serious purpose. Setting the circle constant equal to the circumference over the diameter is an awkward and confusing convention. Although I would love to see mathematicians change their ways, I'm not particularly worried about them; they can take care of themselves. It is the neophytes I am most worried about, for they take the brunt of the damage: as noted in Section~\ref{sec:circles_and_angles}, $\pi$ is a pedagogical disaster. Try explaining to a twelve-year-old (or to a thirty-year-old) why the angle measure for an eighth of a circle---one slice of pizza---is $\pi/8$. Wait, I meant $\pi/4$. See what I mean? It's madness---sheer, unadulterated madness.

  \item \textbf{How can we switch from $\pi$ to $\tau$?} \\ The next time you write something that uses the circle constant, simply say ``For convenience, we set $\tau = 2\pi$'', and then proceed as usual. (Of course, this might just prompt the question, ``Why would you want to do that?'', and I admit it would be nice to have a place to point them to. If only someone would write, say, a \emph{manifesto} on the subject\ldots) The way to get people to start using $\tau$ is to start using it yourself.

  \item \textbf{Isn't it too late to switch? Wouldn't all the textbooks and math papers need to be rewritten?} \\ No on both counts. It is true that some conventions, though unfortunate, are effectively irreversible. For example, Benjamin Franklin's choice for the signs of electric charges leads to the most familiar example of electric current (namely, free electrons in metals) being positive when the charge carriers are negative, and \emph{vice versa}---thereby cursing beginning physics students with confusing negative signs ever since.\footnote{The sign of the charge carriers couldn't be determined with the technology of Franklin's time, so this isn't his fault. It's just bad luck.} To change this convention \emph{would} require rewriting all the textbooks (and burning the old ones) since it is impossible to tell at a glance which convention is being used. In contrast, while \emph{redefining} $\pi$ is effectively impossible, we can switch from $\pi$ to $\tau$ on the fly by using the conversion \[ \pi \leftrightarrow \textstyle{\frac{1}{2}}\tau. \] It's purely a matter of mechanical substitution, completely robust and indeed fully reversible. The switch from $\pi$ to $\tau$ can therefore happen incrementally; unlike a redefinition, it need not happen all at once.

  \item \textbf{Won't using $\tau$ confuse people, especially students?} \\  If you are smart enough to understand radian angle measure, you are smart enough to understand $\tau$---and  why $\tau$ is actually \emph{less} confusing than $\pi$. Also, there is nothing intrinsically confusing about saying ``Let $\tau = 2\pi$''; understood narrowly, it's just a simple substitution. Finally, we can embrace the situation as a teaching opportunity: the idea that $\pi$ might be wrong is \emph{interesting}, and students can engage with the material by converting the equations in their textbooks from $\pi$ to $\tau$ to see for themselves which choice is better.

  \item \textbf{Does any of this really matter?} \\ Of course it matters. \emph{The circle constant is important.} People care enough about it to write entire books on the subject, to celebrate it on a particular day each year, and to memorize tens of thousands of its digits. I care enough to write a whole manifesto, and you care enough to read it. It's precisely because it \emph{does} matter that it's hard to admit that the present convention is wrong. (I mean, how do you break it to \href{https://www.guinnessworldrecords.com/world-records/most-pi-places-memorised}{Rajveer Meena}, a world-record holder, that he just recited 70,000 digits of one half of the true circle constant?) Since the circle constant is important, it's important to get it right, and we have seen in this manifesto that the right number is $\tau$. Although $\pi$ is of great \emph{historical} importance, the \emph{mathematical} significance of $\pi$ is that it is one-half $\tau$.

  \item \textbf{Why did anyone ever use $\pi$ in the first place?} \\ As notation, $\pi$ was popularized around 300 years ago by \href{https://en.wikipedia.org/wiki/Leonhard_Euler}{Leonhard Euler} (based on the work of \href{https://en.wikipedia.org/wiki/William_Jones_(mathematician)}{William Jones}), but the origins of $\pi$-the-number are lost in the mists of time. I suspect that the convention of using $C/D$ instead of $C/r$ arose simply because it is easier to \emph{measure} the diameter of a circular object than it is to measure its radius. But that doesn't make it good mathematics, and I'm surprised that Archimedes, who famously \href{http://itech.fgcu.edu/faculty/clindsey/mhf4404/archimedes/archimedes.html}{approximated the circle constant}, didn't realize that $C/r$ is the more fundamental number. I'm even more surprised that Euler didn't correct the problem when he had the chance; unlike Archimedes, Euler had the benefit of modern algebraic notation, which (as we saw starting in Section~\ref{sec:circles_and_angles}) makes the underlying relationships between circles and the circle constant abundantly clear. Incredibly, Euler actually used the symbol $\pi$ to mean \emph{both} $C/D$ and $C/r$ \href{https://en.wikipedia.org/wiki/Pi#Adoption_of_the_symbol_%CF%80}{at different times}! What a shame that he didn't standardize on the more convenient choice.

  \item \textbf{Why does this subject interest you?} \\ First, as a truth-seeker I care about correctness of explanation. Second, as a teacher I care about clarity of exposition. Third, as a hacker I love a nice hack. Fourth, as a student of history and of human nature I find it fascinating that the absurdity of $\pi$ was lying in plain sight for centuries before anyone seemed to notice. Moreover, many of the people who missed the true circle constant are among the most rational and intelligent people ever to live. What else might be staring us in the face, just waiting for us to discover it?

  \item \textbf{Are you, like, a crazy person?} \\ That's really none of your business, but no. Apart from occasionally wearing \href{https://en.wikipedia.org/wiki/Vibram_FiveFingers}{unusual shoes}, I am to all external appearances normal in every way. You would never guess that, far from being an ordinary citizen, I am in fact a notorious mathematical propagandist.

  \item \textbf{But what about puns?} \\ We come now to the final objection. I know, I know, ``$\pi$ in the sky'' is so very clever. And yet, $\tau$ itself is pregnant with possibilities. $\tau$ism tells us: it is not $\tau$ that is a piece of $\pi$, but $\pi$ that is a piece of $\tau$---one-half $\tau$, to be exact. The identity $e^{i\tau} = 1$ says: ``\emph{Be one with the $\tau$.}'' And though the observation that ``\emph{A rotation by one turn is~1}'' may sound like a $\tau$-tology, it is the true nature of the $\tau$. As we contemplate this nature to seek the \href{https://en.wikipedia.org/wiki/Tao}{way} of the $\tau$, we must remember that $\tau$ism is based on reason, not on faith: $\tau$ists are never $\pi$ous.

\end{itemize}

  % subsection faq (end)

  \subsection{Embrace the tau} % (fold)
  \label{sec:embrace_the_tau}

We have seen in \emph{The Tau Manifesto} that the natural choice for the circle constant is the ratio of a circle's circumference not to its diameter, but to its radius. This number needs a name, and I hope you will join me in calling it~$\tau$:
\[
\begin{split}
\mbox{circle constant} = \tau & \equiv \frac{C}{r} \\
                              & = 6.283185307179586\ldots
\end{split}
\]
The usage is natural, the motivation is clear, and the implications are profound. Plus, it comes with a really cool diagram (Figure~\ref{fig:tauism}). We see in Figure~\ref{fig:tauism} a movement through \emph{yang} (``light, white, moving up'') to $\tau/2$ and a return through \emph{yin} (``dark, black, moving down'') back to $\tau$.\footnote{The interpretations of yin and yang quoted here are from \emph{Zen Yoga: A Path to Enlightenment though Breathing, Movement and Meditation} by Aaron Hoopes.} Using $\pi$ instead of $\tau$ is like having \emph{yang} without \emph{yin}.

\begin{figure}
\begin{center}
\image{images/figures/tauism_rotated.pdf}
\end{center}
\caption{Followers of $\tau$ism seek the way of the $\tau$.\label{fig:tauism}}
\end{figure}

% If you ever hear yourself saying things like, ``Sometimes $\pi$ is the best choice, and sometimes it's $2\pi$'', stop and remember the words of \href{http://vihart.com/}{Vi Hart} in her wonderful \href{https://www.youtube.com/watch?v=jG7vhMMXagQ}{video about tau}: ``No! You're making excuses for $\pi$.'' It's time to stop making excuses.

  % subsection conclusion (end)

  \subsection{Tau Day} % (fold)
  \label{sec:tau_day}

\emph{The Tau Manifesto} first launched on \href{https://tauday.com/}{Tau Day}: June 28 (6/28), 2010. Tau Day is a time to celebrate and rejoice in all things mathematical.\footnote{Since 6 and 28 are the first two \href{https://en.wikipedia.org/wiki/Perfect_number}{\emph{perfect numbers}}, 6/28 is actually a \emph{perfect} day.} If you would like to receive updates about $\tau$, including notifications about possible future Tau Day events, please join the \emph{Tau Manifesto} mailing list below. And if you think that the circular baked goods on Pi Day are tasty, just wait---Tau Day has twice as much pi(e)!

%= <!-- insert_user_engagement -->

  \subsubsection{Acknowledgments} % (fold)
  \label{sec:acknowledgments}

I'd first like to thank \href{https://www.math.utah.edu/~palais}{Bob Palais} for writing ``$\pi$ Is Wrong!''. I don't remember how deep my suspicions about $\pi$ ran before I encountered that article, but ``$\pi$ Is Wrong!''\ definitely opened my eyes, and every section of \emph{The Tau Manifesto} owes it a debt of gratitude. I'd also like to thank Bob for his helpful comments on this manifesto, and especially for being such a good sport about it.

I've been thinking about \emph{The Tau Manifesto} for a while now, and many of the ideas presented here were developed through conversations with my friend Sumit Daftuar. Sumit served as a sounding board and occasional Devil's advocate, and his insight as a teacher and as a mathematician influenced my thinking in many ways.

I have also received encouragement and helpful feedback from several readers. I'd like to thank \href{https://www.youtube.com/watch?v=jG7vhMMXagQ}{Vi Hart} and \href{https://www.youtube.com/watch?v=3174T-3-59Q}{Michael Blake} for their amazing $\tau$-inspired videos, as well as Don ``Blue'' McConnell and Skona Brittain for helping make $\tau$ part of geek culture (through the \href{http://tauclock.com/}{time-in-$\tau$ iPhone app} and the \href{http://www.sbcrafts.net/clocks/}{tau clock}, respectively). The pleasing interpretation of the yin-yang symbol used in \emph{The Tau Manifesto} is due to a suggestion by \href{http://www.harremoes.dk/Peter/}{Peter Harremo\"{e}s}, who (as noted above) has the rare distinction of having independently proposed using $\tau$ for the circle constant. Another pre--\emph{Tau Manifesto} $\tau$ist, \href{https://sites.google.com/site/taubeforeitwascool/}{Joseph Lindenberg}, has also been a staunch supporter, and his enthusiasm is much-appreciated. I got several good suggestions from \href{https://christopherolah.wordpress.com/about-me}{Christopher Olah}, particularly regarding the geometric interpretation of Euler's identity, and Section~\ref{sec:eulerian_identities} on Eulerian identities was inspired by an excellent suggestion from Timothy ``Patashu'' Stiles. \href{http://www.blahedo.org/blog/archives/001083.html}{Don Blaheta} anticipated and inspired some of the material on hyperspheres, and \href{http://spikedmath.com/forum/viewtopic.php?f=30&t=147\#p1577}{John Kodegadulo} put it together in a particularly clear and entertaining way. Then Jeff Cornell, with his observation about the importance of $\tau/4$ in this context, shook my faith and blew my mind.

Finally, I'd like to thank \href{https://techiferous.com/}{Wyatt Greene} for his extraordinarily helpful feedback on a pre-launch draft of the manifesto; among other things, if you ever need someone to tell you that ``pretty much all of [now deleted] section~5 is total crap'', Wyatt is your man.


    % subsubsection acknowledgments (end)

    \subsubsection{About the author} % (fold)
    \label{sec:about_the_author}

    % subsection about_the_author (end)

\emph{The Tau Manifesto} author \href{https://www.michaelhartl.com/}{Michael Hartl} is an educator, author, and entrepreneur. He is cofounder and principal author at \href{https://www.learnenough.com}{Learn Enough} and the \href{https://www.railstutorial.org/}{\emph{Ruby on Rails Tutorial}}. Previously, he taught theoretical and computational physics at the \href{https://www.caltech.edu/}{California Institute of Technology} (Caltech), where he received the \href{https://www.michaelhartl.com/ascit/awards2000.html}{Lifetime Achievement Award for Excellence in Teaching} and served as Caltech's editor for \href{https://www.feynmanlectures.caltech.edu/}{\emph{The Feynman Lectures on Physics}}. He is a graduate of \href{https://college.harvard.edu/}{Harvard College}, has a \href{https://thesis.library.caltech.edu/1940/}{Ph.D. in Physics} from \href{https://www.caltech.edu/}{Caltech}, and is an alumnus of the \href{https://ycombinator.com/}{Y~Combinator} entrepreneur program.

Michael is ashamed to admit that he knows $\pi$ to 50 decimal places---\href{\#fig-futurama_video}{ap\-prox\-imately 48 more than Matt Groening}. To atone for this, he has memorized \href{https://www.wolframalpha.com/input/?i=N[2+Pi,+53]}{52 decimal places} of $\tau$.

    \subsubsection{Copyright} % (fold)
    \label{sec:copyright_and_license}

    \emph{The Tau Manifesto}. Copyright \copyright\ 2010--2018 by Michael Hartl. Ebook versions of \emph{The Tau Manifesto} are available for purchase through \href{https://sales.tauday.com/}{the Tau Manifesto sales site}. Please feel free to print out and distribute copies of \emph{The Tau Manifesto} for classroom or similar uses.

    % subsubsection copyright_and_license (end)
