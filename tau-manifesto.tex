\documentclass{article}
\usepackage{fancyvrb}
\usepackage{xcolor}
\usepackage{pygments}

\usepackage{polytexnic}
\begin{document}

\title{The Tau Manifesto}
\author{Michael Hartl}
\date{Tau Day, 2010}
\maketitle

\section{The circle constant} % (fold)
\label{sec:the_circle_constant}

Welcome to the \emph{Tau Manifesto}. This manifesto is dedicated to one of the most important numbers in mathematics, perhaps \emph{the} most important: the \emph{circle constant} relating the circumference of a circle to its linear dimension. For millennia, the circle has been considered the most perfect of shapes, and the circle constant captures the geometry of the circle in a single number. Of course, the traditional choice of circle constant is $\pi$---but, as mathematician \href{http://www.math.utah.edu/~palais/}{Bob Palais} notes in his delightful article ``$\pi$ Is Wrong!'',\footnote{Palais, Robert. ``$\pi$ Is Wrong!'', \emph{The Mathematical Intelligencer}, Volume~23, Number~3, 2001, pp.~7--8. (``$\pi$ Is Wrong!'' is available online at \href{http://www.math.utah.edu/~palais/pi.html}{Bob Palais' pi page}.)} $\pi$ \emph{is wrong}. It's time to set things right.

% Surely, defining this constant properly is a task that the great mathematicians of history should---nay, \emph{must}---have gotten right\ldots only they didn't. 

  \subsection{A modest proposal} % (fold)
  \label{sec:a_modest_proposal}

We begin repairing the damage wrought by $\pi$ by first understanding the notorious number itself. The traditional definition for the circle constant sets $\pi$ (pi) equal to the ratio of a circle's circumference to its diameter:\footnote{The symbol $\equiv$ means ``is defined as''.}


\[
  \pi \equiv \frac{C}{D} = 3.14159265\ldots
\]

\noindent Unfortunately, this definition is off by a factor of two in the denominator. Since a circle is the set of points a fixed distance---the \emph{radius}---from a given point, the most natural definition for the circle constant uses $r$ in place of $D$:

\[
  \mathrm{circle\ constant} = \frac{C}{r}.
\]

Although he argues persuasively in favor of the above definition, even the plucky Professor Palais can't quite bring himself to take the idea seriously. Indeed, his initial inclination is to \emph{redefine} $\pi$ itself:

\[
  \pi \equiv \frac{C}{r}.
\]

\noindent Given how deeply entrenched $\pi$ is in its current form, this redefinition is a complete nonstarter; it suggests, with a wistful tone, a ``\emph{wouldn't it have been nice if\ldots}'' sentiment that has no chance of real-world usage. Of course, using $\pi$ in the manner above would be hopelessly confusing, so ultimately Professor Palais does introduce a separate symbol for the circle constant, which he calls ``one turn''.  As we'll see, the description is prescient, but the symbol, lamentably, is absurd (\hyperref[fig:palais-tau]{Figure~}\ref{fig:palais-tau}).

\begin{figure}
\begin{center}
\includegraphics{images/figures/palais-tau.png}
\end{center}
\caption{The tragically absurd symbol for the circle constant in ``$\pi$ Is Wrong!''.\label{fig:palais-tau}}
\end{figure}

The \emph{Tau Manifesto} is dedicated to the proposition that the proper response to ``$\pi$ is wrong'' is ``No, \emph{really}.'' And the true circle constant deserves a proper name. The \emph{Tau Manifesto} proposes that this name---as I hope you will not be surprised to learn---should be the Greek letter $\tau$ (tau):

\[
  \tau \equiv \frac{C}{r} = 6.2831853071796586\ldots
\]

\noindent This symbol is neither confusing nor absurd. Throughout the rest of this manifesto, we will see that the \emph{number} $\tau$ is the correct choice, and we will show through usage (\hyperref[sec:the_number_tau]{Section~}\ref{sec:the_number_tau} and \hyperref[sec:circular_area]{Section~}\ref{sec:circular_area}) and by direct argumentation (\hyperref[sec:why_tau]{Section~}\ref{sec:why_tau}) that the \emph{letter} $\tau$ is a natural choice as well.

 \subsection{A powerful enemy} % (fold)

Before proceeding with the demonstration that $\tau$ is the natural choice for the circle constant, let us first acknowledge what we are up against---for there is a powerful conspiracy, centuries old, determined to propagate pro-$\pi$ propaganda. Entire \href{http://www.amazon.com/exec/obidos/ISBN=0802713327/parallaxproductiA/}{books} \href{http://www.amazon.com/Pi-Sky-Counting-Thinking-Being/dp/0198539568}{are} \href{http://www.amazon.com/exec/obidos/ISBN=0312381859/parallaxproductiA/}{written} extolling the virtues of $\pi$. (I mean, \href{http://www.amazon.com/exec/obidos/ISBN=0387989463/parallaxproductiA/}{\emph{books}}\emph{!}) And irrational devotion to $\pi$ has spread even to the highest levels of geekdom; for example, on ``$\pi$ Day'' 2010 \href{http://www.google.com/}{Google} \emph{changed its logo} to honor $\pi$  (\hyperref[fig:google-pi-day]{Figure~}\ref{fig:google-pi-day}).  

\begin{figure}
\begin{center}
\image{images/figures/google-pi-day.gif}
\end{center}
\caption{The Google logo on March 14, 2010 (``$\pi$ Day'').\label{fig:google-pi-day}}
\end{figure}

Meanwhile, some people memorize dozens, hundreds, even \emph{thousands} of digits of this mystical number. What kind of sad sack memorizes even 50 digits of $\pi$ (\hyperref[fig:futurama_video]{Figure~}\ref{fig:futurama_video})? 

\begin{figure}
\begin{center}
%= insert_futurama_video
\end{center}
\caption{\href{#sec:about_the_author}{Michael Hartl} proves \href{http://en.wikipedia.org/wiki/Matt_Groening}{Matt Groening} wrong by reciting $\pi$ to 50 decimal places.\footnote{This is an excerpt from a lecture given by \href{http://mathsci.appstate.edu/~sjg/}{Dr. Sarah Greenwald}, a professor of mathematics at \href{http://www.appstate.edu/}{Appalachian State University}. Dr. Greenwald uses math references from ``The Simpsons'' and especially ``Futurama'' to help her students get over their math anxiety; she is also the maintainer of the \href{http://www.mathsci.appstate.edu/~sjg/futurama/}{Futurama Math Page}. (By the way, \href{http://www.comedycentral.com/shows/futurama/index.jhtml}{Futurama} is back! New episodes are currently airing Thursday nights on Comedy Central at 10pm~/~9c. :-)}\label{fig:futurama_video}}
\end{figure}

Truly, proponents of $\tau$ face a mighty opponent. And yet, we have a powerful ally---for the truth is on our side.

% section the_most_important_number (end)

\section{The number tau} % (fold)
\label{sec:the_number_tau}

Since the diameter of a circle is twice its radius, numerically $\tau$ is simply $2\pi$. It is therefore of great interest to discover that, when browsing through the annals of mathematics, we find that the combination $2\pi$ occurs with astonishing frequency. For example, consider integrals over all angles in polar coordinates:

\[
  \int_0^{2\pi}\int_0^r f(r, \theta)\, dr d\theta.
\]

\noindent The upper limit of the $\theta$ integration is always $2\pi$. The same factor appears in the definition of the \href{http://en.wikipedia.org/wiki/Normal_distribution}{Gaussian (normal) distribution}:

\[
  \frac{1}{\sqrt{2\pi}\sigma}e^{-\frac{(x-\mu)^2}{2\sigma^2}}.
\]

\noindent It recurs in \href{http://en.wikipedia.org/wiki/Cauchy's_integral_formula}{Cauchy's integral formula},

\[
  f(a) = \frac{1}{2\pi i}\oint_\gamma\frac{f(z)}{z-a}\,dz,
\]

\noindent and again in the \href{http://en.wikipedia.org/wiki/Root_of_unity}{roots of unity}:\footnote{In this context, ``unity'' is simply the number 1.}

\[
  z^n = 1 \Rightarrow z = e^\frac{2\pi i}{n}.
\]

\noindent There are many more examples, and the lesson is clear: there's something seemingly inevitable about $2\pi$.

To get to the bottom of this mystery, we must return to first principles by considering the nature of the circle, and especially the nature of \emph{angle}. Although much of this material may be familiar, it pays to revisit it, for this is where the true understanding of $\tau$ begins.

  \subsection{The nature of angle} % (fold)
  \label{sec:the_nature_of_angle}

There is an intimate relationship between circles and \emph{angles}, as shown in \hyperref[fig:angle-arclength]{Figure~}\ref{fig:angle-arclength}. Since the concentric circles in \hyperref[fig:angle-arclength]{Figure~}\ref{fig:angle-arclength} have different radii, the lines in the figure cut off different lengths of arc (or \emph{arclengths}), but the angle~$\theta$ (theta) is the same in each case. In other words, the size of the angle does not depend on the radius of the circle used to define the arc. The principal task of angle measurement is to create a system that captures this radius-invariance.

\begin{figure}
\begin{center}
\image{images/figures/angle-arclength.png}
\end{center}
\caption{An angle $\theta$ with two concentric circles.\label{fig:angle-arclength}}
\end{figure}

Perhaps the most elementary angle system is \emph{degrees}, which arbitrarily breaks a circle into 360 equal parts. One result of this system is the set of special angles familiar to students of trigonometry (\hyperref[fig:degree-angles]{Figure~}\ref{fig:degree-angles}). 

\begin{figure}
\begin{center}
\image{images/figures/degree-angles.png}
\end{center}
\caption{Some special angles, in degrees.\label{fig:degree-angles}}
\end{figure}

A more fundamental system of angle measure involves a direct comparison of the arclength $s$ with the radius $r$. Although the lengths in \hyperref[fig:angle-arclength]{Figure~}\ref{fig:angle-arclength} differ, the arclength grows in proportion to the radius, so the \emph{ratio} of the arclength to the radius is the same in each case:

\[
s\propto r \Rightarrow \frac{s_1}{r_1} = \frac{s_2}{r_2}.
\]

\noindent This suggests the following definition of \emph{radian angle measure}:

\[ \theta \equiv \frac{s}{r}. \]

\noindent This definition has the required property of radius-invariance, and it leads to elegant formulas throughout mathematics. For example, the formula for the derivative of $\sin\theta$ is true only when $\theta$ is in radians:

\[
  \frac{d}{d\theta}\sin\theta = \cos\theta \mathrm{\ \ \ \ (true\ only\ when\ } \theta\mathrm{\ is\ in\ radians).}
\]

Naturally, the special angles in \hyperref[fig:degree-angles]{Figure~}\ref{fig:degree-angles} can be expressed in radians, and when you took high-school trigonometry you probably memorized the special values shown in \hyperref[fig:pi-angles]{Figure~}\ref{fig:pi-angles}. (I call this system of measure $\pi$-radians to emphasize that they are written in terms of $\pi$.)

\begin{figure}
\begin{center}
\image{images/figures/pi-angles.png}
\end{center}
\caption{Some special angles, in $\pi$-radians.\label{fig:pi-angles}}
\end{figure}

\begin{figure}
\begin{center}
\image{images/figures/angle-fractions.png}
\end{center}
\caption{The ``special'' angles are fractions of a full circle.\label{fig:angle-fractions}}
\end{figure}

\noindent Now, a moment's reflection shows that the so-called ``special'' angles are just particularly simple \emph{rational fractions} of a full circle, as shown in \hyperref[fig:angle-fractions]{Figure~}\ref{fig:angle-fractions}. This suggests revisiting the definition of radian angle measure, rewriting the arclength in terms of the fraction~\emph{f} of the full circumference:

\[ \theta = \frac{s}{r} = \frac{fC}{r} =  f\left(\frac{C}{r}\right) \equiv f\tau. \]

\noindent Notice how naturally $\tau$ falls out of this analysis. If you are a believer in $\pi$, I fear that the resulting diagram of special angles---shown in \hyperref[fig:tau-angles]{Figure~}\ref{fig:tau-angles}---will shake your faith to its very core. 

\begin{figure}
\begin{center}
\image{images/figures/tau-angles.png}
\end{center}
\caption{Some special angles, in radians.\label{fig:tau-angles}}
\end{figure}

Although there are many other arguments in $\tau$'s favor, \hyperref[fig:tau-angles]{Figure~}\ref{fig:tau-angles} may be the most striking. Indeed, upon comparing \hyperref[fig:tau-angles]{Figure~}\ref{fig:tau-angles} with \hyperref[fig:angle-fractions]{Figure~}\ref{fig:angle-fractions}, I consider it decisive. We also see from \hyperref[fig:tau-angles]{Figure~}\ref{fig:tau-angles} the genius of Bob Palais' description of the circle constant as ``one turn'': $\tau$ is the radian angle measure for one \emph{turn} of a circle. Moreover, note that with $\tau$ there is \emph{nothing to memorize}: the radian angle measure for an eighth of a circle is simply $\tau$/8, and so on. Finally, by comparing \hyperref[fig:pi-angles]{Figure~}\ref{fig:pi-angles} with \hyperref[fig:tau-angles]{Figure~}\ref{fig:tau-angles}, we see where those pesky factors of $2\pi$ come from: one turn of a circle is $1\tau$, but $2\pi$. Numerically they are equal, but conceptually they are quite distinct.

    \subsubsection{The ramifications} % (fold)
    \label{sec:the_ramifications}
    
    % subsubsection the_ramifications (end)

The unnecessary factors of $2$ arising from the use of $\pi$ are annoying enough by themselves, but far more serious is their tendency to \emph{cancel} when divided by any even number. The absurd results, such as $\pi$/2 for a \emph{quarter} circle, obscure the underlying relationship between angle measure and the circle constant. To those who maintain that it ``doesn't matter'' whether we use $\pi$ or $\tau$ when teaching trigonometry, I simply ask you to view \hyperref[fig:pi-angles]{Figure~}\ref{fig:pi-angles}, \hyperref[fig:angle-fractions]{Figure~}\ref{fig:angle-fractions}, and \hyperref[fig:tau-angles]{Figure~}\ref{fig:tau-angles} through the eyes of a child. You will see that, from the perspective of a beginner, \emph{using $\pi$ instead of $\tau$ is a pedagogical disaster.}

  \subsection{The circle functions} % (fold)
  \label{sec:the_circle_functions}

Although I consider the argument for $\tau$ now to be won, the game is still early, so we might as well run up the score a little. We begin by considering the important elementary functions $\sin\theta$ and $\cos\theta$ from the perspective of $\tau$. Known as the ``circle functions''  because they give the coordinates of a point on the \emph{unit circle} (\hyperref[fig:circle-functions]{Figure~}\ref{fig:circle-functions}), sine and cosine are the fundamental functions of trigonometry.

\begin{figure}
\begin{center}
\image{images/figures/circle-functions.png}
\end{center}
\caption{The circle functions are coordinates on the unit circle.\footnote{A ``unit'' in this context is the number 1, and a unit circle is a circle whose diameter---no, wait, \emph{radius}---is equal to one unit.}\label{fig:circle-functions}}
\end{figure}

Let's examine the graphs of the circle functions to better understand their behavior. You'll notice from \hyperref[fig:sine-with-tau]{Figure~}\ref{fig:sine-with-tau} and \hyperref[fig:cosine-with-tau]{Figure~}\ref{fig:cosine-with-tau} that both functions are \emph{periodic} with period $T$.\footnote{I swear I'm not stacking the deck here; $T$ really is the usual letter in this context.} As shown in \hyperref[fig:sine-with-tau]{Figure~}\ref{fig:sine-with-tau}, the sine function $\sin\theta$ starts at zero, reaches a maximum at a quarter period, passes through zero at a half period, reaches a minimum at three-quarters of a period, and returns to zero after one full period. Meanwhile, the cosine function $\cos\theta$ starts at a maximum, has a minimum at a half period, and passes through zero at one-quarter and three-quarters of a period  (\hyperref[fig:cosine-with-tau]{Figure~}\ref{fig:cosine-with-tau}). For reference, both figures show the value of $\theta$ (in radians) at each special point.

\begin{figure}
\begin{center}
\image{images/figures/sine-with-tau.png}
\end{center}
\caption{Important points for $\sin\theta$ in terms of the period $T$.\label{fig:sine-with-tau}}
\end{figure}

\begin{figure}
\begin{center}
\image{images/figures/cosine-with-tau.png}
\end{center}
\caption{Important points for $\cos\theta$ in terms of the period $T$.\label{fig:cosine-with-tau}}
\end{figure}

Of course, since sine and cosine both go through one full cycle during one turn of the circle, we have $T = \tau$, i.e., the circle functions have periods equal to the circle constant. As a result, the ``special'' values are utterly natural: the value of $\theta$ for a quarter of a period is just $\tau$/4, and so on. In fact, when making \hyperref[fig:sine-with-tau]{Figure~}\ref{fig:sine-with-tau}, I found myself wondering about the numerical value of $\theta$ for the zero of the sine function. Since the zero occurs after half a period, and since $\tau \approx 6.28$, a quick mental calculation led to the following result:

\[
  \theta_\mathrm{zero} = \frac{\tau}{2} \approx 3.14.
\]

\noindent That's right: I was astonished to discover that \emph{I had already forgotten that $\tau$/2 is sometimes called ``$\pi$''.} Perhaps this even happened to you just now. Welcome to my world.

  % subsection the_circle_functions (end)

% section radian_angle_measure (end)

   \subsection{Euler's identity} % (fold)
   \label{sec:euler_s_formula}

I would be remiss in this manifesto not to address \emph{Euler's identity}, ``the most beautiful equation in mathematics''. This identity involves \emph{complex exponentiation}, which is deeply connected both to the circle functions and to the geometry of the circle itself.

Depending on the route chosen, the following equation can be either be proved as a theorem or taken as a definition; either way, it is quite remarkable:

\[ e^{i\theta} = \cos\theta + i\sin\theta. \]

\noindent Known as \emph{Euler's formula} (after \href{http://en.wikipedia.org/wiki/Leonhard_Euler}{Leonard Euler}), this equation relates an exponential with imaginary argument to the circle functions sine and cosine and to the imaginary unit~$i$. Although justifying Euler's formula is beyond the scope of this manifesto, its provenance is above suspicion, and its importance is beyond dispute.

Evaluating Euler's formula at $\theta = \tau$ yields \emph{Euler's identity}:

\[ e^{i\tau} = 1. \]

\noindent In words, this equation makes the following fundamental observation: 

\begin{center}
\emph{The exponential of the imaginary unit times the circle constant is unity.}
\end{center}

\\

\noindent Since complex exponentials correspond to rotations in the complex plane, this can also be stated as follows:

\begin{center}
\emph{The complex exponential of the circle constant is one turn.}
\end{center}

\\

\noindent As in the case of radian angle measure, we see how natural the association is between $\tau$ and one turn of a circle.

    \subsubsection{Not the most beautiful equation} % (fold)

Of course, Euler's identity is traditionally written in terms of $\pi$:


\[ e^{i\pi} = -1. \]

\noindent But that minus sign is so ugly that the formula is almost always rearranged immediately, yielding the traditional version of Euler's identity:

\[ e^{i\pi} + 1 = 0. \]

\noindent (What's up with that minus sign? And why does ``the most beautiful equation in mathematics'' need rearranging? \emph{Move along, move along, these aren't the droids we're looking for\ldots}) At this point, the expositor usually makes some grandiose statement about how Euler's identity relates $0$, $1$, $e$, $i$, and $\pi$---sometimes called the ``five most important numbers in mathematics''. Alert readers might then complain that, because it's missing $0$, Euler's formula with $\tau$ relates only \emph{four} of those five. For any such whiners in the audience, let me note that, since $\sin\tau = 0$, we were already there:

\[ e^{i\tau} = 1 + 0. \]

\noindent This formula, without rearrangement, actually \emph{does} relate the five most important numbers in mathematics: $0$, $1$, $e$, $i$, and $\tau$.

\section{Circular area: the coup de gr\^{a}ce} % (fold)
\label{sec:circular_area}

If you arrived here as a $\pi$ believer, you must by now be questioning your faith. $\tau$ is so natural, its meaning so transparent---is there no example where $\pi$ shines through in all its radiant glory? A memory stirs---yes, there is such a formula---it is the formula for circular area! Behold:

\[ A = \pi r^2. \]

\noindent We see here $\pi$, unadorned, in one of the most important equations of mathematics---a formula first proved by \href{Archimedes}{Archimedes} himself. Order is restored! And yet, the name of this section sounds ominous\ldots If this equation is $\pi$'s crowning glory, how can it also be the \href{http://en.wikipedia.org/wiki/Coup_de_grace}{\emph{coup de gr\^{a}ce}}?


  \subsection{Quadratic forms} % (fold)
  \label{sec:quadratic_forms}

Let us examine this paragon of $\pi$, $A = \pi r^2$. We notice that it involves the diameter---no, wait, the \emph{radius}---raised to the second power. This makes it a simple \emph{quadratic form}. Such forms arise in many contexts, but as a \href{http://thesis.library.caltech.edu/1940/}{physicist} my favorite examples come from the elementary physics curriculum. Let us consider several in turn.

    \subsubsection{Falling in a uniform gravitational field} % (fold)
    \label{sec:falling_in_a_uniform_gravitational_field}

\href{http://en.wikipedia.org/wiki/Galileo_Galilei}{Galileo Galilei} found that the velocity of an object falling in a uniform gravitational field is proportional to the time fallen:

\[ v \propto t. \]

\noindent The constant of proportionality is the gravitational acceleration \emph{g}:

\[ v = g t. \]

\noindent Since velocity is the derivative of position, we can calculate the distance fallen by integration:

\[ y = \int v\,dt = \int_0^t gt\,dt = \textstyle{\frac{1}{2}} gt^2. \]


    \subsubsection{Potential energy in a linear spring} % (fold)
    \label{sec:potential_energy_in_a_linear_spring}

\href{http://en.wikipedia.org/wiki/Robert_Hooke}{Robert Hooke} found that the external force required to stretch a spring is proportional to the distance stretched:

\[ F \propto x. \]

\noindent The constant of proportionality is the spring constant $k$:\footnote{You may have seen this written as $F = -kx$. In this case, $F$ refers to the force exerted by the \emph{spring}. By Newton's third law, the external force discussed above is the \emph{negative} of the spring force.}

\[ F = k x. \]

\noindent The potential energy in a spring is then equal to the work done by the external force:

\[ U = \int F\,dx = \int_0^x kx\,dx = \textstyle{\frac{1}{2}} kx^2. \]

    \subsubsection{Energy of motion} % (fold)

\href{http://en.wikipedia.org/wiki/Isaac_Newton}{Isaac Newton} found that the force on an object is proportional to its acceleration:

\[ F \propto a. \]

\noindent The constant of proportionality is the mass $m$:

\[ F = m a. \]

\noindent The energy of motion, or \emph{kinetic energy}, is equal to the total work done in accelerating the mass to velocity $v$:

\[ K = \int F\,dx = \int ma\,dx = \int m\,\frac{dv}{dt}\,dx = \int m\, \frac{dx}{dt}\,dv = \int_0^v mv\,dv = \textstyle{\frac{1}{2}} mv^2. \]


  \subsection{A sense of foreboding} % (fold)
  \label{sec:a_sense_of_foreboding}

Having seen several examples of simple quadratic forms in physics, you may now have a sense of foreboding as we return to the geometry of the circle. This feeling is justified.

\begin{figure}
\begin{center}
\image{images/figures/circular-area.png}
\end{center}
\caption{Breaking down a circle into rings.\footnote{This is a physicist's diagram. A mathematician would probably use $\Delta r$, limits, and \href{http://en.wikipedia.org/wiki/Big_O_notation#Little-o_notation}{little-o notation}, an approach that is more rigorous but less intuitive.}\label{fig:circular-area}}
\end{figure}


As seen in \hyperref[fig:circular-area]{Figure~}\ref{fig:circular-area}, the area of a circle can be calculated by breaking it down into circular rings of length $C$ and width $dr$, where the area of each ring is $C\,dr$:

\[ dA = C\,dr. \]


\noindent Now, the circumference of a circle is proportional to its radius:

\[ C \propto r. \]

\noindent The constant of proportionality is $\tau$:

\[ C = \tau r. \]

\noindent The area of the circle is then the integral over all rings:

\[ A = \int dA = \int_0^r C\,dr = \int_0^r \tau r\,dr = \textstyle{\frac{1}{2}} \tau r^2. \]

If you were still a $\pi$ partisan at the beginning of this section, your head has now exploded. For we see that even in this case, where $\pi$ supposedly shines, in fact there is a missing factor of 2. Indeed, the original proof by Archimedes shows not that the area  of a circle is $\pi r^2$, but that it is equal to the area of a triangle with base $C$ and height $r$. Using the formula for triangular area then gives

\[
  A = \textstyle{\frac{1}{2}} bh = \textstyle{\frac{1}{2}}Cr = \textstyle{\frac{1}{2}}\tau r^2.
\]

\noindent There is simply no avoiding that factor of a half  (\hyperref[table:quadratic_forms]{Table~}\ref{table:quadratic_forms}).

\begin{table}
\begin{center}
\begin{tabular}{lcc}
Quantity & Symbol & Expression \\ \hline
Distance fallen & $y$ & $\textstyle{\frac{1}{2}}gt^2$ \\
Spring energy & $U$ & $\textstyle{\frac{1}{2}}kx^2$ \\
Kinetic energy & $K$ & $\textstyle{\frac{1}{2}}mv^2$ \\
Circular area & $A$ & $\textstyle{\frac{1}{2}}\tau r^2$
\end{tabular}
\end{center}
\caption{Some common quadratic forms.\label{table:quadratic_forms}}
\end{table}

    \subsubsection{Quod erat demonstrandum} % (fold)
    \label{sec:quod_erat_demonstrandum}
    
    % subsubsection quod_erat_demonstrandum (end)

We set out in this manifesto to show that $\tau$ is the true circle constant. Since the formula for circular area was just about the best argument $\pi$ had going for it, I'm going to go out on a limb here and say: \href{http://en.wikipedia.org/wiki/Q.E.D.}{Q.E.D.}

% section circular_area (end)

\section{Why tau?} % (fold)
\label{sec:why_tau}

The true test of any notation is usage; having seen $\tau$ used throughout this manifesto, I hope you need little convincing that it serves its role well. After reading a few sections, you may even have forgotten that $\tau$ was first introduced in this very document. But for a constant as fundamental as $\tau$ it would be nice to have some deeper reasons for our choice. Why not $\alpha$, or $\omega$? What's so special about $\tau$?

Allow me to offer the following arguments, in order of increasing strength:

\begin{enumerate}
  \item \emph{$\tau$ is available.} \\ Although $\tau$ is used for certain specific variables---e.g., \emph{shear stress} in mechanical engineering, \emph{torque} in rotational mechanics, and \emph{proper time} in special and general relativity---there is no universal conflicting usage.\footnote{With a global namespace, minor conflicts are inevitable, but it's OK: physicists, for example, manage to use $e$ both for the natural number and for the charge on an electron without causing apparent harm.} 
  
  \item \emph{$\tau$ resembles $\pi$.} \\ $\tau$ is typographically similar to $\pi$, thereby evoking the same notion of a circle constant.\footnote{Unfortunately, the number of ``legs'' isn't quite right: it would be poetic if we could write $\pi = 2\tau$, but it wasn't meant to be. Of course, in that case $\pi$ would have the right definition, and we wouldn't have this manifesto.}
  
  \item \emph{$\tau$ is one turn.} \\ We have seen that, geometrically speaking, $\tau$ represents one \emph{turn} of a circle---and you may have noticed that ``$\tau$'' and ``turn'' both start with a ``\emph{t}'' sound. Perhaps this is only a coincidence? Alas, there is no escape, for the root of the English word ``turn'' is the Greek word for ``lathe'': \emph{tornos}---or, as the Greeks would put it, \[ \tau \acute{o}\rho\nu o\varsigma. \] Your head has just exploded again.
\end{enumerate}

Looking at the first letter of that Greek lathe, I'm going to have to whip out my Latin again and say again:  \href{http://en.wikipedia.org/wiki/Q.E.D.}{\emph{quod erat demonstrandum}}.

  \subsection{Frequently Asked Questions} % (fold)
  \label{sec:faq}

Over the years, I have heard many arguments against the wrongness of $\pi$ and against the correctness of $\tau$,\footnote{Among other things, the article ``$\pi$ Is Wrong'' pops up on \href{http://news.ycombinator.com/news}{Hacker News} every once in a while, and I've been known to frequent the  \href{http://news.ycombinator.com/item?id=912082}{comments section}.} so before concluding our discussion allow me to address some of the most frequently asked questions.

\begin{itemize}

  \item \textbf{Are you serious?} \\ Of course. I mean, I'm having fun with this, and the tone is occasionally lighthearted, but there is a serious purpose. $\pi$ is \emph{really}, \emph{truly} wrong: setting the circle constant to the circumference over the diameter is a terribly awkward and confusing convention. Although I would love to see the mathematicians change their ways, I'm not particularly worried about them; they can take care of themselves. It is the neophytes I am most worried about, for they take the brunt of the damage: as noted in \hyperref[sec:the_nature_of_angle]{Section~}\ref{sec:the_nature_of_angle}, $\pi$ is a pedagogical disaster. Try explaining to a twelve-year-old (or to a thirty-year-old) why the angle measure for an eighth of a circle---one slice of pizza---is $\pi$/8. Wait---I meant $\pi$/4. See what I mean? It's madness---sheer, unadulterated madness.

  \item \textbf{How do I start using $\tau$ instead of $\pi$?} \\ The next time you do anything mathematical, simply say ``For convenience, we set $\tau = 2\pi$'', and then proceed as usual. (Of course, this might just prompt the question, ``Why would you want to do that?'' I admit it would be nice to have a place to point to. If only someone would write, say, a \emph{manifesto} on the subject\ldots) The way to get people to start using $\tau$ is to start using it yourself. 

  \item \textbf{Do you really expect people to use $\tau$?} \\ I would certainly love to see $\tau$ get traction outside of internet screeds, and perhaps this will happen, but the great part is that you don't have to wait for everyone else; you can just start by saying ``Let $\tau = 2\pi$'' and you're off to the races. 
  
  \item \textbf{Won't using $\tau$ confuse people?} \\ If you are smart enough to understand radian angle measure, you are smart enough to understand $\tau$---and why $\tau$ is better. Also, there's nothing confusing about ``Let $\tau = 2\pi$''; understood narrowly, it's just a simple substitution.

  \item \textbf{Wouldn't we have to rewrite all the textbooks?} \\ No. It is true that some conventions, though unfortunate, are effectively irreversible. For example, Benjamin Franklin's choice for the signs of electric charges leads to electric current being positive, even though the charge carriers themselves are negative---thereby cursing electrical engineers with confusing minus signs ever since. To change this convention \emph{would} require rewriting all the textbooks (and burning the old ones) since it is impossible to tell at a glance which convention is being used---minus signs having a tendency to cancel, and all. In contrast, we can switch from $\pi$ to $\tau$ ``at runtime'' (as programmers might say), since it's purely a matter of mechanical substitution, completely robust and indeed fully reversible: the conversion \[ \pi \leftrightarrow \textstyle{\frac{1}{2}}\tau \] allows us to change back and forth between the two on the fly. (Of course, once you realize the absurdity of $\pi$, such conversions should only be necessary for historical purposes---if there were not already a separate symbol for $\tau$/2, it seems unlikely that anyone would see fit to introduce one.).
  
  \item \textbf{Why does this subject interest you?} \\ First, as a truth-seeker I care about correctness of explanation. Second, as a pedagogue I care about clarity of exposition. Third, as a hacker I love a nice hack. Fourth, as a student of history and human nature I am amazed that the absurdity of $\pi$ and the beauty of $\tau$ were lying in plain sight for centuries. Moreover, the people who missed the true circle constant are among the most logical and intelligent people ever to live. What else might be staring us in the face, just waiting for us to discover it?
  
  \item \textbf{Are you, like, a crazy person? Or maybe an Aspie?} \\ That's really none of your business, but no and no. Apart from my \href{http://www.vibramfivefingers.com/}{unusual shoes}, I am to all external appearances normal in every way. You would never guess that, far from being an ordinary citizen, I am in fact a notorious mathematical propagandist.
  
  \item \textbf{What about puns?} \\ We come now to the final objection. I know, I know, ``$\pi$ in the sky'' is so very clever. And yet, $\tau$ itself is pregnant with possibilities. $\tau$ism says: it is not $\tau$ that is a piece of $\pi$, but $\pi$ that is a piece of $\tau$---one-half $\tau$, to be exact. This is the true nature of the~$\tau$. But we must remember that $\tau$ism is based on reason, not on faith: $\tau$ists are never $\pi$ous.

\end{itemize}

  % subsection faq (end)

\section{Conclusion}

There is one question I skipped in the section above, but it is one that I must address---indeed, it may be the question \emph{most} frequently asked by those who have studied the arguments and have absorbed their implications. It might even be rattling around inside your brain at this very moment---a question, together with a desperate assertion:

\begin{itemize}
  \item \textbf{Who cares whether we use $\pi$ or $\tau$? It doesn't really matter.}
\end{itemize}

\noindent If you have arrived at this question, it means that you have accepted the truth of the arguments against $\pi$, and you are now ready to take the final leap.

Of course it matters. \emph{The circle constant is important.} It is a \href{http://en.wikipedia.org/wiki/Theory_of_forms}{Platonic Form}; it is woven into the fabric of the Universe. People care enough about it to write entire \emph{books} on the subject, to celebrate it on a particular day every year, and to memorize \emph{tens of thousands if its digits}. I care enough to write a whole manifesto, and you care enough to read it. It's precisely because it matters so much that it's hard to admit that the present convention is wrong. I mean, how do you break it to \href{http://en.wikipedia.org/wiki/Lu_Chao}{Lu Chao}, the present world-record holder, that he just recited 67,891 digits of one half of the true circle constant? 

The argument that ``it makes no difference'' which circle constant we use is a \href{http://en.wikipedia.org/wiki/Denial}{psychological attempt to avoid the truth}; indeed, I have found that the people who assert most strenuously that it makes no difference are exactly the ones with the deepest emotional attachment to $\pi$. These are the people who celebrate $\pi$ Day and wax rhapsodic about Euler's identity. (There is no judgment here; I used to be one of them.) But you can't have it both ways; either defining the proper circle constant matters, in which case you must come to terms with the deficiencies of $\pi$ and the superiority of $\tau$, or it doesn't, in which case you have to explain how you ever came to own that $\pi$ shirt hanging in your closet.

\emph{It is time to let $\pi$ go.} I loved $\pi$ once; I really did. Today, although I bear it no ill will, I can no longer pretend that it deserves special recognition. It is true that $\pi$'s threads are woven throughout the fabric of mathematics, but this is for \emph{historical} reasons, not for logical ones. $\pi$ is not one of the most important numbers in mathematics; it is one half of one of the most important numbers. The mathematical significance of $\pi$ is that it is one-half $\tau$.

    \subsection{Change is gonna come} % (fold)
    \label{sec:change_is_gonna_come}
    
    % subsection change_is_gonna_come (end)

So, if $\pi$ is wrong, why haven't we changed over to something else? The technical barriers to switching are low; as noted in \hyperref[sec:faq]{Section~}\ref{sec:faq}, changing conventions is a matter of simple substitution. Moreover, switching is a \href{http://en.wikipedia.org/wiki/Private_good}{\emph{private good}}; you (or your students) can benefit from the proper convention without having to convert everyone else. We might even view it as a teaching opportunity: ``\emph{Exercise for the student:} convert the equations in the textbook from $\pi$ to $\tau$ and discover that even mathematical geniuses can get something really important really wrong.''

Inertia is certainly a part of the explanation, but that can't be the full story, either, for $\pi$ continues to be used even by those who have recognized its deficiencies. The real reason even these people use $\pi$ is that they are \emph{afraid}---they know that, since \emph{other} people believe in the primacy of $\pi$, they may be mocked and ridiculed if they question the $\pi$ orthodoxy. And if not for that, they will be mocked and ridiculed for \emph{even caring}. I say, \emph{stop being afraid}. Those who mock you for questioning $\pi$ are wrong, and those who mock you for caring are hypocrites. So what are you afraid of? No one ever got burned at the stake for questioning $\pi$---not in this century, anyway. :-)

Of course, there is one more reason we haven't switched away from $\pi$: the lack of a good candidate to replace it. As we have seen in this manifesto, the true circle constant is the ratio of the circumference not to the diameter, but to the radius. This number needs a name, and I hope you will join me in calling it ``$\tau$'':

\[
  \mathrm{circle\ constant} = \tau \equiv \frac{C}{r}.
\]

\noindent The usage is natural, the motivation is clear, and the implications are profound. Plus, the logo is really cool (\hyperref[fig:tauism]{Figure~}\ref{fig:tauism}).

\begin{figure}
\begin{center}
\image{images/figures/tauism.png}
\end{center}
\caption{Followers of $\tau$ism seek the way of the $\tau$.\label{fig:tauism}}
\end{figure}

\subsection{Tau Day} % (fold)
\label{sec:tau_day}

Inspired by the impressive (though lamentable) success of ``$\pi$ Day'', the \emph{Tau Manifesto} was launched on June 28, 2010---i.e., 6/28, or $\tau$ Day. And of course the manifesto itself lives at \href{http://tauday.com/}{tauday.com}.

$\tau$ Day is $\tau$ism's holiday, its holy day---it is a day for celebrating and rejoicing in all things mathematical and true.\footnote{Indeed, 6/28 is a \emph{perfect} day---for 6 and 28 are the first two \href{http://en.wikipedia.org/wiki/Perfect_number}{\emph{perfect numbers}}.} For $\tau$ Day 2010, one way you can celebrate is by spreading word of this manifesto. For $\tau$ Day 2011, I'm hoping for bigger things. ;-) And in the mean time, I have several exciting $\tau$-related projects planned. If you would like to receive updates about $\tau$ and the \emph{Tau Manifesto}, including notifications about any possible $\tau$ Day events, you should join the \emph{Tau Manifesto} mailing list below. (I promise not to spam you; that would violate the spirit of the $\tau$. :-)

Thank you for reading the \emph{Tau Manifesto}. I hope you enjoyed reading it as much as I enjoyed writing it. And I hope even more that you have come to embrace the true circle constant: not $\pi$, but $\tau$.

%= insert_signup_form

    \subsubsection{About the author} % (fold)
    \label{sec:about_the_author}
    
    % subsection about_the_author (end)

\emph{Tau Manifesto} author \href{http://www.michaelhartl.com/}{Michael Hartl} is an educator and entrepreneur. He is the founder of the \href{http://www.railstutorial.org/}{Ruby on Rails Tutorial} project, which teaches web development using \href{http://www.rubyonrails.org/}{Ruby on Rails}. Previously, he taught theoretical and computational physics at  \href{http://www.caltech.edu/}{Caltech}, where he received the Lifetime Achievement Award for Excellence in Teaching. He is a graduate of \href{http://college.harvard.edu/}{Harvard College} and has a \href{http://thesis.library.caltech.edu/1940/}{Ph.D. in Physics} from the \href{http://www.caltech.edu/}{California Institute of Technology}.

Michael is ashamed to admit that he knows 50 digits of $\pi$---approximately 48 more than Matt Groening. To make up for this, he is currently memorizing 52 digits of $\tau$.\footnote{This is so that I know more digits of $\tau$ than I do of $\pi$, and $\tau$ doesn't round off right if you truncate after 51 digits. But you probably figured that out already. By the way, when I say ``digits'', I really mean ``decimal places''. But you probably figured that out, too.}
\end{document}

